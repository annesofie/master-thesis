

\chapter*{Sammendrag}
\addcontentsline{toc}{chapter}{Sammendrag}%

Forfatteren mener at mikro-oppgaver har et uoppdaget potensiale rundt geografiske oppgaver. Dette er bakgrunnen for denne masteroppgaven. Mikro-oppgaver er en stor oppgave delt opp i mange små oppgaver slik at de lettere kan fordeles ut. Oppgavens kompleksitet reduseres også betraktelig ved å dele den i mindre biter. Denne masteroppgaven viser at mikro-oppgaver er mye brukt i halv-automatiske prosesser, der både maskiner og mennesker er involvert. Metoden er blant annet mye brukt i maskinlæring, der mennesker lager treningsdatasettene og retter opp der algoritmen har klassifisert feil. Store geografiske oppgaver, som for eksempel en import, involverer ofte mennesker i prosessen. 

Denne masteroppgaven undersøker om mikro-oppgaver, som omhandler geografisk data, kan løses av alle mennesker, uavhengig av bakgrunnen deres. Oppgaven tester også hvor mange elementer man bør plassere i mikro-oppgavene for at de skal fullføres raskest mulig og med best mulig kvalitet. Et eksperiment ble utviklet for å hente inn data for å gjøre hypotesetester på. Eksperimentet ble implementert i en webapplikasjon som ble distribuert til ulike mailtråder og i GeoForum sine kommunikasjonskanaler. Eksperimentets deltakere gjennomførte tre oppgaver som hver inneholdt de samme to mikro-oppgavene. De tre oppgavene varierte antall elementer hver mikro-oppgave inneholdt. I de to mikro-oppgavene gjorde deltakerne først en \textit{bakgrunnskart} analyse hvor de skulle trykke på det bygningsfotavtrykket som passet best til en vist bygning og i den andre evaluerte deltakerne meta-informasjon av bygninger.

Den innhentede dataen ble brukt i statistiske analyser. Analysene resulterte i at det var en statistisk ulikhet i tid bruks på å fullføre mikro-oppgavene mellom erfarne og uerfarne deltakere. De uerfarne deltakerne brukte kortere tid enn uerfarne. Kvaliteten på de tre oppgavene var det ingen forskjell mellom deltakerne. Om kvalitet på oppgaven er veldig viktig anbefaler forfatteren å bruke færrest mulig elementer i hver mikro-oppgave. Tidsmessig vil det være en fordel med mer enn et element per mikro-oppgave. Masteroppgaven konkluderer med at det er mulig å bruke mikro-oppgaver i store geografiske oppgaver, og så lenge man har en introduksjonsdel som viser hvordan mikro-oppgavene skal løses kan alle, uavhengig av bakgrunn og utdannelse, løse disse. 
