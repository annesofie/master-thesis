

\chapter*{Sammendrag}
\addcontentsline{toc}{chapter}{Sammendrag}%
	
	
Sammendrag på norsk

Forfatteren mener at mikro oppgaver har et uoppdaget potensiale rundt geografiske oppgaver. Mikro oppgaver er en stor oppgave delt opp i mange små oppgaver slik at de lettere kan fordeles ut. Oppgavens kompleksitet reduseres betraktelig ved å dele den i mindre biter. Master oppgaven viser at mikro oppgaver er mye brukt der man har en halv-automatisk prosess, der både maskiner og mennesker må være inkludert. Metoden er blant annet mye brukt i maskinlæring, der mennesker lager treningsdatasettene og retter opp der algoritmen har klassifisert feil. Store geografiske oppgaver, som en import, trenger ofte mennesker i prosessen. Med det mener forfatteren at bruken av mikro oppgaver kan utnyttes der. 

Denne master oppgaven undersøker om mikro oppgaver som omhandler geografisk data kan løses av alle mennesker, uanhengig av bakgrunnen deres. Oppgaven tester også hvor mange elementer man bør plassere i mikro oppgavene for at de skal fullføres raskest mulig og med best mulig kvalitet. Et eksperiment ble dermed utviklet for å hente inn data. Eksperimentet ble implementert i en webapplikasjon som ble distribuert til ulike mail tråder og geoforum sine kommunikasjonskanaler. Eksperimentets deltakers gjennomførte tre oppgaver som hver inneholdt de samme to mikro oppgavene. De tre oppgavene varierte antall elementer hver micro oppgave inneholdt. Den innhentede dataen ble brukt i analyser. Analysene resulterte i at det var en statistisk ulikhet i kvalitet mellom erfarne og uerfarne deltakere. De uerfarne deltakerne hadde flere riktige enn uerfarne. Tid brukt på oppgaven var det ingen forskjell mellom deltakerne. Om kvalitet på oppgaven er veldig viktig anbefaler forfatteren å bruke færrest mulig elementer i hver mikro oppgave. Tidsmessig vil det være en fordel med fler enn et element per mikro oppgave. 
