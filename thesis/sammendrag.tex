

\chapter*{Sammendrag}
\addcontentsline{toc}{chapter}{Sammendrag}%

Prosessen der en deler opp en større oppgave inn i mindre deler for å lettere distribuere oppgavene og redusere kompleksiteten kalles mikro-oppgaver. Forfatteren mener at mikro-oppgaver har et potensiale rundt geografiske oppgaver som idag ikke blir utnyttet. Det er lite, om noe,  forskning på hvordan slike oppgaver lar seg løse gjennom denne oppdelings prosessen. 
%Denne masteroppgaven viser at mikro-oppgaver er mye brukt i halv-automatiske prosesser, der både maskiner og mennesker er involvert. Metoden er blant annet mye brukt i maskinlæring, der mennesker lager treningsdatasettene og retter opp der algoritmen har klassifisert feil. Store geografiske oppgaver, som for eksempel en import, involverer ofte mennesker i prosessen. 

Masteroppgaven undersøker om mikro-oppgaver, som omhandler geografisk data, kan løses av alle mennesker, uavhengig av bakgrunnen deres og om antall elementer i hver mikro-oppgave påvirker kvaliteten på resultatet. Et eksperiment ble utviklet for å undersøke hvordan personer, både eksperter og uefarne, løser geografiske oppgaver delt inn i mikro-oppgaver. Den innhentede dataen ble brukt i statistiske analyser for å teste ulike hypoteser. Masteroppgaven konkluderer med at det er mulig å bruke mikro-oppgaver i store geografiske oppgaver, og så lenge man har en introduksjonsdel som viser hvordan mikro-oppgavene skal løses kan alle, uavhengig av bakgrunn og utdannelse, løse disse.

%Eksperimentet ble implementert i en webapplikasjon som ble distribuert til ulike mailtråder og i GeoForum sine kommunikasjonskanaler. Eksperimentets deltakere gjennomførte tre oppgaver som hver inneholdt de samme to mikro-oppgavene. De tre oppgavene varierte antall elementer hver mikro-oppgave inneholdt. I de to mikro-oppgavene gjorde deltakerne først en \textit{bakgrunnskart} analyse hvor de skulle trykke på det bygningsfotavtrykket som passet best til en vist bygning og i den andre evaluerte deltakerne meta-informasjon av bygninger.

%Den innhentede dataen ble brukt i statistiske analyser. Analysene resulterte i at det var en statistisk ulikhet i tid bruks på å fullføre mikro-oppgavene mellom erfarne og uerfarne deltakere. De uerfarne deltakerne brukte kortere tid enn uerfarne. Kvaliteten på de tre oppgavene var det ingen forskjell mellom deltakerne. Om kvalitet på oppgaven er veldig viktig anbefaler forfatteren å bruke færrest mulig elementer i hver mikro-oppgave. Tidsmessig vil det være en fordel med mer enn et element per mikro-oppgave. Masteroppgaven konkluderer med at det er mulig å bruke mikro-oppgaver i store geografiske oppgaver, og så lenge man har en introduksjonsdel som viser hvordan mikro-oppgavene skal løses kan alle, uavhengig av bakgrunn og utdannelse, løse disse. 
