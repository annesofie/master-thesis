\chapter*{DAIM page}

\section*{Background}
Micro-tasking is an Internet phenomenon that has increased in popularity over the last years. The method is used for solving large tasks that can be divided into many smaller ones (micro-tasks).  This involves both the use of computers and a large number of people. Importing geospatial data over large areas can be considered as a large task. This is a task that can be divided into smaller ones, for example by dividing into smaller areas.

\section*{Task Description}

This Master thesis will have emphasis on the data validation and conflict handling part of the import of geospatial data. These processes are too complicated to do fully automatic through scripts, and the thesis investigates how micro-tasking can be relevant approach to the problem.  
Specific tasks:
\begin{itemize}
	\item Study related literature
	\item Make a web-based experiment in order to answer the following questions:
	\begin{itemize}
		\item What are the number of objects optimal within a task to get it completed as quickly as possible?
		\item Does the quality of the work vary between the different tasks given?
		\item Do amateurs manage to do the tasks?
	\end{itemize}
	\item Explore the micro-tasking methods usage potential within geospatial data
	\item Can other process that requires human interaction make advantage of this method?
\end{itemize}

%The micro-tasking method is becoming more and more popular. Companies like Amazon develop micro-tasking web applications where people can earn money by doing micro-tasks for others. The method is used for tasks that involve both use of technology and a large number of people. By using the micro-tasking methodology, this thesis aims to study how people solves micro-tasks within geospatial data imports, which is a very complex and large process. 

%This study will have an emphasis on the data validation and conflict handling part of the import. These parts are complicated to do fully automatic through scripts. By varying the number of objects to solve at a time, adding rewards on some tasks, among other factors, the study will hopefully find a significant approach to prefer when using the micro-tasking method within geospatial data. What are the number of objects optimal within a task to get it completed as quickly as possible? Does the quality of the work vary between the different tasks given? Do amateurs manage to do the tasks? Do rewards have an impact on how the tasks are solved? 

%This thesis will also explore the micro-tasking methods usage potential within geospatial data. Can other organizations doing a process that needs humans to interfere take advantage of this method? An example is OpenStreetMap, who has taken good advantage of the method both in mapping and import projects.

\section*{Administrative/guidance}
The work on the Master Thesis starts on January 18th, 2017

The thesis report as described above shall be submitted digitally in DAIM at the latest at June 18th, 2017

External supervisor
Atle Frenvik Sveen, Norkart AS

Supervisors at NTNU and professor in charge:
Terje Midtbø
