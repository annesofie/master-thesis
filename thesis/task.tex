\chapter*{DAIM page}

\section*{Background}
HEI

\section*{Task Description}

The micro-tasking method is becoming more and more popular. Companies like Amazon develop micro-tasking web applications where people can earn money by doing micro-tasks for others. The method is used for tasks that involve both use of technology and a large number of people. By using the micro-tasking methodology, this thesis aims to study how people solves micro-tasks within geospatial data imports, which is a very complex and large process. 

This study will have an emphasis on the data validation and conflict handling part of the import. These parts are complicated to do fully automatic through scripts. By varying the number of objects to solve at a time, adding rewards on some tasks, among other factors, the study will hopefully find a significant approach to prefer when using the micro-tasking method within geospatial data. What are the number of objects optimal within a task to get it completed as quickly as possible? Does the quality of the work vary between the different tasks given? Do amateurs manage to do the tasks? Do rewards have an impact on how the tasks are solved? 

This thesis will also explore the micro-tasking methods usage potential within geospatial data. Can other organizations doing a process that needs humans to interfere take advantage of this method? An example is OpenStreetMap, who has taken good advantage of the method both in mapping and import projects.

\textbf{Specific tasks:}
\begin{itemize}
	\item Study related literature
	\item Do a micro-tasking survey
	\item Examine how many elements are optimal when creating geospatial micro-tasks
\end{itemize}