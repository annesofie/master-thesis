\chapter{Proposed sections}

\section{Future work}

Create a survey to test how accurate both experienced and inexperienced participants digitize buildings from aerial images. Can use FKB as the correct polygon and compare it with the drawn polygon from participants. 

Do a study with reward. Compare reward and not reward geo tasks. Do they solve the tasks better with reward? "A reward can be provided for merely participating in the task. The reward can also be provided as a prize for submitting the best solution or one of the best solutions. Thus, the reward can provide an incentive for members of the community to complete the task as well as to ensure the quality of the submissions."  %https://www.google.com/patents/US9305263

\section{Usage potential}
Systems are exploiting the people's physical presence in an environment more, they are more location dependent. This can be particularly important when seeking to improve geospatial data quality [\citep{Meier2013}, p. 323]. "For instance, UrbanMatch (Celino et al. 2012a ) is a mobile location based game that uses player’s familiarity with a city to link photos with points of interest in the city. Players are shown points of interest and known images from a trusted source (e.g. OpenStreetMap) and asked if photos from an untrusted source (e.g. Flickr) might also relate to the point of interest ".  

\citep{Meier2013}:  "As the previous sections show there is a lot of potential for AR systems to use HC to provide content, and to support processing in other ways. However there has been
little research to date combining AR and HC systems. In this section we review the first research efforts in this area. " 

\citep{Meier2013} "Lastly, there is huge untapped potential in leveraging the “cognitive surplus” available in massively multiplayer online games to process humanitarian microtasks during disasters. The online game “League of Legends,” for example, has 32 million players every month and three million on any given day. Over 1 billion hours are spent playing League of Legends every month. Riot Games, the company behind League of Legends is even paying salaries to select League of Legend players. Now imagine if users of the game were given the option of completing microtasks in order to acquire additional virtual currency, which can buy better weapons, armor, etc. Imagine further if users were required to complete a microtask in order to pass to the next level of the game. Hundreds of millions of humanitarian microtasks could be embedded in massively multiplayer online games and instantaneously completed. Maybe the day will come when kids whose parents tell them to get off their computer game and do their homework will turn around and say: “Not now, Dad! I’m microtasking crisis information to help save lives in Haiti!” "