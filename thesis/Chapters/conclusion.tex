\chapter{Conclusion}

We conclude that it is not necessary to only use experienced individuals when solving geospatial micro-tasks, especially if the platform publishing the tasks are designed well and has a thorough introduction session. 
%This is statistically proven in the analysis from \ref{sec:t-test_result}. 
Results in this thesis show that when partitioning a large task into smaller micro-tasks, the quality of the results will increase with fewer elements to handle in each task. Time spent to complete each task is not significantly affected by the number of elements.
%, but the overall time used to solve multiple micro-tasks is longer compared to how much 
We claim that fewer elements in each task can lower the total number of completed micro-tasks because the worker needs to spend more time switching to the next micro-task and fetching new elements. 