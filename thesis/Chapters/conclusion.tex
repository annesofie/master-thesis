\chapter{Conclusion}
The results from this thesis can be used as a guidance on how to break down a large geospatial task. If quality is the key factor, the large task must be divided so that the micro-tasks contains one element. The downside of this approach is the time spent on not task specific operations. If the worker needs to click and wait for the next element to load, the workers will spend unnecessary time. If quality mechanisms already are implemented it is possible to break the task into larger chunks (containing, i.e., three or six elements). The number of mistakes will probably increase with the number of elements present in the micro-tasks, but if the quality mechanisms are well developed, the errors will be handled. Benefits of dividing each task into larger chunks, containing more than one element, is that the worker will most likely spend less time in total doing the micro-tasks. We would not recommend chunks containing more than six elements unless one uses well-developed quality mechanisms. 

We conclude that it is not necessary to only use experienced individuals when solving geospatial micro-tasks, especially if the platform publishing the tasks are well designed and has a thorough introduction session. 
%This is statistically proven in the analysis from \ref{sec:t-test_result}. 
Results in this thesis show that when partitioning a large task into smaller micro-tasks, the quality of the results will increase with fewer elements to handle in each task. Time spent to complete each task is not significantly affected by the number of elements.
%, but the overall time used to solve multiple micro-tasks is longer compared to how much 
% We claim that fewer elements in each task can lower the total number of completed micro-tasks because the worker needs to spend more time switching to the next micro-task and fetching new elements. 