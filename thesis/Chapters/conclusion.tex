\chapter{Conclusion}

We conclude that it is not necessary to only use experienced individuals in geospatial micro-tasks, especially if the interface supporting the tasks are designed well and has a thorough training session. This is proven in the analysis from \ref{sec:t-test_result}. An overall rule when partitioning a large task into small micro-tasks is the fewer elements in each task the higher quality will the results have. Time to complete each task is not effected on the number of elements, but the overall time is natural longer when one has to shift to next micro-task after each completed element. This can lower the number of completed micro-tasks because the worker needs to spent more time on clicking next task and fetching new elements. 