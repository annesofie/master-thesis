\chapter{Conclusion}

This thesis amplifies the assumption that micro-tasking works well together with geospatial data. We conclude that it is not necessary to only use experienced individuals when solving geospatial micro-tasks, especially if the platform publishing the tasks are well designed and provides a training session. Having the ability to use all humans, independent of background, makes it easier to reach a considerable larger mass of people to solve the micro-tasks. We also see an indication that the quality of the task results will increase with fewer elements in each micro-task. Time spent to complete the micro-tasks is not significantly affected by the number of elements, but the time spent in total will probably increase with fewer elements. Therefore, the method used by Los Angeles building import can be another approach to how elements in micro-tasks can be managed. This method needs further research.

The results from this thesis can be used as guidance on how to break down a large geospatial task. If quality is the key factor, the large task must be divided so that the micro-tasks contains few elements. The downside of having too few elements in the micro-tasks is the time spent on not task specific operations. If the worker needs to click and wait for the next element to load, the workers will spend unnecessary time. If quality mechanisms already are implemented it is possible to break the task into larger chunks (containing, i.e., six elements). The number of mistakes will probably increase with the number of elements present in the micro-tasks, but if the quality mechanisms are well developed, the errors will be handled. Benefits of dividing each task into larger chunks are that the worker will most likely spend less time in total doing the micro-tasks. We would not recommend chunks containing an uneven number of elements, like in the New York building import. 

The author would recommend organizations to exploit micro-tasking when it is possible to break the large geospatial task into smaller parts. It is shown that the method makes imports of large datasets in OpenStreetMap easier. The method is also exploited in machine-learning, combining the human computation with the power of computers. Humans manage to evaluate and validate geometric layers, shown through question one and also interpret meta information, shown through question two in the experiment. The theory of which geospatial tasks that are most suitable to solve though micro-tasking is not known yet. This thesis indicates that task design and training sections are almost as important as the content of the tasks.

%This is statistically proven in the analysis from \ref{sec:t-test_result}. 
%, but the overall time used to solve multiple micro-tasks is longer compared to how much 
% We claim that fewer elements in each task can lower the total number of completed micro-tasks because the worker needs to spend more time switching to the next micro-task and fetching new elements. 

