\section{Survey results}\label{sec:survey_results}
This section will first introduce the gathered data and divide the data into samples. Key-values for each sample is listed in tables. The section will test each sample for the normal distribution assumption using D'Agostino and Pearson test and equal variance assumption using Levene's test. When the assumptions in each sample is tested, they are used to answer hypothesis leading to an answer to the research questions listed in the introduction. After each subsection a table containing a summary of the test results is presented. This will hopefully make it easier to get an overview of which analyses is completed in the subsection.   

\subsection{Gathered data}\label{sec:gathereddata}
The gathered data is analyzed on the two dependent variables: 1) total time used to complete each task and 2) number of correctly chosen elements per task. Both variables sum the participants time spent and correct elements on question one and question two together. The gathered data is also divided by the two independent variable pairs: 1) experienced- and inexperienced participants and 2) the three survey tasks. Combining the dependent and independent variables creates the foundation of the 22 different samples in this thesis. The samples are listed in the tables in this section. Sample mean $\overline{x}$, sample median, standard deviation of $\overline{x}$, standard error of $\overline{x}$, minimum and maximum in each sample is included. The sample ID is also listed in the tables and is referred to in the analysis of the data, to easier distinguish which sample is used in which analysis. In all the samples, results from the training task and from participants that were disturbed during the task are removed. We examine four different divisions of the gathered data in the following section.
% The tables in this section, (\ref{sec:gathereddata}), are task results from all participants and all three tasks, excluding the training task. Task results with a total time longer than 2160 seconds are filtered out. This is to remove four outliers that spend more than twice the approximated time (average time on the survey was 1080 seconds in the pilot test). These 4 participants also answered that they were disturbed during the test.

%The gathered data are in the first subsection (\ref{sec:alltasks}) divided into experienced and inexperienced. In subsection \ref{sec:taskdivided_all} the data is divided by the three tasks, containing all participants. Section \ref{sec:taskdivided_experienced}  and \ref{sec:taskdivided_inexperienced} divides the data by the three tasks and also in experienced and inexperienced participants. 

%Removed all participants that said they was distracted. 26 task results was removed, 10 inexperienced and 18 experienced results.

\subsubsection{All  experienced and inexperienced participants}\label{sec:alltasks}

Table \ref{tab:totaltime_all} and \ref{tab:totalcorrect_all} are samples containing task results from all experienced and inexperienced participants. The result is grouped by the two dependent variables, total time and the number of correctly chosen elements.

\begin{table}[H]
	\centering
	\begin{tabular}{l|l|l|l}
		\textit{Total time per task (seconds) } & All  & Experienced & Inexperienced \\ 
		Sample ID &   & 1  & 2   \\ \hline
		Number of observations & 429    & 229    & 200   \\
		Sample mean $\overline{x}$     & 170.32 & 177.65  & 161.94     \\
		Sample median  & 155.00 & 158.00  & 154.00  \\
		Standard deviation of $\overline{x}$  & 82.19  & 88.24  & 73.99   \\
		Standard error of $\overline{x}$  & 3.98  & 5.83 & 5.23  \\
		Minimum in sample & 38.00  & 52.00  & 38.00     \\
		Maximum in sample & 657.00 & 657.00  & 529.00    \\ \hline
	\end{tabular}
	\caption[Total time, all participants]{Total time spent on each task}
	\label{tab:totaltime_all}
\end{table}

\begin{table}[H]
	\centering
	\begin{tabular}{l|l|l|l}
		\textit{Correct elements per task } & All  & Experienced & Inexperienced \\ 
		Sample ID &   & 3  & 4   \\ \hline
		Number of observations & 429    & 229  & 200   \\
		Sample mean $\overline{x}$   & 9.82 & 9.81  & 9.83  \\
		Sample median & 10.00 & 10.00 & 10.00 \\
		Standard deviation of $\overline{x}$   & 1.52  & 1.53  &  1.51 \\
		Standard error of $\overline{x}$   & 0.07  & 0.10 &  0.11 \\
		Minimum in sample & 4.00 & 5.00  &  4.00  \\
		Maximum in sample  & 12.00 & 12.00  & 12.00  \\ \hline
	\end{tabular}
	\caption[Correct elements, all participants]{Number of correctly chosen elements per task}
	\label{tab:totalcorrect_all}
\end{table}

\subsubsection{All participants, divided by task}\label{sec:taskdivided_all}

In table \ref{tab:totaltime_tasks} and \ref{tab:totalcorrect_tasks} the task results are\ divided by the three different tasks, grouped by the dependent variables. Task A is the task that served the participants with one and one elements. Task B is the task that served the participants with three and three elements, and task C gave all six elements at the same time. 

\begin{table}[H]
	\centering
	\begin{tabular}{l|l|l|l}
		\textit{Total time per task (seconds)} & Task A & Task B & Task C \\ 		
		Sample ID & 5  & 6  & 7    \\ \hline
		Number of observations & 146    & 142      & 141     \\
		Sample mean $\overline{x}$  & 166.38  &  172.25   &   172.48  \\
		Sample median & 150.00  &  155.50  & 157.00  \\
		Standard deviation of $\overline{x}$   & 84.57  & 84.21  & 77.95   \\
		Standard error of $\overline{x}$   & 7.00 & 7.07 & 6.56 \\
		Minimum in sample    & 47  & 50 &   38   \\
		Maximum in sample   & 657 & 492  & 529 \\ \hline
	\end{tabular}
	\caption[Total time, divided by task]{Total time divided by task}
	\label{tab:totaltime_tasks}
\end{table}

\begin{table}[H]
	\centering
	\begin{tabular}{l|l|l|l}
		\textit{Correct elements per task} & Task A & Task B & Task C\\ 
		Sample ID & 8  & 9  & 10   \\ \hline
		Number of observations & 146    & 142     & 141        \\
		Sample mean $\overline{x}$ & 10.19  &  9.71  &   9.55   \\
		Sample median & 11.00 &  10.00  &  10.00   \\
		Standard deviation of $\overline{x}$ & 1.43  & 1.53 & 1.52    \\
		Standard error of $\overline{x}$ & 0.12 &  0.13 & 0.13  \\
		Minimum in sample  & 5.00  & 5.00  &   4.00  \\
		Maximum in sample  & 12.00 & 12.00  & 12.00 \\ \hline
	\end{tabular}
	\caption[Correct elements, divided by task]{Number of correctly chosen elements divided by task}
	\label{tab:totalcorrect_tasks}
\end{table}

\subsubsection{Experienced participants, divided by task}\label{sec:taskdivided_experienced}

In the tables in this section, only task results from experienced participants are included, and the result is also divided by the three tasks, grouped by the dependent variables.

\begin{table}[H]
	\centering
	\begin{tabular}{l|l|l|l}
		\textit{Total time per task} & Task A & Task B & Task C \\ 
		Sample ID & 11  & 12  & 13   \\ \hline
		Number of observations & 77  & 80   & 77  \\
		Sample mean $\overline{x}$  & 173.04  &  176.70  &  181.06   \\
		Sample median  & 158.00  &  156.00  &  165.00  \\
		Standard deviation of $\overline{x}$ & 96.76  & 86.13  & 79.70   \\
		Standard error of $\overline{x}$ & 11.03 & 9.63 & 9.08  \\
		Minimum in sample   & 57.00  & 52.00 &  53.00  \\
		Maximum in sample  & 657.00 & 492.00  & 463.00 \\ \hline
	\end{tabular}
	\caption[Total time, divided by task, only experienced]{Experienced total time per task, divided by task}
	\label{tab:totaltime_tasks_experienced}
\end{table}

\begin{table}[H]
	\centering
	\begin{tabular}{l|l|l|l}
		\textit{Correct elements per task} & Task A & Task B & Task C \\ 
		Sample ID & 14 & 15  & 16   \\ \hline
		Number of observations & 77    & 80      &  77  \\
		Sample mean $\overline{x}$ & 10.29  &  9.66  &  9.48   \\
		Sample median & 11.00  &  10.00  &  10.00  \\
		Standard deviation of $\overline{x}$ & 1.32  & 1.64  & 1.47   \\
		Standard error of $\overline{x}$ & 0.15  & 0.18  & 0.17   \\
		Minimum in sample & 7.00 & 5.00 &  5.00 \\
		Maximum in sample  & 12.00 & 12.00  & 12.00 \\ \hline
	\end{tabular}
	\caption[Correct elements, divided by task, only inexperienced]{Experienced number of correct elements per task, divided by task}
	\label{tab:totalcorrect_tasks_experienced}
\end{table}

\subsubsection{Inexperienced participants, divided by task}\label{sec:taskdivided_inexperienced}
In this section, the task results from only inexperienced participants are included, and the result is also divided into the three survey tasks. 
Table \ref{tab:totaltime_tasks_inexperienced} is the total time variable and \ref{tab:totalcorrect_tasks_inexperienced} the number of correctly chosen elements. 

\begin{table}[H]
	\centering
	\begin{tabular}{l|l|l|l}
		\textit{Total time per task (seconds)} & Task A & Task B & Task C \\ 
		Sample ID & 17 & 18 & 19 \\ \hline
		Number of observations & 71    & 64  & 65   \\
		Sample mean $\overline{x}$  & 158.30  &  165.69  &  162.23  \\
		Sample median & 148.00  &  154.50  &  154.00  \\
		Standard deviation of $\overline{x}$  & 67.57 & 80.93 & 74.53  \\
		Standard error of $\overline{x}$  & 8.02  & 10.12 & 9.24  \\
		Minimum in sample & 47.00 & 50.00 &  38.00 \\
		Maximum in sample & 487.00 & 455.00  & 529.00  \\ \hline
	\end{tabular}
	\caption[Total time, inexperienced per task]{Inexperienced total time per task, divided by task}
	\label{tab:totaltime_tasks_inexperienced}
\end{table}

\begin{table}[H]
	\centering
	\begin{tabular}{l|l|l|l}
		\textit{Correct elements per task} & Task A & Task B & Task C \\ 
		Sample ID & 20 & 21 & 22 \\ \hline
		Number of observations & 71 & 64  & 65 \\
		Sample mean $\overline{x}$  & 10.07  &  9.78 &  9.61  \\
		Sample median  & 10.00  & 10.00  &  10.00  \\
		Standard deviation of $\overline{x}$  & 1.54  & 1.38  & 1.57   \\
		Standard error of $\overline{x}$  & 0.18 & 0.17 & 0.19  \\
		Minimum in sample  & 5.00 & 6.00 &  4.00  \\
		Maximum in sample  & 12.00 & 12.00  & 12.00 \\ \hline
	\end{tabular}
	\caption[Correct elements, inexperienced per task]{Inexperienced number of correct elements per task, divided by task}
	\label{tab:totalcorrect_tasks_inexperienced}
\end{table}
\vspace{0.5cm}

\subsection{Normality tests}\label{sec:normality_results}
%To check if a two-sample t-test (subsection \ref{sec:t-test}) and \textit{ANOVA}-test (subsection \ref{sec:anova}) can be used, the samples need to be tested if they are normally distributed or not. Both tests assume normally distributed samples. 
The samples need to be tested to see if they follow a normal distribution. This test is important. It determined if a parametric or a non-parametric test should be used when including the different samples in the hypothesis tests. The section about normal testing (\ref{sec:normaltesting}) concluded that the D'Agostino and Person normality test should be used in this thesis. A visual interpretation of histograms will also be a part of the normality test. D'Agostino-Pearson uses the following hypothesis:\newline

\centerline{$H_{0}$: The data follows the normal distribution} 
\centerline{$H_{A}$: The data does not follow the normal distribution}


\subsubsection[Sample 1 and 2]{Experienced and inexperienced participants, total time variable}\label{sec:sample1,2_normresult}
This section will test if sample 1 and 2 (table \ref{tab:totaltime_all}) follow a normal distribution. Sample 1 and 2 will be used to determine if there are a significant difference in time spent on the tasks between experienced and inexperienced participants. 

A visual interpretation of histogram \ref{fig:totaltimeexclude4_experienced} and \ref{fig:totaltimeexclude4_inexperienced} gives an indication that sample 1 and 2 does not follow the normal distribution. Both histograms are positively skewed (figure \ref{fig:skew}). Samples involving time measurements are rarely normally distributed. This is because the samples will always be skewed since it is impossible to have negative time and there will always be a limit to how fast a participant can finish the task. 

\begin{figure}[H]
	\centering
	\begin{subfigure}[b]{0.48\textwidth}
		\centering
		\includegraphics[width=\linewidth]{../../thesis-statisticmethods/statistic_analysis/figures/experienced_participants/normalplot/totaltime_exclude4}
		\caption{Sample 1 - Experienced}
		\label{fig:totaltimeexclude4_experienced}
	\end{subfigure}
	\begin{subfigure}[b]{0.48\textwidth}
		\centering
		\includegraphics[width=\linewidth]{../../thesis-statisticmethods/statistic_analysis/figures/inexperienced_participants/normalplot/totaltime_exclude4}
		\caption{Sample 2 - Inexperienced}
		\label{fig:totaltimeexclude4_inexperienced}
	\end{subfigure}
\caption{Histograms with normal distribution fit with samples containing total time to complete each task}
\label{fig:sample1,2_histo_original}
\end{figure}

D'Agostino and Pearson tests confirmed the visual interpretation with a significance level of 5\%. Both samples obtained p-values lower than the significance level, and the null hypothesis is rejected. Sample 1 and 2 do not follow the normal distribution with a confidence level of 95\%. \\[0.5cm]

\begin{center}
	\begin{tcolorbox}[width=0.80\textwidth]
		\centering
				\textit{D'Agostino and Pearson normality test}\\
				Significance level: 5\%  \\[0.5cm]
	
				Sample 1\\
				P-value: $3.874 * 10^{-22}$\\
				The p-value is lower than the significance level (0.05), the null hypothesis is \underline{rejected} and $H_A$ accepted.\\[0.5cm]
				
				Sample 2\\
				P-value: $2.574 * 10^{-21}$ \\
				The p-value is lower than the significance level (0.05), the null hypothesis is \underline{rejected} and $H_A$ accepted.\\[0.5cm]
	\end{tcolorbox} 
\end{center}
%Boc-Cox Transformation? https://docs.scipy.org/doc/scipy-0.19.0/reference/generated/scipy.stats.boxcox.html https://www.isixsigma.com/tools-templates/normality/dealing-non-normal-data-strategies-and-tools/
In both sample 1 and 2, the p-value was significantly lower than the significance level of 5\%. Data transformations are commonly used tools to improve the normality of a samples distributions, but there are many types of data transformations. \cite{Osborne2010} claim that almost all tests, even non-parametric tests, benefit from improving the normality of the samples, especially when the normality test is significantly denied. Typical traditional transformations are square root, inverse or converting to logarithmic scales  \citep{Osborne2010}. 

A Box-Cox power transformation (Box-Cox) is used in this thesis. This transformation can only be used on positive data. The data gathered in this thesis will never be below zero, so this is not a concern. Box-Cox takes the idea of having a range of power transformations (i.e., square root $x^{\frac{1}{2}}$, inverse $x^{-1}$) available to improve the effectiveness of normalizing and variance equalizing for both positively- and negatively-skewed variables \citep{Osborne2010}. This transformation will always use the appropriate conversion to be maximally effective in moving each sampled data towards normality. This is the reason why this thesis will use the Box-Cox transformation.

Sample 1 and 2 after a Box-Cox power transformation is shown in histogram \ref{fig:totaltimeboxcoxtransformation_experienced} and \ref{fig:totaltimeboxcoxtransformedtitle_inexperienced}. A visual inspection gives a good indication that the transformed data follows a normal distribution after the transformation. The skewness looks approximately zero.

\begin{figure}[H]
	\centering
	\begin{subfigure}[b]{0.48\textwidth}
		\centering
		\includegraphics[width=\linewidth]{../../thesis-statisticmethods/statistic_analysis/figures/experienced_participants/normalplot/totaltime_boxcox_transformation}
		\caption[Experienced, Box-Cox]{Sample 1 - Experienced}
		\label{fig:totaltimeboxcoxtransformation_experienced}
	\end{subfigure}
	\begin{subfigure}[b]{0.48\textwidth}
		\centering
		\includegraphics[width=\linewidth]{../../thesis-statisticmethods/statistic_analysis/figures/inexperienced_participants/normalplot/totaltime_boxcox_transformed_title}
		\caption[Inexperienced, Box-Cox]{Sample 2 - Inexperienced}
		\label{fig:totaltimeboxcoxtransformedtitle_inexperienced}
	\end{subfigure}
\caption{Histograms with normal distribution fit after Box-Cox power transformation}
\end{figure}

The transformed data is applied to the D'Agostino and Pearson test. This test confirms the visual analysis, since both sample 1 and sample 2 follow a normal distribution after the Box-Cox with a confidence level of 95\%. The calculated p-value is larger than the significance level of 5\%. \\[0.3cm] 

\begin{center}
	\begin{tcolorbox}[width=0.8\textwidth]
		\centering
		\textit{D'Agostino and Pearson normality test}\\
		(After Box-Cox transformation) \\
		Significance level: 5\%  \\[0.5cm]
		
		Sample 1: Experienced, total time per task\\
		P-value: $0.849$\\
		The p-value is higher than the significance level (0.05), the null hypothesis is \underline{accepted}. \\[0.5cm]
		
		Sample 2: Inexperienced, total time per task \\ %*Her bruker jeg data hvor was interupted er fjernet
		P-value: $0.0623$ \\
		The p-value is higher than the significance level (0.05), the null hypothesis is \underline{accepted}. \\[0.5cm]
	\end{tcolorbox}
\end{center}

The assumption that sample 1 and sample 2 follows a normal distribution is now accepted and the transformed data can be used in parametric methods.

\subsubsection[Sample 3 and 4]{Experienced and inexperienced participants, number of correctly chosen elements variable}\label{sec:correct_ex_inex}
%DIscrete variables http://stattrek.com/probability-distributions/discrete-continuous.aspx
This section will test if sample 3 and 4 (table \ref{tab:totalcorrect_all}) are drawn from a normal distribution. These samples will be used to test if there are any difference in the number of correctly chosen elements per task between experienced and inexperienced participants. 

A visual analysis of the samples histogram \ref{fig:correctelementswasnotinterupter_ex} and \ref{fig:correctelementswasnotinterupted_inex}, we see a good indication that sample 3 and 4 are not drawn from a normal distribution. Both are clearly negatively skewed. 

\begin{figure}[h!]
	\centering
	\begin{subfigure}[b]{0.48\textwidth}
		\centering
		\includegraphics[width=\linewidth]{../../thesis-statisticmethods/statistic_analysis/figures/experienced_participants/normalplot/correct_elements_was_not_interupter}
		\caption{Sample 3 - Experienced}
		\label{fig:correctelementswasnotinterupter_ex}
	\end{subfigure}
	\begin{subfigure}[b]{0.48\textwidth}
		\centering
		\includegraphics[width=\linewidth]{../../thesis-statisticmethods/statistic_analysis/figures/inexperienced_participants/normalplot/correct_elements_was_not_interupted}
		\caption{Sample 4 - Inexperienced}
		\label{fig:correctelementswasnotinterupted_inex}
	\end{subfigure}
	\caption{Histograms with normal distribution fit with samples containing the number of correctly chosen elements}
\end{figure}

D'Agostino and Pearson normality test confirm our visual analysis. Both samples accept the alternative hypothesis with p-values ($0.00443$, $0.00013$) lower than the significance level ($0.05$). The null hypothesis is rejected and $H_A$ accepted for sample 3 and 4.

Sample 3 and 4 is Box-Cox power transformed because the null hypothesis was rejected. After the transformation, a new D'Agostino and Pearson normality test was performed. Both samples also failed this test and the alternative hypothesis ($H_A$) is accepted. Hypothesis including sample 3 and 4 need to be tested with non-parametric methods. 

\subsubsection[Sample 5, 6 and 7]{All participants divided by task, total time variable}\label{sec:task123_time_normaltest}

In this section sample 5, 6, and 7 (table \ref{tab:totaltime_tasks}) is normal distribution tested. These samples will be used to test whether there is a significant difference between the three tasks when considering the total time variable. 

A visual analysis of the three histograms in figure \ref{fig:totaltimeallnotinterupted_task1}, \ref{fig:totaltimeallnotinterupted_task2} and \ref{fig:totaltimeallnotinterupted_task3} show a positive skewness, just like the histograms in figure \ref{fig:sample1,2_histo_original}. This gives an indication that the three samples do not follow the normal distribution. 

\begin{figure}[h!]
	\centering
	\begin{subfigure}[b]{0.3\textwidth}
		\centering
		\includegraphics[width=\linewidth]{../../thesis-statisticmethods/statistic_analysis/figures/task1/totaltime_all_not_interupted}
		\caption{Sample 5 - Task A}
		\label{fig:totaltimeallnotinterupted_task1}
	\end{subfigure}
	\begin{subfigure}[b]{0.3\textwidth}
		\centering
		\includegraphics[width=\linewidth]{../../thesis-statisticmethods/statistic_analysis/figures/task2/totaltime_all_not_interupted}
		\caption{Sample 6 - Task B}
		\label{fig:totaltimeallnotinterupted_task2}
	\end{subfigure}
	\begin{subfigure}[b]{0.3\textwidth}
		\centering
		\includegraphics[width=\linewidth]{../../thesis-statisticmethods/statistic_analysis/figures/task3/totaltime_all_not_interupted}
		\caption{Sample 7 - Task C}
		\label{fig:totaltimeallnotinterupted_task3}
	\end{subfigure}
	\caption{Histogram with normal distribution fit - sample with total time per task}
\end{figure}

The D'Agostino and Pearson normality test agreed with the visual analysis. Obtained p-values for all three samples ($2.39 * 10^{-24}$, $2.57 * 10^{-9}$, and $1.71 * 10^{-11}$) are smaller than the significance level (0.05), and the null hypothesis is rejected. The samples do not follow the normal distribution with a confidence interval of 95\%.\\[0.2cm]

Because the null hypothesis was rejected, the samples are Box-Cox power transformed. Histograms of each sample after the transformation is shown in figure \ref{fig:totaltimeallnotinteruptedboxcox_task1}, \ref{fig:totaltimeallnotinteruptedboxcox_task2} and \ref{fig:totaltimeallnotinteruptedboxcox_task3}. A visual analysis of the histograms gives a good indication that the transformed data is approximately normal distributed. The histograms have a skewness of approximately zero. 

\begin{figure}[h!]
	\centering
	\begin{subfigure}[b]{0.3\textwidth}
		\centering
		\includegraphics[width=\linewidth]{../../thesis-statisticmethods/statistic_analysis/figures/task1/totaltime_all_not_interupted_boxcox}
		\caption{Sample 5 - Task A}
		\label{fig:totaltimeallnotinteruptedboxcox_task1}
	\end{subfigure}
	\begin{subfigure}[b]{0.3\textwidth}
		\centering
		\includegraphics[width=\linewidth]{../../thesis-statisticmethods/statistic_analysis/figures/task2/totaltime_all_not_interupted_boxcox}
		\caption{Sample 6 - Task B}
		\label{fig:totaltimeallnotinteruptedboxcox_task2}
	\end{subfigure}
	\begin{subfigure}[b]{0.3\textwidth}
		\centering
		\includegraphics[width=\linewidth]{../../thesis-statisticmethods/statistic_analysis/figures/task3/totaltime_all_not_interupted_boxcox}
		\caption{Sample 7 - Task C}
		\label{fig:totaltimeallnotinteruptedboxcox_task3}
	\end{subfigure}
	\caption{Histogram with normal distribution fit after Box-Cox transformation, total time variable}
\end{figure}

The D'Agostino and Pearson normality test confirms the visual interpretation. The p-values ($0.164$, $0.982$, and $0.354$) of all three samples are higher than the significance level (0.05), and the null hypothesis is accepted. Sample 5, 6 and 7 follows the normal distribution after the transformation, and parametric methods can be used with these samples.
 
 \subsubsection[Sample 8, 9 and 10]{All participants divided by task,  correct element variable}\label{sec:task123_correct_normaltest}
This section will examine sample 8, 9 and 10 (table \ref{tab:totalcorrect_tasks}) for the normal distribution assumption. These samples will be used to test whether there is a significant difference between the three tasks when looking at the number of correctly chosen elements variable. 

A visual analysis of the three histograms in figure \ref{fig:correctallnotinterupted_task1}, \ref{fig:correctallnotinterupted_task2} and \ref{fig:correctallnotinterupted_task3} show a negative skewness, just like the histograms in section \ref{sec:correct_ex_inex}. This gives an indication that the three samples are not drawn from a normal distribution.
 
 \begin{figure}[H]
 	\centering
	 \begin{subfigure}[b]{0.3\textwidth}
	 	\centering
	 	\includegraphics[width=\linewidth]{../../thesis-statisticmethods/statistic_analysis/figures/task1/correct_all_not_interupted}
	 	\caption{Sample 8 -Task A}
	 	\label{fig:correctallnotinterupted_task1}
	 \end{subfigure}
	\begin{subfigure}[b]{0.3\textwidth}
		\centering
		\includegraphics[width=\linewidth]{../../thesis-statisticmethods/statistic_analysis/figures/task2/correct_all_not_interupted}
		\caption{Sample 9 - Task B}
		\label{fig:correctallnotinterupted_task2}
	\end{subfigure}
	 \begin{subfigure}[b]{0.3\textwidth}
	 	\centering
	 	\includegraphics[width=\linewidth]{../../thesis-statisticmethods/statistic_analysis/figures/task3/correct_all_not_interupted}
	 	\caption{Sample 10 - Task C}
	 	\label{fig:correctallnotinterupted_task3}
	 \end{subfigure}
 \caption{Histogram with normal distribution fit showing samples with number of correct elements per task}
 \end{figure}
 D'Agostino and Pearson normality test confirms our visual analysis of the histograms in two of three samples. Sample 9 passes the normality test, even though the p-value ($0.099$) is close to the significance level (0.05). Sample 8 and sample 10 do not pass the normal assumption test. Both samples obtained a p-value ($0.00022$ and $0.0047$) smaller than the significance level. The null hypothesis is rejected for sample 8 and 10, and the alternative hypothesis is accepted. The null hypothesis is accepted for sample 9. 

A Box-Cox power transformation is applied to all three samples. A transformation changes the data, and to correctly compare the results sample 9 has to be transformed, even though it follows a normal distribution. The transformed data is shown in histogram \ref{fig:correctboxcox_task1}, \ref{fig:correctboxcox_task2}, and \ref{fig:correctboxcox_task3}. All three are negatively skewed, sample 9, and 10 less than sample 8. The conclusion is not obvious in these histograms.

\begin{figure}[H]
	\centering
	\begin{subfigure}[b]{0.3\textwidth}
		\centering
		\includegraphics[width=\linewidth]{../../thesis-statisticmethods/statistic_analysis/figures/task1/correct_boxcox}
		\caption{Sample 8 - Task A}
		\label{fig:correctboxcox_task1}
	\end{subfigure}
	\begin{subfigure}[b]{0.3\textwidth}
		\centering
		\includegraphics[width=\linewidth]{../../thesis-statisticmethods/statistic_analysis/figures/task2/correct_boxcox}
		\caption{Sample 9 - Task B}
		\label{fig:correctboxcox_task2}
	\end{subfigure}
	\begin{subfigure}[b]{0.3\textwidth}
		\centering
		\includegraphics[width=\linewidth]{../../thesis-statisticmethods/statistic_analysis/figures/task3/correct_boxcox}
		\caption{Sample 10 - Task C}
		\label{fig:correctboxcox_task3}
	\end{subfigure}
	\caption{Histogram with normal distribution fit after Box-Cox}
\end{figure}

D'Agostino and Pearson normality test accepts the null hypothesis on sample 9 and 10 and rejects it on sample 8. Sample 9 and 10 has p-values ($0.0752$ and $0.2104$) higher than the significance level, while computed p-value for sample 8 ($0.0027$) is significantly lower. When using these three samples in hypothesis tests, a non-parametric method should be used. This is because sample 8 do not follow the normal distribution.

\subsubsection[Sample 11, 12 and 13]{Experienced participants divided by task, total time variable}
In this section, sample 11, 12, and 13, shown in table \ref{tab:totaltime_tasks_experienced}, is normal distribution tested. The three samples will be used to test whether there is a significant difference between the total time results for the three tasks when considering only experienced participants. 

A visual interpretation of the histograms in figure \ref{fig:sample11_12_13_histogram} show that all three samples are positively skewed (figure \ref{fig:skew}). Skew gives a fairly strong evidence that the samples are not normally distributed.

 \begin{figure}[H]
 	\centering
	 \begin{subfigure}[b]{0.32\textwidth}
	 	\centering
	 	\includegraphics[width=\linewidth]{../../thesis-statisticmethods/statistic_analysis/figures/task1/totaltime_experienced}
	 	\caption{Sample 11 - Task A}
	 	\label{fig:totaltimeexperienced_task1}
	 \end{subfigure}
	 \begin{subfigure}[b]{0.32\textwidth}
	 	\centering
	 	\includegraphics[width=\linewidth]{../../thesis-statisticmethods/statistic_analysis/figures/task2/totaltime_experienced}
	 	\caption{Sample 12 - Task B}
	 	\label{fig:totaltimeexperienced_task2}
	 \end{subfigure}
	 \begin{subfigure}[b]{0.32\textwidth}
	 	\centering
	 	\includegraphics[width=\linewidth]{../../thesis-statisticmethods/statistic_analysis/figures/task3/totaltime_experienced}
	 	\caption{Sample 13 - Task C}
	 	\label{fig:totaltimeexperienced}
	 \end{subfigure}
 	\caption{Histograms with normal distribution fit with samples containing total time to complete each task}
 	\label{fig:sample11_12_13_histogram}
 \end{figure}
 
D'Agostino and Pearson normality test confirms our visual interpretation of the three histograms. All three p-values ($1.229 * 10^{-14}$, $2.678 * 10^{-5}$ and $0.000884$) are lower than the significance level (5\%). Sample 11, 12 and 13 do not pass the normality assumption with a confidence interval of 95\%. 

A Box-Cox power transformation is applied to all three samples since the null hypothesis was rejected. Histograms with normal distribution fit containing the transformed data is shown in figure \ref{fig:sample11_12_13_boxcox_histogram}. Visually, the histograms look like they follow a normal distribution with minimal skewness. 

\begin{figure}[H]
	\centering
	\begin{subfigure}[b]{0.32\textwidth}
		\centering
		\includegraphics[width=\linewidth]{../../thesis-statisticmethods/statistic_analysis/figures/task1/totaltime_experienced_boxcox}
		\caption{Sample 11 - Task A}
		\label{fig:totaltimeexperiencedboxcox_task1}
	\end{subfigure}
	\begin{subfigure}[b]{0.32\textwidth}
		\centering
		\includegraphics[width=\linewidth]{../../thesis-statisticmethods/statistic_analysis/figures/task2/totaltime_experienced_boxcox}
		\caption{Sample 12 - Task B}
		\label{fig:totaltimeexperiencedboxcox_task2}
	\end{subfigure}
	\begin{subfigure}[b]{0.32\textwidth}
		\centering
		\includegraphics[width=\linewidth]{../../thesis-statisticmethods/statistic_analysis/figures/task3/totaltime_experienced_boxcox}
		\caption{Sample 13 - Task C}
		\label{fig:totaltimeexperiencedboxcox_task3}
	\end{subfigure}
	\caption{Histograms with normal distribution fit containing Box-Cox transformed data}
	\label{fig:sample11_12_13_boxcox_histogram}
\end{figure}

The Box-Cox transformed data is tested with D'Agostino and Pearson normality test. All three samples obtained p-values ($0.694$, $0.955$ and $0.887$) larger than the significance level (0.05). Within a confidence interval of 95\%, the test concludes that sample 11, 12 and 13 is normally distributed. These samples can be used in parametric methods.

\subsubsection[Sample 14, 15 and 16]{Experienced participants divided by task, correct elements variable}
This section will test if sample 14, 15, and 16, shown in table \ref{tab:totalcorrect_tasks_experienced}, follows the normal distribution. The three samples will be used to test whether there is a significant difference in the number of correct elements between the three tasks when comparing only experienced participants. 

A visual interpretation of the histograms in \ref{fig:sample14,15,16_normhistogram} show that all three samples are slightly negatively skewed. Sample 15 (\ref{fig:correctexperienced_task1}) and sample 16 (\ref{fig:correctexperienced_task2}) has less skewness than sample 14 (\ref{fig:correctexperienced_task3}).   

\begin{figure}[H]
	\centering
	\begin{subfigure}[b]{0.32\textwidth}
		\centering
		\includegraphics[width=\linewidth]{../../thesis-statisticmethods/statistic_analysis/figures/task1/correct_experienced}
		\caption{Sample 14 - Task A}
		\label{fig:correctexperienced_task1}
	\end{subfigure}
	\begin{subfigure}[b]{0.32\textwidth}
		\centering
		\includegraphics[width=\linewidth]{../../thesis-statisticmethods/statistic_analysis/figures/task2/correct_experienced}
		\caption{Sample 15 - Task B}
		\label{fig:correctexperienced_task2}
	\end{subfigure}
	\begin{subfigure}[b]{0.32\textwidth}
		\centering
		\includegraphics[width=\linewidth]{../../thesis-statisticmethods/statistic_analysis/figures/task3/correct_experienced}
		\caption{Sample 16 - Task C}
		\label{fig:correctexperienced_task3}
	\end{subfigure}
	\caption{Histogram with normal distribution fit showing samples with number of correct elements results for experienced participants}
	\label{fig:sample14,15,16_normhistogram}
\end{figure}

 D'Agostino and Pearson normality test confirms our visual interpretation of the three histograms. All three p-values ($0.0588$, $0.2067$ and $0.2975$) are higher than the significance level (0.05). Notice that sample 14 has a lower p-value than the two other samples. This sample is not as significant as the two other samples. Sample 14, 15 and 16 pass the normality test with a confidence interval of 95\%. The null hypothesis is accepted. These samples can be used in parametric methods. 

\subsubsection[Sample 17, 18 and 19]{Inexperienced participants divided by task, total time variable}\label{sec:sample_17,18,19_normalitytest}
This section will test if sample 17, 18 and 19, table \ref{tab:totaltime_tasks_inexperienced}, follows the normal distribution. These samples will be used to test whether there is a significant difference in total time between the three tasks when only looking at inexperienced participants.

A visual analysis of the histograms \ref{fig:totaltimeinexperienced_task1}, \ref{fig:totaltimeinexperienced_task2} and \ref{fig:totaltimeinexperienced_task3} show a positive skewness. The skew is less than the histograms in figure \ref{fig:sample11_12_13_histogram}, but is most likely too large for the samples to be normally distributed.

\begin{figure}[H]
	\centering
	\begin{subfigure}[b]{0.32\textwidth}
		\centering
		\includegraphics[width=\linewidth]{../../thesis-statisticmethods/statistic_analysis/figures/task1/totaltime_inexperienced}
		\caption{Sample 17 - Task A}
		\label{fig:totaltimeinexperienced_task1}
	\end{subfigure}
	\begin{subfigure}[b]{0.32\textwidth}
		\centering
		\includegraphics[width=\linewidth]{../../thesis-statisticmethods/statistic_analysis/figures/task2/totaltime_inexperienced}
		\caption{Sample 18 - Task B}
		\label{fig:totaltimeinexperienced_task2}
	\end{subfigure}
	\begin{subfigure}[b]{0.32\textwidth}
		\centering
		\includegraphics[width=\linewidth]{../../thesis-statisticmethods/statistic_analysis/figures/task3/totaltime_inexperienced}
		\caption{Sample 19 - Task C}
		\label{fig:totaltimeinexperienced_task3}
	\end{subfigure}
	\caption{Histogram with normal distribution fit}
\end{figure}

The D'Agostino and Pearson normality test agrees with the visual analysis. The three obtained p-values ($1.586 * 10^{-11}$, $1.773 * 10^{-6}$ and $2.312 * 10 ^{-11}$) are all significantly lower than the significance level of 0.05. Sample 17, 18, and 19 do not follow a normal distribution with a confidence interval of 95\%. The null hypothesis is rejected. 

The samples are Box-Cox power transformed. The histograms after transformation (\ref{fig:totaltimeinexperiencedboxcox_task1}, \ref{fig:totaltimeinexperiencedboxcox_task2}, and \ref{fig:totaltimeinexperiencedboxcox_task3}) are visually evaluated to be normally distributed. 

\begin{figure}[H]
	\centering
	\begin{subfigure}[b]{0.32\textwidth}
		\centering
		\includegraphics[width=\linewidth]{../../thesis-statisticmethods/statistic_analysis/figures/task1/totaltime_inexperienced_boxcox}
		\caption{Sample 17 - Task A}
		\label{fig:totaltimeinexperiencedboxcox_task1}
	\end{subfigure}
	\begin{subfigure}[b]{0.32\textwidth}
		\centering
		\includegraphics[width=\linewidth]{../../thesis-statisticmethods/statistic_analysis/figures/task2/totaltime_inexperienced_boxcox}
		\caption{Sample 18 - Task B}
		\label{fig:totaltimeinexperiencedboxcox_task2}
	\end{subfigure}
	\begin{subfigure}[b]{0.32\textwidth}
		\centering
		\includegraphics[width=\linewidth]{../../thesis-statisticmethods/statistic_analysis/figures/task3/totaltime_inexperienced_boxcox}
		\caption{Sample 19 - Task C}
		\label{fig:totaltimeinexperiencedboxcox_task3}
	\end{subfigure}
	\caption{Histogram with normal distribution fit after Box-Cox}
\end{figure}

A new D'Agostino and Pearson normality test on the transformed data confirms that all three samples are drawn from a normal distribution with a significance level of 5\%. The obtained p-values ($0.139$, $0.909$ and $0.067$) are larger than 0.05, and the null hypothesis is accepted. These samples can be used in parametric methods \\[0.2cm]

\subsubsection[Sample 20, 21 and 22]{Inexperienced participants  divided by task, correct elements variable}\label{sec:sample_20,21,22_normalitytest}
This section will test if sample 20, 21, and 22, in table \ref{tab:totalcorrect_tasks_inexperienced}, follows the normal distribution. These samples contain the number of correctly chosen elements in each of the three tasks from only inexperienced participants. The samples will be used to test if inexperienced participants do better in one of the tasks. 

A visual interpretation of histogram \ref{fig:correctinexperienced_task1}, \ref{fig:correctinexperienced_task2} and \ref{fig:correctinexperienced_task3} show a negative skew, similar to the histograms containing results from only experienced participants (\ref{fig:sample14,15,16_normhistogram}).

\begin{figure}[H]
	\centering
	\begin{subfigure}[b]{0.32\textwidth}
		\centering
		\includegraphics[width=\linewidth]{../../thesis-statisticmethods/statistic_analysis/figures/task1/correct_inexperienced}
		\caption{Sample 20 - Task A}
		\label{fig:correctinexperienced_task1}
	\end{subfigure}
	\begin{subfigure}[b]{0.32\textwidth}
		\centering
		\includegraphics[width=\linewidth]{../../thesis-statisticmethods/statistic_analysis/figures/task2/correct_inexperienced}
		\caption{Sample 21 - Task B}
		\label{fig:correctinexperienced_task2}
	\end{subfigure}
	\begin{subfigure}[b]{0.32\textwidth}
		\centering
		\includegraphics[width=\linewidth]{../../thesis-statisticmethods/statistic_analysis/figures/task3/correct_inexperienced}
		\caption{Sample 22 - Task C}
		\label{fig:correctinexperienced_task3}
	\end{subfigure}
	\caption{Histogram with normal distribution fit}
	\label{fig:sample20,21,22_normtest_original}
\end{figure}

The D'Agostino and Pearson normality test rejects the null hypothesis on sample 20 and 22 and accepts the null hypothesis on sample 21. Sample 20 and 22 obtained p-values ($0.007$ and $0.004$) lower than 0.05 and sample 21 obtained a p-value ($0.523$) higher than 0.05. The test concludes that sample 21 follows the normal distribution, and sample 20 and 22 does not with a significant level of 5\%. 

A Box-Cox power transformation is applied to all three samples. The transformation changes the data, and to correctly compare the data, sample 21 also has to be transformed, even though the original data was followed the normal distribution. The transformed samples are shown in histogram \ref{fig:correctinexperiencedboxcox_task1}, \ref{fig:correctinexperiencedboxcox_task2} and \ref{fig:correctinexperiencedboxcox_task3}. All three histograms are less skewed than the original histograms (\ref{fig:sample20,21,22_normtest_original}).

\begin{figure}[H]
	\centering
	\begin{subfigure}[b]{0.32\textwidth}
		\centering
		\includegraphics[width=\linewidth]{../../thesis-statisticmethods/statistic_analysis/figures/task1/correct_inexperienced_boxcox}
		\caption{Sample 20 - Task A}
		\label{fig:correctinexperiencedboxcox_task1}
	\end{subfigure}
	\begin{subfigure}[b]{0.32\textwidth}
		\centering
		\includegraphics[width=\linewidth]{../../thesis-statisticmethods/statistic_analysis/figures/task2/correct_inexperienced_boxcox}
		\caption{Sample 21 - Task B}
		\label{fig:correctinexperiencedboxcox_task2}
	\end{subfigure}
	\begin{subfigure}[b]{0.32\textwidth}
		\centering
		\includegraphics[width=\linewidth]{../../thesis-statisticmethods/statistic_analysis/figures/task3/correct_inexperienced_boxcox}
		\caption{Sample 22 - Task C}
		\label{fig:correctinexperiencedboxcox_task3}
	\end{subfigure}
	\caption{Histogram with normal distribution fit after Box-Cox transformation}
\end{figure}

The D'Agostino and Pearson normality method is executed on the transformed samples. The three obtained p-values ($0.061$, $0.714$ and $0.311$) are higher than 0.05. The null hypothesis is accepted. Sample 20, 21 and 22 follows the normal distribution with a significance level of 5\% and can be used in parametric methods.  

\subsubsection{Normality test summary}\label{sec:normaltest_summary}

	\begin{longtable}{p{0.30\textwidth}|l|p{2cm}|p{0.25\textwidth}}
	\caption[Summary, normality tests]{Summary of normality tests done in section \ref{sec:normality_results}} \label{tab:normaltest_summary} \\
		  & Sample ID & Normally distributed  & Normally distributed after Box-Cox power transformation  \\ \hline
		\textit{Total time} & & & \\
		Experienced & 1 &No   & \textbf{Yes}   \\
		Inexperienced  & 2 & No & \textbf{Yes}     \\ \hline
		\textit{Correct elements} & & & \\
		Experienced & 3 & No  & No   \\
		Inexperienced  & 4 & No & No   \\ \hline
		\textit{Total time }& & & \\
		Task A & 5 &No  & \textbf{Yes}  \\
		Task B & 6 &No  & \textbf{Yes}   \\
		Task C & 7 & No & \textbf{Yes}  \\ \hline
		\textit{Correct elements} & & & \\
		Task A & 8 & No  & No  \\
		Task B & 9 &\textbf{Yes}  & \textbf{Yes}   \\
		Task C & 10 & No & \textbf{Yes}  \\ \hline
		\textit{Total time, experienced participants} & & & \\
		Task A & 11 & No  & \textbf{Yes}  \\
		Task B & 12 & No  & \textbf{Yes}   \\
		Task C & 13 & No & \textbf{Yes}  \\ \hline
		\textit{Correct elements, experienced participants} & & & \\
		Task A & 14 & \textbf{Yes}  & \textit{not tested} \\
		Task B & 15 & \textbf{Yes}  &  \textit{not tested} \\
		Task C & 16 & \textbf{Yes} & \textit{not tested} \\ \hline
		\textit{Total time, inexperienced participants} & & & \\
		Task A & 17& No  & \textbf{Yes}  \\
		Task B & 18 & No  & \textbf{Yes}   \\
		Task C & 19 & No & \textbf{Yes}  \\ \hline
		\textit{Correct elements, inexperienced participants} & & & \\
		Task A & 20 & No  & \textbf{Yes} \\
		Task B & 21 & \textbf{Yes}  & \textbf{Yes} \\
		Task C & 22 & No & \textbf{Yes} \\ \hline
	\end{longtable}

\vspace{0.7cm}

\subsection{Levene's test of Equality of Variance}\label{sec:levene_test_results}
As mentioned in section \ref{sec:t-test}, \ref{sec:anova}, and \ref{sec:mannwhiteyu}, the two sample t-test, one-way \textit{ANOVA} and Mann-Whitey U test assumes that the samples come from populations with equal variances. This assumption will be examined with Levene's test. The hypothesis tested is: \\[0.3cm]

\centerline{$H_{0}$: Input samples are from populations with equal variances} 
\centerline{$H_{A}$: Input samples are from populations that do not have equal variances}
\vspace{0.3cm}

The null hypothesis is accepted if the obtained p-value is higher than the significance level. Table \ref{tab:levenestest_summary} contains the summary of the Levene's test performed on all sample pairs. All sample pairs accepted the null hypothesis except sample 1 and 2, who obtained a p-value lower than the significance level. The test used a significance level of 5\% on all the tests. \\[0.2cm]

\begin{longtable}{p{0.15\textwidth}|p{0.4\textwidth}|p{0.1\textwidth}|p{0.25\textwidth}}
	\caption[Summary, Levene's tests]{Summary of Levene's tests} \label{tab:levenestest_summary} \\
	 Participants & & Obtained p-value & Samples come from populations with equal variances \\ \hline
	 All & & & \\
	 & \textit{Total time} & & \\
	& Sample 1 and 2 & 0.030 & \textbf{No}  \\ 
	& \textit{Correct elements} & & \\
	& Sample 3 and 4 &0.823& Yes   \\ 
	& \textit{Total time, divided by task}& & \\
	& Sample 5, 6, and 7 & 0.636 & Yes \\ 
	& \textit{Correct elements, divided by task} & & \\
	& Sample 8, 9, and 10 & 0.805  & Yes  \\ \hline
	Experienced & & & \\
	& \textit{Total time, divided by task} & & \\
	& Sample 11, 12, and 13 & 0.972 & Yes  \\ 
	& \textit{Correct elements, divided by task} & & \\
	& Sample 14, 15, and 16 & 0.724 & Yes \\ \hline
	Inexperienced & & & \\
	& \textit{Total time, divided by task} & & \\
	& Sample 17, 18, and 19 & 0.499  & Yes  \\ 
	& \textit{Correct elements, divided by task} & & \\
	& Sample 20, 21, and 22 & 0.626 & Yes \\ \hline
\end{longtable}

\vspace{0.5cm}

\subsection{Hypothesis testing}\label{sec:hypothesis_results}
This section will test the hypothesis listed in figure \ref{fig:hypothesis_ttest} and \ref{fig:hypothesis_anova} to answer the three research questions written in the introduction. The order of the tests will be the same as the numbering of the samples. Which statistic method, from the theory section (\ref{sec:hypothesistesting}), that is used to answer the hypothesis tests is determined by the results of the normality test (\ref{sec:normality_results}) and equal variance test (\ref{sec:levene_test_results}).

\subsubsection[Sample 1, 2]{Differences in total time between experienced and inexperienced participants}\label{sec:t-test_result} 
%Two sample t-test results
%\textbf{Test differences between experienced and inexperienced participants}\\
This section will test if there is any difference in total time spent on the tasks between experienced and inexperienced participants. The test is covered by sample 1 and sample 2 in table \ref{tab:totaltime_all}. Sample 1 is experienced and sample 2 inexperienced participants. Both samples was normally distributed after a Box-Cox transformation (\ref{sec:sample1,2_normresult}). A two-sample t-test will be used to answer this hypothesis since the normality assumption is valid. The hypothesis tested in this section is number one in figure \ref{fig:hypothesis_ttest}: \\[0.3cm]

\centerline{$H_{0}$: Equal task time between experienced and inexperienced participants} 
\centerline{$H_{A}$: Unequal task time between experienced and inexperienced participants}
%* Which of the hypothesis sentences to use? These or the one in the tables?
\vspace{0.3cm}

If $\overline{x}_1$ equals the mean time for experienced, and $\overline{x}_2$ the mean time for inexperienced participants, the hypothesis can be written as:\\[0.3cm]

\centerline{$H_{0}$: $\overline{x}_1$ = $\overline{x}_2$} 
\centerline{$H_{A}$: $\overline{x}_1$ $\neq$ $\overline{x}_2$}

\vspace{0.3cm}

Since we cannot assume equal variances in the two samples (Table \ref{tab:levenestest_summary}), this test will use the Welch's t-test for unequal variances [\citep{Walpole2012}, p. 345]. Equation \ref{eq:ttest_twoway} is still valid. The obtained values from the test are shown in the box below. The obtained T-statistic is smaller than the critical value. The t-test therefore conclude that there is a significant difference between the means of the two population samples with a confidence interval of 95\%.\\[0.2cm]

 \begin{center}
	\begin{tcolorbox}[width=0.8\textwidth]
		\centering
		\textit{Two sample, two-way t-test}\\
		Sample 1 and 2\\[0.5cm]
		
		Degree of freedom ($v$): 447 \\ %http://web.utk.edu/~cwiek/TwoSampleDoF
		Significance level ($\alpha$): 0.05 \\
		Critical value: 1.960\\[0.2cm]
		$T-statistic$: -60.442 \\
		
		Using equation \ref{eq:ttest_twoway}, the absolute value of the $T-statistic$ is larger than the critical value ($|60.442|$ > $1.960$) and the null hypothesis is \underline{rejected} and $H_A$ accepted.\\[0.5cm]

	\end{tcolorbox} 
\end{center}

\vspace{0.5cm}

\textbf{Test if experienced or inexperienced participants finish the task fastest} 

Because there was a statistical significant difference between time spent on each task between the participants, this section will test which group finished the task fastest. The second hypothesis tested in this section is number two in figure \ref{fig:hypothesis_ttest}:\newline

%\centerline{$H_{0}$: Inexperienced participants has a lower or equal total time compared to experienced participants.}
\centerline{$H_{0}$: Experienced do not finish the tasks faster}
\centerline{$H_{A}$: Experience participants finish the tasks faster}

With sample 1 being experienced participants and sample 2 inexperienced participants we get the hypothesis:\\[0.2cm]

\centerline{$H_{0}$: $\overline{x}_1$ = $\overline{x}_2$}
\centerline{$H_{A}$: $\overline{x}_1$ < $\overline{x}_2$}

This test gives the same T-statistics as the previous test, but the critical value is changed since this test used in the second hypothesis is a two sample, one-way t-test. The Welch's test is used since we cannot assume equal variances in the two samples. Obtained $T-statistic$ is still smaller than the critical value (-64.654 < 1.645). Our test is to check if the mean value of sample 1 is significantly larger than the mean value of sample 2. We use the comparison test written in equation \ref{eq:ttest_greater}, section \ref{sec:t-test}. Our $T-statistic$ is not larger than the critical value, and we need to accept the null hypothesis. There is no evidence that experienced participants use less time on the tasks than the inexperienced. \\[0.2cm]

 \begin{center}
	\begin{tcolorbox}[width=0.8\textwidth]
		\centering
		\textit{Two sample, one-way t-test}\\
		Sample 1 and 2\\
		Significance level: 5\%  \\[0.5cm]
		
		$T-statistic$: -60.442 \\
		Degree of freedom ($v$): 447 \\ %http://web.utk.edu/~cwiek/TwoSampleDoF
		Significance level ($\alpha$): 0.05 \\
		Critical value: 1.645\\[0.2cm]
		
		T-statistic is smaller than the critical value ($-60.442$ < $1.645$) and the null hypothesis is \underline{accepted}.\\[0.5cm]
	\end{tcolorbox} 
\end{center}

\begin{figure}[H]
	\centering
	\includegraphics[width=0.7\linewidth]{../../thesis-statisticmethods/statistic_analysis/figures/boxplot/mean_std_participants_time}
	\caption{Sample 1 and 2 - mean (green dot) and standard deviation (blue line)}
	\label{fig:meanstdparticipantstime}
\end{figure}

Since we know that there is a statistically significant difference between the two sample means, we conclude that the inexperienced participants finished the task faster than the experienced participants. The time difference can also be seen in plot \ref{fig:meanstdparticipantstime}. Inexperienced participants finished the tasks in average 16 seconds faster than experienced participants.\newline

\subsubsection[Sample 3, 4]{Difference between experienced and inexperienced participants in total correct elements} %Mann-Whitey U Test results
%\textbf{Test differences between experienced and inexperienced participants}\\

This section will test if there is a difference between experienced- and inexperienced participants when looking at the number of correctly chosen elements. Sample 3 and 4, table \ref{tab:totalcorrect_all}, is the correct samples to use in this test. Neither samples followed the normal distribution (\ref{sec:correct_ex_inex}), and we need to use a non-parametric method. Both samples have ties (identical observations), and, as mentioned in section \ref{sec:mannwhiteyu}, the Mann-Whitey U test is then preferred. From histogram \ref{fig:correctelementswasnotinterupter_ex} and \ref{fig:correctelementswasnotinterupted_inex} we see that the samples are identical in some cases. Mann-Whitey U test should, therefore, be used to compare the population medians. The hypothesis to be tested is:\\[0.3cm]

\centerline{$H_{0}$: $median_3$ = $median_4$}
\centerline{$H_{A}$: $median_3$ $\neq$ $median_4$}

Using equation \ref{eq:mannwhitey-ciritcalvalue} in section \ref{sec:mannwhiteyu} and the obtained $U-statistic$, we conclude that there is not enough evidence to reject the null hypothesis with a confidence interval of 95\%. The $U-statistic$ is larger than the critical value, and the null hypothesis is accepted. 

 \begin{center}
	\begin{tcolorbox}[width=0.8\textwidth]
		\centering
		\textit{Two sample t-test}\\
		Sample 3 and 4\\
		Significance level: 5\%  \\[0.5cm]
		
		$U-statistic$: 17012 \\
		Significance level ($\alpha$): 0.05 \\
		Sample size, n1:  229\\
		Sample size, n2: 200\\
		Critical-value: 127 \\[0.2cm] %http://web.utk.edu/~cwiek/TwoSampleDoF
		
		$U-statistic$ is larger than the critical value ($17012$ > $127$) and the null hypothesis is \underline{accepted}.\\[0.5cm]
	\end{tcolorbox} 
\end{center}

\begin{figure}[H]
	\centering
	\includegraphics[width=0.7\linewidth]{../../thesis-statisticmethods/statistic_analysis/figures/boxplot/mean_std_participants_correct}
	\caption{Sample 3 and 4 - mean (green dot) and standard deviation (blue line)}
	\label{fig:meanstdparticipantscorrect}
\end{figure}

Results from this section show that there is not enough evidence to conclude that there is any difference between experienced and inexperienced participants when looking at the number of correctly chosen elements per task. We conclude that experienced and inexperienced participants did equally well on the task. This can also be seen visually in figure \ref{fig:meanstdparticipantscorrect}. Mean values shown in the figure is approximately equal between the participants. 

\subsubsection[Sample 5, 6 and 7]{Test if total time differs between the three tasks}\label{sec:anova_result} %One-way \textit{ANOVA} test results
This section will test if time spent varies between each of the three tasks which are hypothesis number one in figure \ref{fig:hypothesis_anova}. Sample 5, 6, and 7, table \ref{tab:totaltime_tasks}, is used in this test. The one-way \textit{ANOVA} method will be applied to answer the hypothesis. All three samples come from populations with equal variances (\ref{tab:levenestest_summary}), the samples are also normally distributed after a Box-Cox transformation (\ref{sec:task123_time_normaltest}).\\

\centerline{$H_{0}$: $\overline{x}_5$ = $\overline{x}_6$ = $\overline{x}_7$}
\centerline{$H_{A}$: Total time differs between at least two of the tasks}

Using equation \ref{eq:anova_reject} in section \ref{sec:anova} and results obtained from the calculations, the one-way \textit{ANOVA} test rejects the null hypothesis. The obtained $f-value$ is lower than the critical value. With a confidence interval of 95\% we claim that there is a difference between the mean value of the three tasks.
\vspace{0.2cm}
 \begin{center}
	\begin{tcolorbox}[width=0.8\textwidth]
		\centering
		\textit{One-way \textit{ANOVA}}\\
		Sample 5, 6 and 7\\[0.5cm]
		
		%$P-value$: $2.805 * 10^{-222}$ \\
		Significance level ($\alpha$): 0.05 \\
		$v_1$ = 2, $v_2$ = 426 \\
		Critical-value: 3.00 \\%http://web.utk.edu/~cwiek/TwoSampleDoF
		$f-value$: $2123.308$ \\[0.2cm] 
		
		$f-value$ is significantly higher than the critical value ($2123.308$ > $3.00$) and the null hypothesis is \underline{rejected}, $H_A$ is accepted\\[0.5cm]
	\end{tcolorbox} 
\end{center}
\vspace{0.2cm}
When rejecting the null hypothesis, \textit{Tukey's method} is used to make comparisons between task A, B, and C. This test did not find any statistically significant difference between the three tasks. Visual evaluation of figure \ref{fig:meanstdtask123pngtime} show that task A was completed slightly faster than the two other tasks. 
\vspace{0.2cm}
 \begin{center}
	\begin{tcolorbox}[width=0.8\textwidth]
		\centering
		\textit{Tukey's test}\\
		Sample 5, 6 and 7\\
		Significance level: 5\%  \\[0.5cm] 
		%Task 1 - Sample 5, Task 2 - Sample 6 and Task 3 - Sample 7 \\[0.5cm]
		
		Task A and Task B do not differ significantly \\
		Task A and Task C do not differ significantly \\
		Task B and Task C do not differ significantly \\[0.2cm]
	\end{tcolorbox} 
\end{center}

\begin{figure}[H]
	\centering
	\includegraphics[width=0.7\linewidth]{../../thesis-statisticmethods/statistic_analysis/figures/boxplot/mean_std_task123png_time}
	\caption{Sample 5, 6 and 7 - mean (green dot) and standard deviation (blue line)}
	\label{fig:meanstdtask123pngtime}
\end{figure}

\vspace{0.3cm}
Based on the results from this section, we conclude that there is a statistically significant difference in total time between at least two of the tasks, but the difference is not enough for \textit{Tukey's test} to notice a difference. 

\subsubsection[Sample 8, 9, and 10]{Test if the number of correct elements differs between the three tasks}\label{sec:sample_8_9_10_kruskal)} %Kruskal-Wallis test results
This section will test if there is a difference in the number of correctly chosen elements between the three tasks which are hypothesis two in figure \ref{fig:hypothesis_anova}. The test will use sample 8, 9 and 10 from table \ref{tab:totalcorrect_tasks}. Since sample 8 is not normally distributed (\ref{sec:task123_correct_normaltest}) a non-parametric test should be applied. The Kruskal-Wallis test is the non-parametric equivalent to one-way \textit{ANOVA} (\ref{sec:kruskal-w-test}). The method tests equality of medians when the samples do not follow the normal distribution. The hypothesis tested is: \\[0.2cm]

\centerline{$H_{0}$: $median_8$ = $median_9$ = $median_{10}$}
\centerline{$H_{A}$: Number of correctly chosen elements differ between at least two of the tasks}
\vspace{0.2cm}

Using equation \ref{eq:kruskapw-accept} in section \ref{sec:kruskal-w-test}, the Kruskal-Wallis test rejects the null hypothesis. The obtained $H-value$ is smaller than the critical value. The p-value is approximately zero, and this gives a good indication that the result is significant. With a confidence interval of 95\%, we claim that there is a difference between the median value of the three tasks. \\[0.2cm]

 \begin{center}
	\begin{tcolorbox}[width=0.8\textwidth]
		\centering
		\textit{Kruskal-Wallis test}\\
		Sample 8, 9 and 10\\[0.5cm]
		
		Significance level ($\alpha$): 0.05 \\
		$v$ = 2\\ %k-1 = 3-1 = 2
		Critical-value:  5.991\\[0.2cm] %http://web.utk.edu/~cwiek/TwoSampleDoF
		$P-value$: $3.967* 10^{-72}$ \\
		$H-value$: $328.816$ \\
		
		$H-value$ is significantly higher than the critical value ($328.816$ > $5.991$) and the null hypothesis is \underline{rejected}, $H_A$ is accepted\\[0.5cm]
	\end{tcolorbox} 
\end{center}
\vspace{0.3cm}

Like in the one-way \textit{ANOVA}, a \textit{post hoc} test should be used to make paired comparisons to determine which groups differ. The \textit{post hoc} test applied is Tukey's test. Results from Tukey's test resulted in a significant difference in the number of correctly chosen elements between task A and task B, and task A and task C. This can also visually be seen in figure \ref{fig:meanstdtask123pngtime}. The participants had in average 0.8 more correct elements in task A compared to the two other tasks. Task A also has a smaller standard deviation than the other tasks. \\[0.2cm]

 \begin{center}
	\begin{tcolorbox}[width=0.8\textwidth]
		\centering
		\textit{Tukey's test}\\
		Sample 8, 9 and 10 \\
		Significance level: 5\%  \\[0.5cm]
		%Task 1 - Sample 8, Task 2 - Sample 9 and Task 3 - Sample 10 \\[0.5cm]
		
		Task A and Task B differs significantly \\
		Task A and Task C differs significantly \\
		Task B and Task C does not differ significantly \\[0.2cm]
	\end{tcolorbox} 
\end{center}

\begin{figure}[H]
	\centering
	\includegraphics[width=0.7\linewidth]{../../thesis-statisticmethods/statistic_analysis/figures/boxplot/mean_std_task123png_correct}
	\caption{Sample 8, 9 and 10 - mean (green dot) and standard deviation (blue line)}
	\label{fig:meanstdtask123pngcorrect}
\end{figure}

\vspace{0.3cm}

The results in this section find statistically significant evidence that the participants had a higher number of correct elements in task A than in the two other tasks.

\subsubsection[Sample 11 - 16]{Test differences in results from experienced participants}\label{sec:sample_11_12_13_anova}
This section will answer two hypothesis about experienced participant's results divided by task. The first hypothesis will test if time spent on each of the three tasks differ and the second hypothesis will test if the number of correct elements in each of the three tasks differs when the samples only include results from experienced participants. Sample 11, 12 and 13, from table \ref{tab:totaltime_tasks_experienced}, will be used to answer the first hypothesis. All three samples are normally distributed after a Box-Cox power transformation (\ref{tab:normaltest_summary}) and also come from populations with equal variances (\ref{tab:levenestest_summary}). Sample 14, 15 and 16, from table \ref{tab:totalcorrect_tasks_experienced}, will be used on the second hypothesis. These three samples are also normally distributed (\ref{tab:normaltest_summary}) and come from populations with equal variances (\ref{tab:levenestest_summary}). The one-way \textit{ANOVA} method will be used to test both hypotheses. 

The first hypothesis is:\\
\centerline{$H_{0}$: $\overline{x}_{11}$ = $\overline{x}_{12}$ = $\overline{x}_{13}$}
\centerline{$H_{A}$: Total time differs between at least two of the tasks}
\vspace{0.2cm}

Using equation \ref{eq:anova_reject}  from section \ref{sec:anova}, the one-way \textit{ANOVA} test rejects the null hypothesis. The obtained $f-value$ ($1216.919$) is higher than the critical value ($3.00$). The calculated p-value is approximately zero, which gives a good indication that the result is significant. With a confidence interval of 95\% the author claim that there is a time difference between the three tasks. 

When the null hypothesis is rejected, a \textit{post hoc} test is used to compare each task with each other. Tukey's \textit{post hoc} test did not find any significant difference between the three tasks with a significance level of 5\%. Figure \ref{fig:meanstdexperiencedtask123time} show an approximately similar mean value in all three tasks. Task B has a lower mean and less standard deviation, but it is not statistically significantly different.

\begin{figure}[H]
	\centering
	\includegraphics[width=0.7\linewidth]{../../thesis-statisticmethods/statistic_analysis/figures/boxplot/mean_std_experienced_task123_time}
	\caption{Sample 11, 12 and 13 - mean (green dot) and standard deviation (blue line)}
	\label{fig:meanstdexperiencedtask123time}
\end{figure}

%The author concludes that there is a significant difference between at least two of the tasks when looking at experienced participants mean task time, but the difference is not enough so that \textit{Tukey's test} can find which tasks that differ. \newline

The second hypothesis is:\\
\centerline{$H_{0}$: $\overline{x}_11$ = $\overline{x}_12$ = $\overline{x}_13$}
\centerline{$H_{A}$: Number of correct elements in each task differs between at least two of the tasks}
\vspace{0.2cm}

Using equation \ref{eq:anova_reject}, the one-way \textit{ANOVA} test rejects the null hypothesis. The obtained $f-value$ ($8.210$) is higher than the critical value ($3.00$). The p-value is approximately zero, which gives a good indication that the result is significant. With a confidence interval of 95\%, we claim that there is a difference between the mean value of at least two of the tasks. 

Since the null hypothesis was rejected, a \textit{post hoc} test should be used to make paired comparisons to determine which groups differ. Tukey's \textit{post-hoc} test resulted in a significant difference in the number of correctly chosen elements between task A and task B, and task A and task C. Figure \ref{fig:meanstdexperiencedtask123correct} show that task A has a higher mean value than the two other tasks. Task A also has a smaller standard deviation. 

\begin{figure}[H]
	\centering
	\includegraphics[width=0.7\linewidth]{../../thesis-statisticmethods/statistic_analysis/figures/boxplot/mean_std_experienced_task123_correct}
	\caption{Sample 14, 15 and 16 - mean (green dot) and standard deviation (blue line)}
	\label{fig:meanstdexperiencedtask123correct}
\end{figure}

\vspace{0.3cm}

With the results found in this section, we conclude that there is a statistically significant difference in the number of correctly chosen elements between the three tasks for experienced participants. They got the best result on task A. Time spent on each task also differs between at least two of the tasks, but the difference is not significant enough so that Tukey's test can determine a difference. Figure \ref{fig:meanstdexperiencedtask123time} show that task B has the lowest mean time value of the three tasks.  

\subsubsection[Sample 17 -  22]{Test differences in results from inexperienced participants}\label{sec:sample_17_18_19_anova}
This section will answer the same hypothesis as the previous section, but using results from inexperienced participants. The first hypothesis will test if time spent on each task differ and the second hypothesis will test if the number of correct elements in each task differs. Sample 17, 18 and 19, from table \ref{tab:totaltime_tasks_inexperienced}, will be applied in the first hypothesis test. These samples are normally distributed (\ref{sec:sample_17,18,19_normalitytest}) and come from populations with equal variances (\ref{tab:levenestest_summary}). Sample 20, 21 and 22, from table \ref{tab:totalcorrect_tasks_inexperienced}, will be used on the second hypothesis. All three samples are normally distributed (\ref{sec:sample_20,21,22_normalitytest}) and also come from populations with equal variances (\ref{tab:levenestest_summary}). The one-way \textit{ANOVA} will be used to test both hypotheses. 

The first hypothesis is:\\
\centerline{$H_{0}$: $\overline{x}_{17}$ = $\overline{x}_{18}$ = $\overline{x}_{19}$}
\centerline{$H_{A}$: Total time differ between at least two of the tasks}
\vspace{0.2cm}

The one-way \textit{ANOVA} test rejects the null hypothesis and accepts the alternative hypothesis ($H_{A}$) with a significance level of 5\%. The obtained f-value from the test is higher than the critical value ($905.34$ > $3.00$). Since the alternative hypothesis was accepted, Tukey's \textit{post hoc} test is used to make compared comparisons between task A, task B and task C. The test does not find a significant difference when comparing each of the three tasks with a significance level of 5\%. Figure \ref{fig:meanstdinexperiencedtask123time} show that inexperienced participants spent more time on task B than the other tasks, but the difference is not significant according to Tukey's test.

\begin{figure}[H]
	\centering
	\includegraphics[width=0.6\linewidth]{../../thesis-statisticmethods/statistic_analysis/figures/boxplot/mean_std_inexperienced_task123_time}
	\caption{Mean (green dot) and standard deviation (blue line) for sample 17, 18 and 19}
	\label{fig:meanstdinexperiencedtask123time}
\end{figure}

The second hypothesis is:\\
\centerline{$H_{0}$: $\overline{x}_{20}$ = $\overline{x}_{21}$ = $\overline{x}_{22}$}
\centerline{$H_{A}$: Number of correct elements differ between at least two of the tasks}
\vspace{0.2cm}

The one-way \textit{ANOVA} test rejects the null hypothesis ($H_0$) and accepts the alternative hypothesis ($H_{A}$) with a significance level of 5\%. The obtained f-value from the test is higher than the critical value ($189.05$ > $3.00$). Since the null hypothesis was rejected, Tukey's \textit{post hoc} test will be used to make comparisons between the three tasks. This test cannot find a significant difference when comparing the tasks with a significance level of 5\%. Looking at figure \ref{fig:meanstdinexperiencedtask123correct}, task A has a higher mean value than the two other tasks, and task B also has a higher mean than task C. Even though there are differences in the number of correct elements between the tasks, it is not significant according to Tukey's test. 

\begin{figure}[H]
	\centering
	\includegraphics[width=0.6\linewidth]{../../thesis-statisticmethods/statistic_analysis/figures/boxplot/mean_std_inexperienced_task123_correct}
	\caption{Mean (green dot) and standard deviation (blue line) for sample 20, 21 and 22}
	\label{fig:meanstdinexperiencedtask123correct}
\end{figure}

With the results found in this section, we conclude that there is a statistically significant difference in both total time spent on each task and the number of correctly chosen elements between at least two of the tasks. The differences are not significant enough so that Tukey's test can determine which task differs. Figure \ref{fig:meanstdinexperiencedtask123time} and \ref{fig:meanstdinexperiencedtask123correct} show that task A has the lowest mean time value and the highest mean correct value. 

\subsubsection{Hypothesis test summary}

	\begin{longtable}{p{0.65\textwidth}|p{0.15\textwidth}|p{0.15\textwidth}}  %\multicolumn{1}{c}{}
	\caption[Summary, hypothesis tests]{Summary of hypothesis tests done in section \ref{sec:hypothesis_results}} \label{tab:hypothesistest_summary} \\
		Hypothesis (\textcolor{cyan}{Dependent variable}, \textcolor{blue}{Independent variable}) & \textit{Participants} Sample ID& Hypothesis is accepted \\[0.2cm] \hline
		\textcolor{cyan}{Total time}, \textcolor{blue}{Experienced} and \textcolor{blue}{Inexperienced} & \textit{All} &  \\
		There is a difference between experienced and inexperienced participants & 1 and 2 & \textbf{Yes} \\
		Experienced participants finish the tasks faster than inexperienced  & 1 and 2 & No   \\ 
		Inexperienced participants finish the tasks faster than experienced  & 1 and 2 & \textbf{Yes}   \\ \hline
		\textcolor{cyan}{Correct elements}, \textcolor{blue}{Experienced} and \textcolor{blue}{Inexperienced} & \textit{All} &  \\
		There is a difference between experienced and inexperienced participants & 3 and 4 & No   \\ \hline
		\textcolor{cyan}{Total time}, \textcolor{blue}{Task A}, \textcolor{blue}{Task B} and \textcolor{blue}{Task C}& \textit{All} &  \\
		 Total time is different between at least two of the tasks & 5, 6 and 7 & \textbf{Yes}   \\
		 Task A significantly differs from Task B & 5 and 6 & No  \\ 
		 Task A significantly differs from Task C & 5 and 7 & No  \\ 
		 Task B significantly differs from Task C & 6 and 7 & No  \\ \hline
		\textcolor{cyan}{Correct elements}, \textcolor{blue}{Task A}, \textcolor{blue}{Task B}, \textcolor{blue}{Task C} & \textit{All} &  \\
		The number of correctly chosen elements is different between at least two of the tasks & 8, 9 and 10 & \textbf{Yes}  \\
		Task A significantly differs from Task B & 8, and 9 & \textbf{Yes}  \\ 
		Task A significantly differs from Task C & 8 and 10 & \textbf{Yes}  \\ 
		Task B significantly differs from Task C &  9 and 10 & No  \\ \hline
		\textcolor{cyan}{Total time}, \textcolor{blue}{Task A}, \textcolor{blue}{Task B}, \textcolor{blue}{Task C} & \textit{Experienced}  &  \\
		Total time is different between at least two of the tasks & 11, 12 and 13 & \textbf{Yes}  \\
		Task A significantly differs from Task B & 11 and 12  & No  \\ 
		Task A significantly differs from Task C & 11 and 13  & No  \\ 
		Task B significantly differs from Task C & 12 and 13 & No  \\ \hline
		\textcolor{cyan}{Correct elements}, \textcolor{blue}{Task A}, \textcolor{blue}{Task B}, \textcolor{blue}{Task C} & \textit{Experienced}  &  \\
		Number of correct elements differs between at least two of the tasks & 14, 15 and 16 & \textbf{Yes}  \\
		Task A significantly differs from Task B & 14 and 15 & \textbf{Yes}  \\ 
		Task A significantly differs from Task C & 14 and 16 & \textbf{Yes} \\ 
		Task B significantly differs from Task C & 15 and 16 & No  \\ \hline
		\textcolor{cyan}{Total time}, \textcolor{blue}{Task A}, \textcolor{blue}{Task B}, \textcolor{blue}{Task C} & \textit{Inexperienced}  &  \\
		Total time is different between at least two of the tasks & 17, 18 and 19 & \textbf{Yes}  \\
		Task A significantly differs from Task B & 17 and 18 & No  \\ 
		Task A significantly differs from Task C & 17 and 19 & No  \\ 
		Task B significantly differs from Task C & 18 and 19 & No  \\ \hline
		\textcolor{cyan}{Correct elements}, \textcolor{blue}{Task A}, \textcolor{blue}{Task B}, \textcolor{blue}{Task C} & \textit{Inexperienced}  &  \\
		Number of correct elements differs between at least two of the tasks & 20, 21 and 22 & \textbf{Yes}  \\
		Task A significantly differs from Task B & 20 and 21 & No  \\ 
		Task A significantly differs from Task C & 20 and 22 & No \\ 
		Task B significantly differs from Task C & 21 and 22 & No  \\ \hline
	\end{longtable}