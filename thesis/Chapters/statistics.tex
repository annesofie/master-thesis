\section{Survey results}\label{sec:survey_results}

- All participants ordered by age, excluded by task 4 \\
- All results in one task, ordered by age \\ 
- Average time per micro-task \\
- Is there a difference in task number 1, 2, 3? time and correct  \\
Can use it to explain the data

\subsection{Gathered data}\label{sec:gathereddata}

The gathered data will be analyse on the two variables: 1) total time used to complete each task and 2) number of correctly chosen elements per task. Total time and number of correct elements sums the time and correctly chosen elements on question one and question two together. Sample mean $\overline{x}$, standard deviation of $\overline{x}$, standard error ($\frac{standard deviation}{\sqrt{sample size}}$) of $\overline{x}$, minimum in sample and maximum in sample are listed in the tables. In these tables, results from the training task is removed. Only results from the three tasks is used. Results from participants that was disturbed during the survey was also removed, four of the participants spent more than twice the estimated time on the survey. Maximum possible correct elements per task is twelve. There are six elements in question one and six elements in question two, and the number of correctly chosen elements in each task is added together, maximum twelve correctly chosen elements.

% The tables in this section, (\ref{sec:gathereddata}), are task results from all participants and all three tasks, excluding the training task. Task results with total time longer than 2160 seconds are filtered out. This is to remove 4 outliers that spend more than twice the approximated time (average time on the survey was 1080 seconds in the pilot test). These 4 participants also answered that they were disturbed during the test.

 The gathered data are in the first subsection (\ref{sec:alltasks}) divided into experienced and inexperienced. In subsection \ref{sec:taskdivided_all} the data is divided into the three tasks, containing all participants. Section \ref{sec:taskdivided_experienced}  and \ref{sec:taskdivided_inexperienced} divides the data into the three tasks and also in experienced and inexperienced participants. 
 
%Removed all participants that said they was distracted. 26 task results was removed, 10 inexperienced and 18 experienced results.

\subsubsection{All,  experienced and inexperienced participants}\label{sec:alltasks}

The mean, standard deviation, minimum and maximum values are listen i table \ref{tab:totaltime_all} and \ref{tab:totalcorrect_all}. I. 

\begin{table}[H]
	\centering
	\begin{tabular}{l|l|l|l}
		Total time per task (seconds)   & All  & Experienced & Inexperienced \\ \hline
		Sample number &   & 1  & 2   \\
		Number of observations & 429    & 229    & 200   \\
		Sample mean $\overline{x}$     & 170.32 & 177.65  & 161.94     \\
		Sample median  & 155.0 & 158.0  & 154.0  \\
		Standard deviation of $\overline{x}$  & 82.19  & 88.24  & 73.99   \\
		Standard error of $\overline{x}$  & 3.98  & 5.83 & 5.23  \\
		Minimum in sample & 38.00  & 52.00  & 38.00     \\
		Maximum in sample & 657.00 & 657.00  & 529.00    \\ \hline
	\end{tabular}
	\caption[All participants and all tasks]{Total time used per task}
	\label{tab:totaltime_all}
\end{table}

\begin{figure}[H]
	\centering
	\begin{subfigure}[b]{0.32\textwidth}
		\centering
		\includegraphics[width=\linewidth]{../../thesis-statisticmethods/statistic_analysis/figures/all_participants/histogram/totaltime}
		\caption{All}
		\label{fig:totaltime_all}
	\end{subfigure}
	\begin{subfigure}[b]{0.32\textwidth}
		\centering
		\includegraphics[width=\linewidth]{../../thesis-statisticmethods/statistic_analysis/figures/experienced_participants/histogram/totaltime}
		\caption{Experienced}
		\label{fig:totaltime_experienced}
	\end{subfigure}
	\begin{subfigure}[b]{0.32\textwidth}
		\centering
		\includegraphics[width=\linewidth]{../../thesis-statisticmethods/statistic_analysis/figures/inexperienced_participants/histogram/totaltime}
		\caption{Inexperienced}
		\label{fig:totaltime_inexperienced}
	\end{subfigure}
	\caption[Total time, participants sorted]{Total time per task divided in  all-, experienced- and inexperienced-participants}
\end{figure}

The histogram show that finishing a task within 100 seconds has a higher probability when the individual is experienced. Finishing the task in about 200 seconds is equally likely for both experienced and inexperienced individuals. 

\begin{table}[H]
	\centering
	\begin{tabular}{l|l|l|l}
		Correct elements per task  & All  & Experienced & Inexperienced \\ \hline
		Sample number &   & 3  & 4   \\
		Number of observations & 429    & 229  & 200   \\
		Sample mean $\overline{x}$   & 9.82 & 9.81  & 9.83  \\
		Sample median & 10.0 & 10.0 & 10.0 \\
		Standard deviation of $\overline{x}$   & 1.52  & 1.53  &  1.51 \\
		Standard error of $\overline{x}$   & 0.07  & 0.10 &  0.11 \\
		Minimum in sample & 4.00 & 5.00  &  4.00  \\
		Maximum in sample  & 12.00 & 12.00  & 12.00  \\ \hline
	\end{tabular}
	\caption[Correct elements, all participants and all tasks]{Number of correctly chosen elements per task}
	\label{tab:totalcorrect_all}
\end{table}

\begin{figure}[H]
	\centering
	\begin{subfigure}[b]{0.32\textwidth}
		\centering
		\includegraphics[width=\linewidth]{../../thesis-statisticmethods/statistic_analysis/figures/all_participants/histogram/total_correct_elements}
		\caption{All}
		\label{fig:totalcorrectelements_all}
	\end{subfigure}
	\begin{subfigure}[b]{0.32\textwidth}
		\centering
		\includegraphics[width=\linewidth]{../../thesis-statisticmethods/statistic_analysis/figures/experienced_participants/histogram/total_correct_elements}
		\caption{Experienced}
		\label{fig:totalcorrectelements_experienced}
	\end{subfigure}
	\begin{subfigure}[b]{0.32\textwidth}
		\centering
		\includegraphics[width=\linewidth]{../../thesis-statisticmethods/statistic_analysis/figures/inexperienced_participants/histogram/total_correct_elements}
		\caption{Inexperienced}
		\label{fig:totalcorrectelements_inexperienced}
	\end{subfigure}
	\caption[Correct elements, participants sorted]{Correctly chosen elements per task divided in  all-, experienced- and inexperienced-participants}
\end{figure}

The histogram shows that for experienced individuals there are a higher probability of getting 12 correct than for inexperienced individuals. 

\subsubsection{All participants, divided in task 1, task 2 and task 3}\label{sec:taskdivided_all}

In table \ref{tab:totaltime_tasks} and \ref{tab:totalcorrect_tasks} mean, standard deviation, minimum and maximum is listed for the three different task the participants did in the survey. Task 1 is the task that served the participants with one and one elements. Task 2 is the task that served the participants with three and three elements, and task 3 gave all six elements at the same time. 

\begin{table}[H]
	\centering
	\begin{tabular}{l|l|l|l}
		Total time per task (seconds) & Task 1 & Task 2 & Task 3 \\ \hline
		Sample number & 5  & 6  & 7    \\
		Number of observations & 146    & 142      & 141     \\
		Sample mean $\overline{x}$  & 166.38  &  172.25   &   172.48  \\
		Sample median & 150.0  &  155.5  & 157.0  \\
		Standard deviation of $\overline{x}$   & 84.57  & 84.21  & 77.95   \\
		Standard error of $\overline{x}$   & 7.00 & 7.07 & 6.56 \\
		Minimum in sample    & 47  & 50 &   38   \\
		Maximum in sample   & 657 & 492  & 529 \\ \hline
	\end{tabular}
	\caption[Total time, divided into task 1, 2 and 3]{Total time divided into task 1, task 2 and task 3}
	\label{tab:totaltime_tasks}
\end{table}

\begin{table}[H]
	\centering
	\begin{tabular}{l|l|l|l}
		Correct elements per task & Task 1 & Task 2 & Task 3\\ \hline
		Sample number & 8  & 9  & 10   \\
		Number of observations & 146    & 142     & 141        \\
		Sample mean $\overline{x}$ & 10.19  &  9.71  &   9.55   \\
		Sample median & 11.0 &  10.0  &  10.0   \\
		Standard deviation of $\overline{x}$ & 1.43  & 1.53 & 1.52    \\
		Standard error of $\overline{x}$ & 0.12 &  0.13 & 0.13  \\
		Minimum in sample  & 5.00  & 5.00  &   4.00  \\
		Maximum in sample  & 12.00 & 12.00  & 12.00 \\ \hline
	\end{tabular}
	\caption[Correct elements, divided into task 1, task 2 and task 3]{Number of correctly chosen elements divided into task 1, task 2 and task 3}
	\label{tab:totalcorrect_tasks}
\end{table}

\subsubsection{Experienced participants, divided in task 1, task 2 and task 3}\label{sec:taskdivided_experienced}

Dividing task 1, task 2 and task 3 results into experienced and inexperienced. Table \ref{tab:totaltime_tasks_experienced} are data gathered about experienced participants total time per task. Table \ref{tab:totalcorrect_tasks_experienced} are data gathered about experienced participants number of correctly chosen elements per task.

\begin{table}[H]
	\centering
	\begin{tabular}{l|l|l|l}
		Total time per task & Task 1 & Task 2 & Task 3 \\ \hline
		Sample number & 11  & 12  & 13   \\
		Number of observations & 81    & 84      & 82   \\
		Sample mean $\overline{x}$  & 187.74  &  174.65  &  191.93   \\
		Sample median  & 162.5  &  153.0  &  165.0  \\
		Standard deviation of $\overline{x}$ & 122.08  & 84.30  & 115.18    \\
		Standard error of $\overline{x}$ & 13.64  & 9.20 & 12.79   \\
		Minimum in sample   & 57.00  & 52.00 &  44.00  \\
		Maximum in sample  & 657.00 & 492.00  & 752.00 \\ \hline
	\end{tabular}
	\caption[Total time, task and experienved divided]{Experienced total time per task, divided by task}
	\label{tab:totaltime_tasks_experienced}
\end{table}

\begin{table}[H]
	\centering
	\begin{tabular}{l|l|l|l}
		Correct elements per task & Task 1 & Task 2 & Task 3 \\ \hline
		Sample number & 14 & 15  & 16   \\
		Number of observations & 81    & 84      & 82   \\
		Sample mean $\overline{x}$ & 10.23  &  9.69  &  9.51   \\
		Sample median & 11.0  &  10.0  &  10.0  \\
		Standard deviation of $\overline{x}$ & 1.30  & 1.62  & 1.46   \\
		Standard error of $\overline{x}$ & 0.14  & 0.18  & 0.16   \\
		Minimum in sample & 7.00 & 5.00 &  5.00 \\
		Maximum in sample  & 12.00 & 12.00  & 12.00 \\ \hline
	\end{tabular}
	\caption[Correct elements, task and experienved divided]{Experienced participant's number of correct elements per task, divided by task}
	\label{tab:totalcorrect_tasks_experienced}
\end{table}

\subsubsection{Inexperienced participants, divided in task 1, task 2 and task 3}\label{sec:taskdivided_inexperienced}

Table \ref{tab:totaltime_tasks_inexperienced} are mean time, standard deviation, minimum time and maximum time spent on each task for inexperienced participants. Number of correctly chosen elements per task for inexperienced participants is shown in table \ref{tab:totalcorrect_tasks_inexperienced}. 

\begin{table}[H]
	\centering
	\begin{tabular}{l|l|l|l}
		Total time per task (seconds) & Task 1 & Task 2 & Task 3 \\ \hline
		Number of observations & 71    & 69  & 70   \\
		Sample mean $\overline{x}$  & 159.28  &  174.42  &  169.50  \\
		Sample median & 150.0  &  155.5  &  157.0  \\
		Standard deviation of $\overline{x}$  & 68.24  & 106.69  & 83.17   \\
		Standard error of $\overline{x}$  & 8.10  & 12.84  & 9.94   \\
		Minimum in sample & 47.00 & 26.00 &  38.00 \\
		Maximum in sample & 487.00 & 755.00  & 529.00  \\ \hline
	\end{tabular}
	\caption[Total time, inexperienced per task]{Inexperienced participant's time spent per task, divided by task}
	\label{tab:totaltime_tasks_inexperienced}
\end{table}

\begin{table}[H]
	\centering
	\begin{tabular}{l|l|l|l}
		Correct elements per task & Task 1 & Task 2 & Task 3 \\ \hline
		Number of observations & 71    & 69  & 70  \\
		Sample mean $\overline{x}$  & 9.99  &  9.65  &  9.60  \\
		Sample median  & 10.0  & 10.0  &  10.0  \\
		Standard deviation of $\overline{x}$  & 1.50  & 1.45  & 1.56   \\
		Standard error of $\overline{x}$  & 0.18 & 0.17 & 0.19  \\
		Minimum in sample  & 5.00 & 6.00 &  4.00  \\
		Maximum in sample  & 12.00 & 12.00  & 12.00 \\ \hline
	\end{tabular}
	\caption[Correct elements, inexperienced per task]{Inexperienced participant's number of correct elements per task, divided by task}
	\label{tab:totalcorrect_tasks_inexperienced}
\end{table}

\subsection{Normality tests}\label{sec:normality_results}
To check if a two-sample t-test (subsection \ref{sec:t-test}) and \textit{ANOVA}-test (subsection \ref{sec:anova}) can be used, the samples need to be tested if they are normally distributed or not. Both tests assume normally distributed samples. The normality section \ref{sec:normaltesting} concluded that the D'Agostino and Person normality test should be used in this thesis. A visual interpretation of histograms will also be a part of the normality tests. The D'Agostino-Pearson test uses the following hypothesis:\newline

\centerline{$H_{0}$: The data follows the normal distribution} 
\centerline{$H_{A}$: The data do not follow the normal distribution}


\subsubsection[Sample 1 and 2]{Experienced and inexperienced participants - total time samples}\label{sec:totaltime_ex_inex}
The histograms \ref{fig:totaltimeexclude4_experienced} and \ref{fig:totaltimeexclude4_inexperienced} are positively skewed (see figure \ref{fig:skew}). This gives an indication that sample 1 and 2 are not normally distributed. Samples involving time measurements are rarely normally distributed. This is because the sample will always be skewed since it is impossible to have negative time. There will always be a limit to how fast a participant can finish the tasks. 

\begin{figure}[H]
	\centering
	\begin{subfigure}[b]{0.48\textwidth}
		\centering
		\includegraphics[width=\linewidth]{../../thesis-statisticmethods/statistic_analysis/figures/experienced_participants/normalplot/totaltime_exclude4}
		\caption{Sample 1 - Experienced}
		\label{fig:totaltimeexclude4_experienced}
	\end{subfigure}
	\begin{subfigure}[b]{0.48\textwidth}
		\centering
		\includegraphics[width=\linewidth]{../../thesis-statisticmethods/statistic_analysis/figures/inexperienced_participants/normalplot/totaltime_exclude4}
		\caption{Sample 2 - Inexperienced}
		\label{fig:totaltimeexclude4_inexperienced}
	\end{subfigure}
\caption{Histograms with normal distribution fit with samples containing total time to complete each task}
\end{figure}

An D'Agostino and Pearson normality test (\ref{sec:normaltesting}) confirmed the visual assessment conclusion with an significance level of 5\% (0.05). Both samples are not normally distributed with a confidence level of 95\%. \\[0.5cm]

\begin{center}
	\begin{tcolorbox}[box align=center,width=\textwidth-5cm]
			\centering
				\textit{D'Agostino and Pearson normality test}\\
				Significance level: 5\%  \\[0.5cm]
	
				Sample 1: Experienced, total time per task\\
				P-value: $3.874 * 10^{-22}$\\
				The p-value is lower than the significance level (0.05), the null hypothesis is \underline{rejected} and $H_1$ accepted.\\[0.5cm]
				
				Sample 2: Inexperienced, total time per task \\
				P-value: $2.574 * 10^{-21}$ \\
				The p-value is lower than the significance level (0.05), the null hypothesis is \underline{rejected} and $H_1$ accepted.\\[0.5cm]
	\end{tcolorbox} 
\end{center}
%Boc-Cox Transformation? https://docs.scipy.org/doc/scipy-0.19.0/reference/generated/scipy.stats.boxcox.html https://www.isixsigma.com/tools-templates/normality/dealing-non-normal-data-strategies-and-tools/

In both sample 1 and 2, the p-value was significantly lower than the significance level of 0.05. Data transformations are commonly used tools to improve normality of a sample distributions, but there are many types of data transformations. \cite{Osborne2010} claim that almost all analyses, even non-parametric tests, benefit from improving the normality of the samples, especially when the normality test is significantly denied. Common traditional transformations are square root, inverse or converting to logarithmic scales  \citep{Osborne2010}. 

A Box-Cox power transformation is used in this thesis. This transformation can be used on positive data and the data used in this thesis will never negative. Box-Cox takes the idea of having a range of power transformations (square root $x^{\frac{1}{2}}$, inverse $x^{-1}$ etc.) available to improve the effectiveness of normalizing and variance equalizing for both positively- and negatively-skewed variables \citep{Osborne2010}. This transformation will always use the appropriate transformation to be maximally effective in moving each sampled data towards normality. This is the reason why this thesis will use the Box-Cox transformation to hopefully achieve normally distributed samples.

The transformed data is shown in histogram \ref{fig:totaltimeboxcoxtransformation_experienced} and \ref{fig:totaltimeboxcoxtransformedtitle_inexperienced}. A visual inspection gives a good indication that the transformed data is normally distributed.

\begin{figure}[H]
	\centering
	\begin{subfigure}[b]{0.48\textwidth}
		\centering
		\includegraphics[width=\linewidth]{../../thesis-statisticmethods/statistic_analysis/figures/experienced_participants/normalplot/totaltime_boxcox_transformation}
		\caption[Experienced, Box-Cox]{Sample 1 - Experienced participants}
		\label{fig:totaltimeboxcoxtransformation_experienced}
	\end{subfigure}
	\begin{subfigure}[b]{0.48\textwidth}
		\centering
		\includegraphics[width=\linewidth]{../../thesis-statisticmethods/statistic_analysis/figures/inexperienced_participants/normalplot/totaltime_boxcox_transformed_title}
		\caption[Inexperienced, Box-Cox]{Sample 2 - Inexperienced participants}
		\label{fig:totaltimeboxcoxtransformedtitle_inexperienced}
	\end{subfigure}
\caption{Histograms with normal distribution fit after Box-Cox transformation}
\end{figure}

D'Agostino and Pearson normality test is then completed on the transformed data. This test confirms the visual inspection, both sample 1 and sample 2 are normally distributed after the Box-Cox transformation with an confidence level of 95\%. Calculated p-value is larger than the significance level (0.05). \\[0.5cm] 

\begin{center}
	\begin{tcolorbox}[box align=center,width=\textwidth-5cm]
		\centering
		\textit{D'Agostino and Pearson normality test}\\
		(After Box-Cox transformation) \\
		Significance level: 5\%  \\[0.5cm]
		
		Sample 1: Experienced, total time per task\\
		P-value: $0.849$\\
		The p-value is higher than the significance level (0.05), the null hypothesis is \underline{accepted}. \\[0.5cm]
		
		Sample 2: Inexperienced, total time per task \\ %*Her bruker jeg data hvor was interupted er fjernet
		P-value: $0.0623$ \\
		The p-value is higher than the significance level (0.05), the null hypothesis is \underline{accepted}. \\[0.5cm]
	\end{tcolorbox}
\end{center}

The assumption that sample 1 and sample 2 are normally distributed is now accepted and can be used in parametric methods as the two sample t-test and \textit{ANOVA} test.

\subsubsection[Sample 3 and 4]{Experienced and inexperienced participants - number of correctly chosen elements samples}\label{sec:correct_ex_inex}
%DIscrete variables http://stattrek.com/probability-distributions/discrete-continuous.aspx
Visual inspection of histogram \ref{fig:correctelementswasnotinterupter_ex} and \ref{fig:correctelementswasnotinterupted_inex} gives a good indication that sample 3 and 4 are not normally distributed. Both are negatively skewed (see figure \ref{fig:skew}).   

\begin{figure}[h!]
	\centering
	\begin{subfigure}[b]{0.48\textwidth}
		\centering
		\includegraphics[width=\linewidth]{../../thesis-statisticmethods/statistic_analysis/figures/experienced_participants/normalplot/correct_elements_was_not_interupter}
		\caption{Sample 3 - Experienced}
		\label{fig:correctelementswasnotinterupter_ex}
	\end{subfigure}
	\begin{subfigure}[b]{0.48\textwidth}
		\centering
		\includegraphics[width=\linewidth]{../../thesis-statisticmethods/statistic_analysis/figures/inexperienced_participants/normalplot/correct_elements_was_not_interupted}
		\caption{Sample 4 - Inexperienced}
		\label{fig:correctelementswasnotinterupted_inex}
	\end{subfigure}
	\caption{Histograms with normal distribution fit with samples containing the number of correctly chosen elements in each task}
\end{figure}

D'Agostino and Pearson normality test confirm ou visual interpretation. Both samples accept the alternative hypothesis with p-values lower than the significant level $0.05$. \\[0.5cm]

\begin{center}
	\begin{tcolorbox}[box align=center,width=\textwidth-5cm]
		\centering
		\textit{D'Agostino and Pearson normality test}\\
		Significance level: 5\%  \\[0.5cm]
		
		Sample 1: Experienced, correct elements per task\\
		P-value: $0.00443$\\
		The p-value is lower than the significance level (0.05), the null hypothesis is \underline{rejected} and $H_1$ accepted.\\[0.5cm]
		
		Sample 2: Inexperienced, correct elements per task \\
		P-value: $0.00013$ \\
		The p-value is lower than the significance level (0.05), the null hypothesis is \underline{rejected} and $H_1$ accepted.\\[0.5cm]
	\end{tcolorbox} 
\end{center}


Sample 3 and 4 was then Box-Cox transformed. After transformation a new D'Agostino and Pearson normality test was done. Both samples also failed this test. Sample 3 and 4 are not normally distributed and need to be tested with non-parametric methods. 

\subsubsection[Sample 5, 6 and 7]{All participants - Task 1, Task 2 and Task 3 - total time per task}\label{sec:task123_time}

In this section the data is separated in three samples, each sample containing one of the tasks. The participants had to do three different tasks in the survey. Task 1 is the one element task, task 2 is the three elements task and task 3 is the six elements task. The three samples are named sample 5, 6 and 7. These samples will be used to test weather there are an significant difference between the three tasks when looking at the total time variable. The total time variable tells us how much time each participants spent on each of the three tasks. In this section sample 5, 6 and 7 will be normality tested. 
 
 Visual analysis of the three histograms in figure \ref{fig:totaltimeallnotinterupted_task1}, \ref{fig:totaltimeallnotinterupted_task2} and \ref{fig:totaltimeallnotinterupted_task3} show a positive skewness as the time histograms in section \ref{sec:totaltime_ex_inex}. This gives an indication that the three samples are not normally distributed. 

\begin{figure}[h!]
	\centering
	\begin{subfigure}[b]{0.3\textwidth}
		\centering
		\includegraphics[width=\linewidth]{../../thesis-statisticmethods/statistic_analysis/figures/task1/totaltime_all_not_interupted}
		\caption{Sample 5 - Task 1}
		\label{fig:totaltimeallnotinterupted_task1}
	\end{subfigure}
	\begin{subfigure}[b]{0.3\textwidth}
		\centering
		\includegraphics[width=\linewidth]{../../thesis-statisticmethods/statistic_analysis/figures/task2/totaltime_all_not_interupted}
		\caption{Sample 6 - Task 2}
		\label{fig:totaltimeallnotinterupted_task2}
	\end{subfigure}
	\begin{subfigure}[b]{0.3\textwidth}
		\centering
		\includegraphics[width=\linewidth]{../../thesis-statisticmethods/statistic_analysis/figures/task3/totaltime_all_not_interupted}
		\caption{Sample 7 - Task 3}
		\label{fig:totaltimeallnotinterupted_task3}
	\end{subfigure}
	\caption{Histogram with normal distribution fit - sample with total time per task}
\end{figure}

The D'Agostino and Pearson normality test agreed with the visual analysis. P-values for all samples are smaller than the significance level 0.05, and the null hypothesis is rejected. The samples are not normally distributed with a confidence level of 95\%.\\[0.5cm]

\begin{center}
	\begin{tcolorbox}[box align=center,width=\textwidth-5cm]
		\centering
		\textit{D'Agostino and Pearson normality test}\\
		Significance level: 5\%  \\[0.5cm]
		
		Sample 5: All, total time on task 1 \\
		P-value: $2.39 * 10^{-24}$\\
		The p-value is lower than the significance level (0.05), the null hypothesis is \underline{rejected} and $H_1$ accepted.\\[0.5cm]
		
		Sample 6: All, total time on task 2 \\
		P-value: $2.57 * 10^{-9}$ \\
		The p-value is lower than the significance level (0.05), the null hypothesis is \underline{rejected} and $H_1$ accepted.\\[0.5cm]
		
		Sample 7: All, total time on task 3 \\
		P-value: $1.71 * 10^{-11}$ \\
		The p-value is lower than the significance level (0.05), the null hypothesis is \underline{rejected} and $H_1$ accepted.\\[0.5cm]
	\end{tcolorbox} 
\end{center}

The samples are then Box-Cox transformed. The histograms after the transformation is shown in figure \ref{fig:totaltimeallnotinteruptedboxcox_task1}, \ref{fig:totaltimeallnotinteruptedboxcox_task2} and \ref{fig:totaltimeallnotinteruptedboxcox_task3}. A visual analysis says that these histograms looks approximately normally distributed. The histograms looks like they have a skewness of approximately zero. 

\begin{figure}[h!]
	\centering
	\begin{subfigure}[b]{0.3\textwidth}
		\centering
		\includegraphics[width=\linewidth]{../../thesis-statisticmethods/statistic_analysis/figures/task1/totaltime_all_not_interupted_boxcox}
		\caption{Sample 5 - Task 1}
		\label{fig:totaltimeallnotinteruptedboxcox_task1}
	\end{subfigure}
	\begin{subfigure}[b]{0.3\textwidth}
		\centering
		\includegraphics[width=\linewidth]{../../thesis-statisticmethods/statistic_analysis/figures/task2/totaltime_all_not_interupted_boxcox}
		\caption{Sample 6 - task 2}
		\label{fig:totaltimeallnotinteruptedboxcox_task2}
	\end{subfigure}
	\begin{subfigure}[b]{0.3\textwidth}
		\centering
		\includegraphics[width=\linewidth]{../../thesis-statisticmethods/statistic_analysis/figures/task3/totaltime_all_not_interupted_boxcox}
		\caption{Sample 7 - task 3}
		\label{fig:totaltimeallnotinteruptedboxcox_task3}
	\end{subfigure}
	\caption{Histogram with normal distribution fit after Box-Cox transformation, sample with total time per task}
\end{figure}

The D'Agostino and Pearson normality test confirms the visual conclusion. The data is normally distributed after the Box-Cox transformation with a confidence interval of 95\%. The p-values of all three samples are higher than the significance level (0.05). \\[0.5cm]

\begin{center}
	\begin{tcolorbox}[box align=center,width=\textwidth-5cm]
		\centering
		\textit{D'Agostino and Pearson normality test}\\
		(After Box-Cox transformation)\\
		Significance level: 5\%  \\[0.5cm]
		
		Sample 5: All, total time on task 1 \\
		P-value: $0.164$\\
		The p-value is higher than the significance level (0.05), the null hypothesis is \underline{accepted}. \\[0.5cm]
		
		Sample 6: All, total time on task 2 \\
		P-value: $0.982$ \\
		The p-value is higher than the significance level (0.05), the null hypothesis is \underline{accepted}. \\[0.5cm]
		
		Sample 7: All, total time on task 3 \\
		P-value: $0.354$ \\
		The p-value is higher than the significance level (0.05), the null hypothesis is \underline{accepted}. \\[0.5cm]
	\end{tcolorbox} 
\end{center}

 Sample 5, 6 and 7 are normally distributed after the transformation and the assumptions of parametric tests are met. 
 
 \subsubsection[Sample 8, 9 and 10]{All participants - Task 1, Task 2 and Task 3 - correct elements per task}\label{sec:task123_correct}
 In this section the data is also separated in three samples, each sample containing one of the tasks. Task 1 is the one element task, task 2 is the three elements task and task 3 is the six elements task. The three samples are named sample 8, 9 and 10. These samples will be used to test weather there are an significant difference between the three tasks when looking at the number of correctly chosen elements variable. This variable tells us how many correct elements each participants chose on each of the three tasks. Total number of elements is twelve, six in each question. Before doing the hypothesis tests, sample 8, 9 and 10 will be normality tested. 
 
 A visual analysis of the three histograms in figure \ref{fig:correctallnotinterupted_task1}, \ref{fig:correctallnotinterupted_task2} and \ref{fig:correctallnotinterupted_task3} show a negative skewness, just like the histograms in section \ref{sec:correct_ex_inex}. This give an indication that the three samples are not normally distributed. 
 
 \begin{figure}[H]
 	\centering
	 \begin{subfigure}[b]{0.3\textwidth}
	 	\centering
	 	\includegraphics[width=\linewidth]{../../thesis-statisticmethods/statistic_analysis/figures/task1/correct_all_not_interupted}
	 	\caption{Sample 8 -Task 1}
	 	\label{fig:correctallnotinterupted_task1}
	 \end{subfigure}
	\begin{subfigure}[b]{0.3\textwidth}
		\centering
		\includegraphics[width=\linewidth]{../../thesis-statisticmethods/statistic_analysis/figures/task2/correct_all_not_interupted}
		\caption{Sample 9 - Task 2}
		\label{fig:correctallnotinterupted_task2}
	\end{subfigure}
	 \begin{subfigure}[b]{0.3\textwidth}
	 	\centering
	 	\includegraphics[width=\linewidth]{../../thesis-statisticmethods/statistic_analysis/figures/task3/correct_all_not_interupted}
	 	\caption{Sample 10 - Task 3}
	 	\label{fig:correctallnotinterupted_task3}
	 \end{subfigure}
 \caption{Histogram with normal distribution fit showing samples with number of correct elements per task}
 \end{figure}
 
 D'Agostino and Pearson normality test confirms our visual analysis of the histograms in two of three samples. Sample 9 actually passes the normality test, even though the p-value (0.099) is close to the significance level (0.05). Sample 8 and sample 10 do not pass the normality test with a confidence interval of 95\%. \\[0.2cm]
 
 \begin{center}
 	\begin{tcolorbox}[box align=center,width=\textwidth-5cm]
 		\centering
 		\textit{D'Agostino and Pearson normality test}\\
 		Significance level: 5\%  \\[0.5cm]
 		
 		Sample 8 \\
 		P-value: $0.00022$\\
 		The p-value is lower than the significance level (0.05), the null hypothesis is \underline{rejected} and $H_1$ accepted.\\[0.5cm]
 		
 		Sample 9 \\
 		P-value: $0.099$ \\
 		The p-value is higher than the significance level (0.05), the null hypothesis is \underline{accepted}. \\[0.5cm]
 		
 		Sample 10 \\
 		P-value: $0.0047$ \\
 		The p-value is lower than the significance level (0.05), the null hypothesis is \underline{rejected} and $H_1$ accepted.\\[0.5cm]
 	\end{tcolorbox} 
 \end{center}

\vspace{0.2cm}

To try to get all the samples approved in the normality test the Box-Cox transformation is applied to all three samples. The transformation changes the data, at to correctly compare the samples, sample 9 also has to be transformed. \\[0.2cm]


\begin{figure}[h!]
	\centering
	\begin{subfigure}[b]{0.3\textwidth}
		\centering
		\includegraphics[width=\linewidth]{../../thesis-statisticmethods/statistic_analysis/figures/task1/correct_boxcox}
		\caption{Sample 8 - Task 1}
		\label{fig:correctboxcox_task1}
	\end{subfigure}
	\begin{subfigure}[b]{0.3\textwidth}
		\centering
		\includegraphics[width=\linewidth]{../../thesis-statisticmethods/statistic_analysis/figures/task2/correct_boxcox}
		\caption{Sample 9 - Task 2}
		\label{fig:correctboxcox_task2}
	\end{subfigure}
	\begin{subfigure}[b]{0.3\textwidth}
		\centering
		\includegraphics[width=\linewidth]{../../thesis-statisticmethods/statistic_analysis/figures/task3/correct_boxcox}
		\caption{Sample 10 - Task 3}
		\label{fig:correctboxcox_task3}
	\end{subfigure}
	\caption{Histogram with normal distribution fit showing samples with number of correct elements per task - after Box-Cox transformation}
\end{figure}

 \begin{center}
	\begin{tcolorbox}[box align=center,width=\textwidth-5cm]
		\centering
		\textit{D'Agostino and Pearson normality test}\\
		(After Box-Cox transformation) \\
		Significance level: 5\%  \\[0.5cm]
		
		Sample 8: All, correct elements in task 1 \\
		P-value: $0.00269$\\
		The p-value is lower than the significance level (0.05), the null hypothesis is \underline{rejected} and $H_1$ accepted.\\[0.5cm]
		
		Sample 9: All, correct elements in task 2 \\
		P-value: $0.0752$ \\
		The p-value is higher than the significance level (0.05), the null hypothesis is \underline{accepted}. \\[0.5cm]
		
		Sample 10: All, correct elements in task 3 \\
		P-value: $0.2104$ \\
		The p-value is higher than the significance level (0.05), the null hypothesis is \underline{accepted}. \\[0.5cm]
	\end{tcolorbox} 
\end{center}

\vspace{0.4cm}

\subsubsection[Sample 11, 12 and 13]{Experienced participants - task 1, 2 and 3, total time variable}
 In this section the data is divided in experienced participants and also into task 1, 2 and 3. Task 1 is the one element task, task 2 is the three elements task and task 3 is the six elements task. The three samples are named sample 11, 12 and 13. These samples will be used to test whether there are an significant difference between the three tasks total time results when looking at only experienced participants. Before testing the hypothesis the samples need to be normality tested. 
 
  A visual interpretation of the histograms in \ref{fig:sample11_12_13_histogram} show that all three samples are positively skewed (figure \ref{fig:skew}). This is a fairly strong evidence that the samples are not normally distributed.
 
 \begin{figure}[H]
 	\centering
	 \begin{subfigure}[b]{0.32\textwidth}
	 	\centering
	 	\includegraphics[width=\linewidth]{../../thesis-statisticmethods/statistic_analysis/figures/task1/totaltime_experienced}
	 	\caption{Sample 11 - task 1}
	 	\label{fig:totaltimeexperienced_task1}
	 \end{subfigure}
	 \begin{subfigure}[b]{0.32\textwidth}
	 	\centering
	 	\includegraphics[width=\linewidth]{../../thesis-statisticmethods/statistic_analysis/figures/task2/totaltime_experienced}
	 	\caption{Sample 12 - task 2}
	 	\label{fig:totaltimeexperienced_task2}
	 \end{subfigure}
	 \begin{subfigure}[b]{0.32\textwidth}
	 	\centering
	 	\includegraphics[width=\linewidth]{../../thesis-statisticmethods/statistic_analysis/figures/task3/totaltime_experienced}
	 	\caption{Sample 13 - task 3}
	 	\label{fig:totaltimeexperienced}
	 \end{subfigure}
 	\caption{Histograms with normal distribution fit with samples containing total time to complete each task}
 	\label{fig:sample11_12_13_histogram}
 \end{figure}
 
  D'Agostino and Pearson normality test confirms our visual interpretation of the three histograms. All three p-values is lower than the significant level (0.05). Sample 11 ,12 and 13 do not pass the normality test with a confidence interval of 95\%. 
 
  \begin{center}
 	\begin{tcolorbox}[box align=center,width=\textwidth-5cm]
 		\centering
 		\textit{D'Agostino and Pearson normality test}\\
 		Significance level: 5\%  \\[0.5cm]
 		
 		Sample 11 \\
 		P-value: $6.258 * 10^{-14}$\\
 		The p-value is lower than the significance level (0.05), the null hypothesis is \underline{rejected} and $H_A$ accepted.\\[0.5cm]
 		
 		Sample 12 \\
 		P-value: $4.566 * 10^{-6}$ \\
		The p-value is lower than the significance level (0.05), the null hypothesis is \underline{rejected} and $H_A$ accepted.\\[0.5cm]
 		
 		Sample 13 \\
 		P-value: $1.179 * 10^{-13}$ \\
 		The p-value is lower than the significance level (0.05), the null hypothesis is \underline{rejected} and $H_A$ accepted.\\[0.5cm]
 	\end{tcolorbox} 
 \end{center}

A Box-Cox transformation is applied to all three samples. Histograms with normal distribution fit containing the transformed data is shown in figure \ref{fig:sample11_12_13_boxcox_histogram}. Visually, the histograms looks normally distributed with minimal skewness. \\

\begin{figure}[H]
	\centering
	\begin{subfigure}[b]{0.32\textwidth}
		\centering
		\includegraphics[width=\linewidth]{../../thesis-statisticmethods/statistic_analysis/figures/task1/totaltime_experienced_boxcox}
		\caption{Sample 11 - task 1}
		\label{fig:totaltimeexperiencedboxcox_task1}
	\end{subfigure}
	\begin{subfigure}[b]{0.32\textwidth}
		\centering
		\includegraphics[width=\linewidth]{../../thesis-statisticmethods/statistic_analysis/figures/task2/totaltime_experienced_boxcox}
		\caption{Sample 12 - task 2}
		\label{fig:totaltimeexperiencedboxcox_task2}
	\end{subfigure}
	\begin{subfigure}[b]{0.32\textwidth}
		\centering
		\includegraphics[width=\linewidth]{../../thesis-statisticmethods/statistic_analysis/figures/task3/totaltime_experienced_boxcox}
		\caption{Sample 13 - task 3}
		\label{fig:totaltimeexperiencedboxcox_task3}
	\end{subfigure}
	\caption{Histograms with normal fit containing Box-Cox transformed data}
	\label{fig:sample11_12_13_boxcox_histogram}
\end{figure}

The Box-Cox transformed data is tested with D'Agostino and Pearson normality test. All three samples has a p-value larger than the significance level (0.05). Within a confidence interval of 95\% the test concludes that sample 11, 12 and 13 is normally distributed.

 \begin{center}
	\begin{tcolorbox}[box align=center,width=\textwidth-5cm]
		\centering
		\textit{D'Agostino and Pearson normality test}\\
		(After Box-Cox transformation) \\
		Significance level: 5\%  \\[0.5cm]
		
		Sample 11 \\
		P-value: $0.674$\\
		The p-value is higher than the significance level (0.05), the null hypothesis is \underline{accepted}. \\[0.5cm]
		
		Sample 12 \\
		P-value: $0.943$ \\
		The p-value is higher than the significance level (0.05), the null hypothesis is \underline{accepted}. \\[0.5cm]
		
		Sample 13 \\
		P-value: $0.614$ \\
		The p-value is higher than the significance level (0.05), the null hypothesis is \underline{accepted}. \\[0.5cm]
	\end{tcolorbox} 
\end{center}

\subsubsection{Normality test summary}\label{sec:normaltest_summary}

	\begin{longtable}{p{0.30\textwidth}|l|p{2cm}|p{0.25\textwidth}}
	\caption[Summary, normality tests]{Summary of normality tests done in section \ref{sec:normality_results}} \label{tab:normaltest_summary} \\
		  & Sample & Normally distributed  & Normally distributed after Box-Cox  \\ \hline
		\textit{Total time} & & & \\
		Experienced & 1 &No   & \textbf{Yes}   \\
		Inexperienced  & 2 & No & \textbf{Yes}     \\ \hline
		\textit{Correct elements} & & & \\
		Experienced & 3 & No  & No   \\
		Inexperienced  & 4 & No & No   \\ \hline
		\textit{Total time }& & & \\
		Task 1 & 5 &No  & \textbf{Yes}  \\
		Task 2 & 6 &No  & \textbf{Yes}   \\
		Task 3 & 7 & No & \textbf{Yes}  \\ \hline
		\textit{Correct elements} & & & \\
		Task 1 & 8 & No  & No  \\
		Task 2 & 9 &\textbf{Yes}  & \textbf{Yes}   \\
		Task 3 & 10 & No & \textbf{Yes}  \\ \hline
		\textit{Total time, experienced participants} & & & \\
		Task 1 & 11 & No  & \textbf{Yes}  \\
		Task 2 & 12 & No  & \textbf{Yes}   \\
		Task 3 & 13 & No & \textbf{Yes}  \\ \hline
	\end{longtable}

\subsection{Levene's test}
As mentioned in section \ref{sec:normaltesting} and \ref{sec:anova}, the two sample t-test and \textit{ANOVA} assumes that the samples come from a population with equal variances. This assumption will be examined with levene's test. The hypothesis in this test is: \\[0.5cm]

\centerline{$H_{0}$: Input samples are from populations with equal variances} 
\centerline{$H_{A}$: Input samples are from populations that do not have equal variances}

\subsubsection[Sample 1 and 2]{Sample 1 and 2}\label{sec:sample1,2}
Since sample 1 and 2 was normal distributed accepted after the Box-Cox transformation, will the Levene's test use the transformed data in the analysis. Both samples contain total time spent on each task where sample 1 is for experienced and sample 2 is for inexperienced participants.

 \begin{center}
	\begin{tcolorbox}[box align=center,width=\textwidth-5cm]
		\centering
		\textit{Levene's test, sample 1 and 2}\\
		Significance level: 5\%  \\[0.5cm]
		
		P-value: $1.258 * 10^{-25}$\\[0.2cm]
		
		$P-value$ is smaller than the significance level ($1.258*10^{-25}$ < $0.05$) and the null hypothesis is \underline{rejected} and $H_A$ accepted.\\[0.5cm]
	\end{tcolorbox} 
\end{center}

We cannot assume that sample 1 and sample 2 comes from populations with equal variances. This has to be handled when testing hypothesis with these samples.

\subsubsection{Sample 5, 6 and 7}\label{sec:sample5,6,7}
Sample 5, 6 and 7 was normal distributed accepted after the Box-Cox transformation. Before using the samples in a \textit{ANOVA test} they need to be equal variance tested. All samples contain total time spent on each task, separated in task 1 (sample 5), task 2 (sample 6) and task 3 (sample 7). 

 \begin{center}
	\begin{tcolorbox}[box align=center,width=\textwidth-5cm]
		\centering
		\textit{Levene's test, sample 5, 6 and 7}\\
		Significance level: 5\%  \\[0.5cm]
		
		P-value: $0.636$\\[0.2cm]
		
		$P-value$ is larger than the significance level ($0.636$ > $0.05$) and the null hypothesis is \underline{accepted}.\\[0.5cm]
	\end{tcolorbox} 
\end{center}

\subsubsection{Sample 11, 12 and 13}\label{sec:sample5,6,7}
Sample 11, 12 and 13 was normal distributed accepted after the Box-Cox transformation. Before using the samples in a \textit{ANOVA test} they need to be equal variance tested. All samples contain experienced participants total time spent on each task, separated in task 1 (sample 11), task 2 (sample 12) and task 3 (sample 13). 

 \begin{center}
	\begin{tcolorbox}[box align=center,width=\textwidth-5cm]
		\centering
		\textit{Levene's test, sample 5, 6 and 7}\\
		Significance level: 5\%  \\[0.5cm]
		
		P-value: $0.636$\\[0.2cm]
		
		$P-value$ is larger than the significance level ($0.636$ > $0.05$) and the null hypothesis is \underline{accepted}.\\[0.5cm]
	\end{tcolorbox} 
\end{center}


\subsection{Hypothesis testing}\label{sec:hypothesis_results}

\subsubsection{Two sample t-test results}\label{sec:t-test_result}

\textbf{Test differences between experienced and inexperienced participants}\\
\textit{Sample 1 and 2}\newline

First hypothesis tested with the two sample t-test is: \\[0.3cm]

\centerline{$H_{0}$: Equal task time between participants} 
\centerline{$H_{A}$: Unequal task time between participants}

\vspace{0.3cm}

This test is covered by sample 1 and sample 2 from section \ref{sec:totaltime_ex_inex}. Sample mean and standard deviation for both samples is listed in table \ref{tab:totaltime_all}. Sample 1 is experienced participants and sample 2 inexperienced participants. $\overline{x}_1$ equals the mean total time for experienced-, and $\overline{x}_2$ the mean total time for inexperienced participants, the hypothesis can be written as:\\[0.3cm]

\centerline{$H_{0}$: $\overline{x}_1$ = $\overline{x}_2$} 
\centerline{$H_{A}$: $\overline{x}_1$ $\neq$ $\overline{x}_2$}

\vspace{0.3cm}

Since we cannot assume equal variances in the two samples, this test will use the Welch's t-test for unequal variances [\citep{Walpole2012}, p. 345]. Equation \ref{eq:ttest_twoway} is still valid. The T-statistic calculated is smaller than the critical value. We then conclude with that there is a significant difference between the means of the two population samples wit a confidence interval of 95\%.\\[0.2cm]

 \begin{center}
	\begin{tcolorbox}[box align=center,width=\textwidth-5cm]
		\centering
		\textit{Two sample t-test, sample 1 and 2}\\
		Significance level: 5\%  \\[0.5cm]
		
		$T-statistic$: -60.442 \\
		Degree of freedom ($v$): 447 \\ %http://web.utk.edu/~cwiek/TwoSampleDoF
		Significance level ($\alpha$): 0.05 \\
		Critical value: 1.960\\[0.2cm]
		
		Using equation \ref{eq:ttest_twoway}, the absolute value of the $T-statistic$ is larger than the critical value ($|60.442|$ > $1.960$) and the null hypothesis is \underline{rejected} and $H_A$ accepted.\\[0.5cm]

	\end{tcolorbox} 
\end{center}

\vspace{0.7cm}

\textbf{Test if  experienced or inexperienced participants finish the task fastest} 

The hypothesis tested is:\newline

%\centerline{$H_{0}$: Inexperienced participants has a lower or equal total time compared to experienced participants.}
\centerline{$H_{0}$: Equal total task time between all participants}
\centerline{$H_{A}$: Experience participants has a lower total time}

With sample 1 equals experienced participants and sample 2 equals inexperienced participants we get the hypothesis:\\[0.2cm]

\centerline{$H_{0}$: $\overline{x}_1$ = $\overline{x}_2$}
\centerline{$H_{A}$: $\overline{x}_1$ < $\overline{x}_2$}

This test gives the same T-statistics as the previous test, but the critical value is changed since it is an two sample, one way t-test. Since we cannot assume equal variances in the two samples a Welch's test is performed. The test results are shown under. The $T-statistic$ is still smaller than the critical value (-64.654 < 1.645). Our test is to check if sample 1 mean is significantly larger than sample 2 mean, then we use the test written in equation \ref{eq:ttest_greater}. Our $T-statistic$ is not larger than the critical value, therefore we need to accept the null hypothesis. There is no evidence that experienced participants use less time on the tasks than the inexperienced. 

 \begin{center}
	\begin{tcolorbox}[box align=center,width=\textwidth-5cm]
		\centering
		\textit{Two sample t-test, sample 1 and 2}\\
		Significance level: 5\%  \\[0.5cm]
		
		$T-statistic$: -60.442 \\
		Degree of freedom ($v$): 447 \\ %http://web.utk.edu/~cwiek/TwoSampleDoF
		Significance level ($\alpha$): 0.05 \\
		Critical value: 1.645\\[0.2cm]
		
		T-statistic is smaller than the critical value ($-60.442$ < $1.645$) and the null hypothesis is \underline{accepted}.\\[0.5cm]
	\end{tcolorbox} 
\end{center}

\begin{figure}[H]
	\centering
	\includegraphics[width=0.7\linewidth]{../../thesis-statisticmethods/statistic_analysis/figures/boxplot/mean_std_participants_time}
	\caption{Sample 1 and 2 - mean (green dot) and standard deviation (blue line)}
	\label{fig:meanstdparticipantstime}
\end{figure}

Since we know that there is a significant difference between the two sample means, the author concludes that the inexperienced participants finished the task faster than the experienced participants. This can also be seen in plot \ref{fig:meanstdparticipantstime}. Inexperienced participants finished the tasks 16 seconds faster than experienced participants. \newline

\subsubsection{Mann-Whitey U Test results}

\textbf{Test differences between experienced and inexperienced participants}\\
\textit{Sample 3 and 4}\newline

This section will test if there is a difference between experienced- and inexperienced participants when looking at the number of correctly chosen elements. Sample 3 and 4 is the correct samples to use in this test. Both samples are not normally distributed, we need to use a non-parametric method. The Mann-Whitey U test is the preferred test to use on these samples. As mentioned in section \ref{sec:Wilcoxon}, the Mann-Whitey U test is preferred when the samples are ties (identical observations) in the data. From histogram \ref{fig:correctelementswasnotinterupter_ex} and \ref{fig:correctelementswasnotinterupted_inex} we see that the samples for both independent variables are similar. Mann-Whitey U test can therefore be used to compare the population medians. The hypothesis to be tested is:\\[0.3cm]

\centerline{$H_{0}$: $median_3$ = $median_4$}
\centerline{$H_{A}$: $median_3$ $\neq$ $median_4$}

The results from the test, the statistical value and finding the critical value in the Mann-Whitey U table is shown in the box under. Using equation \ref{eq:mannwhitey-ciritcalvalue} we conclude that there are not enough evidence to reject the null hypothesis with an confidence interval of 95\%. The $U-statistic$ is larger than the critical value. 

 \begin{center}
	\begin{tcolorbox}[box align=center,width=\textwidth-5cm]
		\centering
		\textit{Two sample t-test, sample 3 and 4}\\
		Significance level: 5\%  \\[0.5cm]
		
		$U-statistic$: 17012 \\
		Significance level ($\alpha$): 0.05 \\
		Sample size, n1:  229\\
		Sample size, n2: 200\\
		Critical-value: 127 \\[0.2cm] %http://web.utk.edu/~cwiek/TwoSampleDoF
		
		$U-statistic$ is larger than the critical value ($17012$ > $127$) and the null hypothesis is \underline{accepted}.\\[0.5cm]
	\end{tcolorbox} 
\end{center}

\begin{figure}[H]
	\centering
	\includegraphics[width=0.7\linewidth]{../../thesis-statisticmethods/statistic_analysis/figures/boxplot/mean_std_participants_correct}
	\caption{Sample 3 and 4 - mean (green dot) and standard deviation (blue line)}
	\label{fig:meanstdparticipantscorrect}
\end{figure}

There is not enough evidence to conclude that there are any difference between experienced- and inexperienced participants when we use the dependent variable number of correctly chosen elements. 

\subsubsection{One-way \textit{ANOVA} test results}\label{sec:anova_result}

The one-way \textit{ANOVA} test will be used to test population means between the three task samples. This test is used when there is more than two samples being compared, see section \ref{sec:anova}. The assumption that sample 5, 6 and 7 come from populations with equal variances are met ()\ref{sec:sample5,6,7}). The hypothesis tested here is: \\

\centerline{$H_{0}$: $\overline{x}_5$ = $\overline{x}_6$ = $\overline{x}_7$}
\centerline{$H_{A}$: Total time is different between at least two of the tasks}

Using the one-way \textit{ANOVA} test to answer the hypothesis. Using equation \ref{eq:anova_reject}, the \textit{ANOVA} test rejects the null hypothesis. P-value is approximately zero and this gives a good indication that the result is significant. With a confidence interval of 95\% the author claim that there is a difference between the mean value of the three tasks.

 \begin{center}
	\begin{tcolorbox}[box align=center,width=\textwidth-5cm]
		\centering
		\textit{One-way \textit{ANOVA}, sample 5, 6 and 7}\\
		Significance level: 5\%  \\[0.5cm]
		
		$P-value$: $2.805 * 10^{-222}$ \\
		$f-value$: $2123.308$ \\
		Significance level ($\alpha$): 0.05 \\
		$v_1$ = 2, $v_2$ = 426
		Critical-value: 3.00 \\[0.2cm] %http://web.utk.edu/~cwiek/TwoSampleDoF
		
		$f-value$ is significantly lower than the critical value ($2123.308$ > $3.00$) and the null hypothesis is \underline{rejected}, $H_A$ is accepted\\[0.5cm]
	\end{tcolorbox} 
\end{center}

\begin{figure}[h!]
	\centering
	\includegraphics[width=0.7\linewidth]{../../thesis-statisticmethods/statistic_analysis/figures/boxplot/mean_std_task123png_time}
	\caption{Sample 5, 6 and 7 - mean (green dot) and standard deviation (blue line)}
	\label{fig:meanstdtask123pngtime}
\end{figure}

Using \textit{Tukey's test} to make compared comparisons between task one, two and three since the null hypothesis. This test did not find any significant difference between the three tasks. 

\subsubsection{Kruskal-Wallis test results}

The Kruskal-Wallis test is the non-parametric eqvivalent to one-way \textit{ANOVA} (\ref{sec:kruskal-w-test}). It is used to test equality of medians when the samples are not normally distributed. This method will test if there is any difference between task 1, 2 and 3 when looking at the dependent variable number of correctly chosen elements. The test will use sample 8, 9 and 10. Since sample 8 are not normally distributed a non-parametric test should be used. The hypothesis tested is: \\[0.2cm]

\centerline{$H_{0}$: $median_8$ = $median_9$ = $median_10$}
\centerline{$H_{A}$: Number of correctly chosen elements is different between at least two of the tasks}

Using equation \ref{eq:kruskapw-accept}, the Kruskal-Wallis test rejects the null hypothesis. P-value is approximately zero and this gives a good indication that the result is significant. With a confidence interval of 95\% the author claim that there is a difference between the median value of the three tasks. \\[0.2cm]

 \begin{center}
	\begin{tcolorbox}[box align=center,width=\textwidth-5cm]
		\centering
		\textit{Kruskal-Wallis test, sample 8, 9 and 10}\\
		Significance level: 5\%  \\[0.5cm]
		
		$P-value$: $3.967* 10^{-72}$ \\
		$H-value$: $328.816$ \\
		Significance level ($\alpha$): 0.05 \\
		$v$ = 2\\ %k-1 = 3-1 = 2
		Critical-value:  5.991\\[0.2cm] %http://web.utk.edu/~cwiek/TwoSampleDoF
		
		$H-value$ is significantly lower than the critical value ($328.816$ > $5.991$) and the null hypothesis is \underline{rejected}, $H_A$ is accepted\\[0.5cm]
	\end{tcolorbox} 
\end{center}

\begin{figure}[h!]
	\centering
	\includegraphics[width=0.7\linewidth]{../../thesis-statisticmethods/statistic_analysis/figures/boxplot/mean_std_task123png_correct}
	\caption{Sample 8, 9 and 10 - mean (green dot) and standard deviation (blue line)}
	\label{fig:meanstdtask123pngcorrect}
\end{figure}

Like in one-way \textit{ANOVA}, a \textit{post hoc} test should be used to make paired comparisons to determine which groups differs. Tukey's test resulted in a significant difference in the number of correctly chosen elements between task 1 and task 2, and task 1 and task 3. Figure \ref{fig:meanstdtask123pngcorrect} show that task 1 has a higher mean value than the other two tasks. Task 1 also has a smaller standard deviation than the other tasks. 

 \begin{center}
	\begin{tcolorbox}[box align=center,width=\textwidth-5cm]
		\centering
		\textit{Tukey's test, sample 8, 9 and 10}\\
		Significance level: 5\%  \\[0.5cm]
		
		Task 1 and Task 2 differs significantly \\
		Task 1 and Task 3 differs significantly \\
		Task 2 and Task 3 do not differ significantly \\[0.2cm]
	\end{tcolorbox} 
\end{center}

\vspace{0.5cm}

\subsubsection{Hypothesis test summary}

	\begin{longtable}{p{0.65\textwidth}|l|p{0.15\textwidth}}  %\multicolumn{1}{c}{}
	\caption[Summary, normality tests]{Summary of hypothesis tests done in section \ref{sec:hypothesis_results}} \label{tab:hypothesistest_summary} \\
		Hypothesis (\textcolor{cyan}{Dependent variable}, \textcolor{blue}{Independent variable}) & Sample & Hypothesis is accepted \\[0.2cm] \hline
		\textcolor{cyan}{Total time}, \textcolor{blue}{Experienced} and \textcolor{blue}{Inexperienced} & &  \\
		There are a difference between experienced and inexperienced participants & 1 and 2 & \textbf{Yes} \\
		Experienced participants finish the tasks faster than inexperienced  & 1 and 2 & No   \\ \hline
		\textcolor{cyan}{Correct elements}, \textcolor{blue}{Experienced} and \textcolor{blue}{Inexperienced} & &  \\
		There are a difference between experienced and inexperienced participants & 3 and 4 & No   \\ \hline
		\textcolor{cyan}{Total time}, \textcolor{blue}{Task 1}, \textcolor{blue}{Task 2} and \textcolor{blue}{Task 3}& &  \\
		 Total time is different between at least two of the tasks & 5, 6 and 7 & \textbf{Yes}   \\
		 Task 1 significantly differs from Task 2 & 5, 6 and 7 & No  \\ 
		 Task 1 significantly differs from Task 3 & 5, 6 and 7 & No  \\ 
		 Task 2 significantly differs from Task 3 & 5, 6 and 7 & No  \\ \hline
		\textcolor{cyan}{Correct elements}, \textcolor{blue}{Task 1}, \textcolor{blue}{Task 2}, \textcolor{blue}{Task 3} & &  \\
		The number of correctly chosen elements is different between at least two of the tasks & 8, 9 and 10 & \textbf{Yes}  \\
		Task 1 significantly differs from Task 2 & 8, 9 and 10 & \textbf{Yes}  \\ 
		Task 1 significantly differs from Task 3 & 8, 9 and 10 & \textbf{Yes}  \\ 
		Task 2 significantly differs from Task 3 & 8, 9 and 10 & No  \\ \hline
	\end{longtable}