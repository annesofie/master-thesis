\section{Survey results}\label{sec:survey_results}

- All participants ordered by age, excluded by task 4 \\
- All results in one task, ordered by age \\ 
- Average time per micro-task \\
- Is there a difference in task number 1, 2, 3? time and correct  \\
Can use it to explain the data

\subsection{Gathered data}\label{sec:gathereddata}

The gathered data will be analyzed on the two dependent variables: 1) total time used to complete each task and 2) number of correctly chosen elements per task. Total time and the number of correct elements sums the time and correctly chosen elements on question one and question two. Sample mean $\overline{x}$, sample median, standard deviation of $\overline{x}$, standard error ($\frac{standard deviation}{\sqrt{sample size}}$) of $\overline{x}$, minimum in sample and maximum in sample are listed in the tabled in this section. The sample number is also written in the tables. These numbers are used in the analysis of the data, to easier understand which sample is used in the different analysis. In all samples, results from the training task are removed. Only results from the three tasks are used. Results from participants that was disturbed during the survey were also removed, four of the participants spent more than twice the estimated time on the survey. Maximum possible correct elements per task are twelve. There are six elements in question one and six elements in question two, and the number of correctly chosen elements in each task is added together maximum twelve correctly chosen elements.

% The tables in this section, (\ref{sec:gathereddata}), are task results from all participants and all three tasks, excluding the training task. Task results with total time longer than 2160 seconds are filtered out. This is to remove 4 outliers that spend more than twice the approximated time (average time on the survey was 1080 seconds in the pilot test). These 4 participants also answered that they were disturbed during the test.

 The gathered data are in the first subsection (\ref{sec:alltasks}) divided into experienced and inexperienced. In subsection \ref{sec:taskdivided_all} the data is divided into the three tasks, containing all participants. Section \ref{sec:taskdivided_experienced}  and \ref{sec:taskdivided_inexperienced} divides the data into the three tasks and also in experienced and inexperienced participants. 
 
%Removed all participants that said they was distracted. 26 task results was removed, 10 inexperienced and 18 experienced results.

\subsubsection{All,  experienced and inexperienced participants}\label{sec:alltasks}

Table \ref{tab:totaltime_all} and \ref{tab:totalcorrect_all} are samples containing task results from all, experienced and inexperienced participants. The result is divided into the two dependent variables, total time and the number of correctly chosen elements.

\begin{table}[H]
	\centering
	\begin{tabular}{l|l|l|l}
		\textit{Total time per task (seconds) } & All  & Experienced & Inexperienced \\ 
		Sample number &   & 1  & 2   \\ \hline
		Number of observations & 429    & 229    & 200   \\
		Sample mean $\overline{x}$     & 170.32 & 177.65  & 161.94     \\
		Sample median  & 155.0 & 158.0  & 154.0  \\
		Standard deviation of $\overline{x}$  & 82.19  & 88.24  & 73.99   \\
		Standard error of $\overline{x}$  & 3.98  & 5.83 & 5.23  \\
		Minimum in sample & 38.00  & 52.00  & 38.00     \\
		Maximum in sample & 657.00 & 657.00  & 529.00    \\ \hline
	\end{tabular}
	\caption[Total time, all participants]{Total time spent on each task}
	\label{tab:totaltime_all}
\end{table}

\begin{table}[H]
	\centering
	\begin{tabular}{l|l|l|l}
		\textit{Correct elements per task } & All  & Experienced & Inexperienced \\ 
		Sample number &   & 3  & 4   \\ \hline
		Number of observations & 429    & 229  & 200   \\
		Sample mean $\overline{x}$   & 9.82 & 9.81  & 9.83  \\
		Sample median & 10.0 & 10.0 & 10.0 \\
		Standard deviation of $\overline{x}$   & 1.52  & 1.53  &  1.51 \\
		Standard error of $\overline{x}$   & 0.07  & 0.10 &  0.11 \\
		Minimum in sample & 4.00 & 5.00  &  4.00  \\
		Maximum in sample  & 12.00 & 12.00  & 12.00  \\ \hline
	\end{tabular}
	\caption[Correct elements, all participants]{Number of correctly chosen elements per task}
	\label{tab:totalcorrect_all}
\end{table}

\subsubsection{All participants, divided by task}\label{sec:taskdivided_all}

In table \ref{tab:totaltime_tasks} and \ref{tab:totalcorrect_tasks} the task results is divided into the three different tasks. Task 1 is the task that served the participants with one and one elements. Task 2 is the task that served the participants with three and three elements, and task 3 gave all six elements at the same time. Table  \ref{tab:totaltime_tasks} contains the total time variable, and \ref{tab:totalcorrect_tasks} the number of correct elements variable. 

\begin{table}[H]
	\centering
	\begin{tabular}{l|l|l|l}
		\textit{Total time per task (seconds)} & Task 1 & Task 2 & Task 3 \\ 		
		Sample number & 5  & 6  & 7    \\ \hline
		Number of observations & 146    & 142      & 141     \\
		Sample mean $\overline{x}$  & 166.38  &  172.25   &   172.48  \\
		Sample median & 150.0  &  155.5  & 157.0  \\
		Standard deviation of $\overline{x}$   & 84.57  & 84.21  & 77.95   \\
		Standard error of $\overline{x}$   & 7.00 & 7.07 & 6.56 \\
		Minimum in sample    & 47  & 50 &   38   \\
		Maximum in sample   & 657 & 492  & 529 \\ \hline
	\end{tabular}
	\caption[Total time, divided into task 1, 2 and 3]{Total time divided into task 1, task 2 and task 3}
	\label{tab:totaltime_tasks}
\end{table}

\begin{table}[H]
	\centering
	\begin{tabular}{l|l|l|l}
		\textit{Correct elements per task} & Task 1 & Task 2 & Task 3\\ 
		Sample number & 8  & 9  & 10   \\ \hline
		Number of observations & 146    & 142     & 141        \\
		Sample mean $\overline{x}$ & 10.19  &  9.71  &   9.55   \\
		Sample median & 11.0 &  10.0  &  10.0   \\
		Standard deviation of $\overline{x}$ & 1.43  & 1.53 & 1.52    \\
		Standard error of $\overline{x}$ & 0.12 &  0.13 & 0.13  \\
		Minimum in sample  & 5.00  & 5.00  &   4.00  \\
		Maximum in sample  & 12.00 & 12.00  & 12.00 \\ \hline
	\end{tabular}
	\caption[Correct elements, divided into task 1, task 2 and task 3]{Number of correctly chosen elements divided into task 1, task 2 and task 3}
	\label{tab:totalcorrect_tasks}
\end{table}

\subsubsection{Experienced participants, divided by task}\label{sec:taskdivided_experienced}

In the tables in this section, only task results from experienced participants are included and the result is also divided into the three survey tasks. Table \ref{tab:totaltime_tasks_experienced} is data gathered about experienced participants total time per task. Table \ref{tab:totalcorrect_tasks_experienced} is data gathered about experienced participants number of correctly chosen elements per task.

\begin{table}[H]
	\centering
	\begin{tabular}{l|l|l|l}
		\textit{Total time per task} & Task 1 & Task 2 & Task 3 \\ 
		Sample number & 11  & 12  & 13   \\ \hline
		Number of observations & 77  & 80   & 77  \\
		Sample mean $\overline{x}$  & 173.04  &  176.70  &  181.06   \\
		Sample median  & 158.0  &  156.0  &  165.0  \\
		Standard deviation of $\overline{x}$ & 96.76  & 86.13  & 79.70   \\
		Standard error of $\overline{x}$ & 11.03 & 9.63 & 9.08  \\
		Minimum in sample   & 57.00  & 52.00 &  53.00  \\
		Maximum in sample  & 657.00 & 492.00  & 463.00 \\ \hline
		Sample number & 11  & 12  & 13   \\ \hline
	\end{tabular}
	\caption[Total time, task and experienved divided]{Experienced total time per task, divided by task}
	\label{tab:totaltime_tasks_experienced}
\end{table}

\begin{table}[H]
	\centering
	\begin{tabular}{l|l|l|l}
		\textit{Correct elements per task} & Task 1 & Task 2 & Task 3 \\ 
		Sample number & 14 & 15  & 16   \\ \hline
		Number of observations & 77    & 80      &  77  \\
		Sample mean $\overline{x}$ & 10.29  &  9.66  &  9.48   \\
		Sample median & 11.0  &  10.0  &  10.0  \\
		Standard deviation of $\overline{x}$ & 1.32  & 1.64  & 1.47   \\
		Standard error of $\overline{x}$ & 0.15  & 0.18  & 0.17   \\
		Minimum in sample & 7.00 & 5.00 &  5.00 \\
		Maximum in sample  & 12.00 & 12.00  & 12.00 \\ \hline
	\end{tabular}
	\caption[Correct elements, task and experienved divided]{Experienced number of correct elements per task, divided by task}
	\label{tab:totalcorrect_tasks_experienced}
\end{table}

\subsubsection{Inexperienced participants, divided by task}\label{sec:taskdivided_inexperienced}

In this section, the task results from only inexperienced participants are included and the result is also divided into the three survey tasks. 
Table \ref{tab:totaltime_tasks_inexperienced} is the total time variable and \ref{tab:totalcorrect_tasks_inexperienced} the number of correctly chosen elements. 

\begin{table}[H]
	\centering
	\begin{tabular}{l|l|l|l}
		\textit{Total time per task (seconds)} & Task 1 & Task 2 & Task 3 \\ 
		Sample number & 17 & 18 & 19 \\ \hline
		Number of observations & 71    & 64  & 65   \\
		Sample mean $\overline{x}$  & 158.30  &  165.69  &  162.23  \\
		Sample median & 148.0  &  154.5  &  154.0  \\
		Standard deviation of $\overline{x}$  & 67.57 & 80.93 & 74.53  \\
		Standard error of $\overline{x}$  & 8.02  & 10.12 & 9.24  \\
		Minimum in sample & 47.00 & 50.00 &  38.00 \\
		Maximum in sample & 487.00 & 455.00  & 529.00  \\ \hline
	\end{tabular}
	\caption[Total time, inexperienced per task]{Inexperienced total time per task, divided by task}
	\label{tab:totaltime_tasks_inexperienced}
\end{table}

\begin{table}[H]
	\centering
	\begin{tabular}{l|l|l|l}
		\textit{Correct elements per task} & Task 1 & Task 2 & Task 3 \\ 
		Sample number & 20 & 21 & 22 \\ \hline
		Number of observations & 71 & 64  & 65 \\
		Sample mean $\overline{x}$  & 10.07  &  9.78 &  9.61  \\
		Sample median  & 10.0  & 10.0  &  10.0  \\
		Standard deviation of $\overline{x}$  & 1.54  & 1.38  & 1.57   \\
		Standard error of $\overline{x}$  & 0.18 & 0.17 & 0.19  \\
		Minimum in sample  & 5.00 & 6.00 &  4.00  \\
		Maximum in sample  & 12.00 & 12.00  & 12.00 \\ \hline
	\end{tabular}
	\caption[Correct elements, inexperienced per task]{Inexperienced number of correct elements per task, divided by task}
	\label{tab:totalcorrect_tasks_inexperienced}
\end{table}

\subsection{Normality tests}\label{sec:normality_results}
To check if a two-sample t-test (subsection \ref{sec:t-test}) and \textit{ANOVA}-test (subsection \ref{sec:anova}) can be used, the samples need to be tested if they are normally distributed or not. Both tests assume normally distributed samples. The normality section \ref{sec:normaltesting} concluded that the D'Agostino and Person normality test should be used in this thesis. A visual interpretation of histograms will also be a part of the normality tests. The D'Agostino-Pearson test uses the following hypothesis:\newline

\centerline{$H_{0}$: The data follows the normal distribution} 
\centerline{$H_{A}$: The data do not follow the normal distribution}


\subsubsection[Sample 1 and 2]{Experienced and inexperienced participants, total time variable}\label{sec:sample1,2_normresult}
This section will test if sample 1 and 2 (table \ref{tab:totaltime_all}) are normally distributed. Sample 1 and 2 will be used to test if there are any difference on time spent per task between experienced and inexperienced participants. 

A visual interpretation of the histograms \ref{fig:totaltimeexclude4_experienced} and \ref{fig:totaltimeexclude4_inexperienced} gives an indication that sample 1 and 2 are not normally distributed. Both histograms are positively skewed (figure \ref{fig:skew}). Samples involving time measurements are rarely normally distributed. This is because the samples will always be skewed since it is impossible to have negative time and there will always be a limit to how fast a participant can finish the task. 

\begin{figure}[H]
	\centering
	\begin{subfigure}[b]{0.48\textwidth}
		\centering
		\includegraphics[width=\linewidth]{../../thesis-statisticmethods/statistic_analysis/figures/experienced_participants/normalplot/totaltime_exclude4}
		\caption{Sample 1 - Experienced}
		\label{fig:totaltimeexclude4_experienced}
	\end{subfigure}
	\begin{subfigure}[b]{0.48\textwidth}
		\centering
		\includegraphics[width=\linewidth]{../../thesis-statisticmethods/statistic_analysis/figures/inexperienced_participants/normalplot/totaltime_exclude4}
		\caption{Sample 2 - Inexperienced}
		\label{fig:totaltimeexclude4_inexperienced}
	\end{subfigure}
\caption{Histograms with normal distribution fit with samples containing total time to complete each task}
\end{figure}

A D'Agostino and Pearson normality test confirmed the visual interpretation with an significance level of 5\% (0.05). Both samples have p-values lower than the significance level. They are not normally distributed with a confidence level of 95\%. \\[0.5cm]

\begin{center}
	\begin{tcolorbox}[width=0.80\textwidth]
		\centering
				\textit{D'Agostino and Pearson normality test}\\
				Significance level: 5\%  \\[0.5cm]
	
				Sample 1\\
				P-value: $3.874 * 10^{-22}$\\
				The p-value is lower than the significance level (0.05), the null hypothesis is \underline{rejected} and $H_A$ accepted.\\[0.5cm]
				
				Sample 2\\
				P-value: $2.574 * 10^{-21}$ \\
				The p-value is lower than the significance level (0.05), the null hypothesis is \underline{rejected} and $H_A$ accepted.\\[0.5cm]
	\end{tcolorbox} 
\end{center}
%Boc-Cox Transformation? https://docs.scipy.org/doc/scipy-0.19.0/reference/generated/scipy.stats.boxcox.html https://www.isixsigma.com/tools-templates/normality/dealing-non-normal-data-strategies-and-tools/

In both sample 1 and 2, the p-value was significantly lower than the significance level of 0.05. Data transformations are commonly used tools to improve normality of a sample's distributions, but there are many types of data transformations. \cite{Osborne2010} claim that almost all tests, even non-parametric tests, benefit from improving the normality of the samples, especially when the normality test is significantly denied. Common traditional transformations are square root, inverse or converting to logarithmic scales  \citep{Osborne2010}. 

A Box-Cox power transformation is used in this thesis. This transformation can be used on positive data and the data used in this thesis will never negative. Box-Cox takes the idea of having a range of power transformations (square root $x^{\frac{1}{2}}$, inverse $x^{-1}$ etc.) available to improve the effectiveness of normalizing and variance equalizing for both positively- and negatively-skewed variables \citep{Osborne2010}. This transformation will always use the appropriate transformation to be maximally effective in moving each sampled data towards normality. This is the reason why this thesis will use the Box-Cox transformation.

Sample 1 and 2 after Box-Cox transformation is shown in histogram \ref{fig:totaltimeboxcoxtransformation_experienced} and \ref{fig:totaltimeboxcoxtransformedtitle_inexperienced}. A visual inspection gives a good indication that the transformed data is normally distributed.

\begin{figure}[H]
	\centering
	\begin{subfigure}[b]{0.48\textwidth}
		\centering
		\includegraphics[width=\linewidth]{../../thesis-statisticmethods/statistic_analysis/figures/experienced_participants/normalplot/totaltime_boxcox_transformation}
		\caption[Experienced, Box-Cox]{Sample 1 - Experienced participants}
		\label{fig:totaltimeboxcoxtransformation_experienced}
	\end{subfigure}
	\begin{subfigure}[b]{0.48\textwidth}
		\centering
		\includegraphics[width=\linewidth]{../../thesis-statisticmethods/statistic_analysis/figures/inexperienced_participants/normalplot/totaltime_boxcox_transformed_title}
		\caption[Inexperienced, Box-Cox]{Sample 2 - Inexperienced participants}
		\label{fig:totaltimeboxcoxtransformedtitle_inexperienced}
	\end{subfigure}
\caption{Histograms with normal distribution fit after Box-Cox transformation}
\end{figure}

The D'Agostino and Pearson normality method is then tested on the transformed data. This test confirms the visual analysis, both sample 1 and sample 2 are normally distributed after the Box-Cox transformation with a confidence level of 95\%. The calculated p-value is larger than the significance level (0.05). \\[0.5cm] 

\begin{center}
	\begin{tcolorbox}[width=0.8\textwidth]
		\centering
		\textit{D'Agostino and Pearson normality test}\\
		(After Box-Cox transformation) \\
		Significance level: 5\%  \\[0.5cm]
		
		Sample 1: Experienced, total time per task\\
		P-value: $0.849$\\
		The p-value is higher than the significance level (0.05), the null hypothesis is \underline{accepted}. \\[0.5cm]
		
		Sample 2: Inexperienced, total time per task \\ %*Her bruker jeg data hvor was interupted er fjernet
		P-value: $0.0623$ \\
		The p-value is higher than the significance level (0.05), the null hypothesis is \underline{accepted}. \\[0.5cm]
	\end{tcolorbox}
\end{center}

The assumption that sample 1 and sample 2 are normally distributed is now accepted and can be used in parametric methods when using these samples.

\subsubsection[Sample 3 and 4]{Experienced and inexperienced participants, number of correctly chosen elements variable}\label{sec:correct_ex_inex}
%DIscrete variables http://stattrek.com/probability-distributions/discrete-continuous.aspx
This section will test if sample 3 and 4 (table \ref{tab:totalcorrect_all}) are normally distributed. Sample 3 and 4 will be used to test if there are any difference on the number of correct elements per task between experienced and inexperienced participants. 

A visual inspection of the samples histogram \ref{fig:correctelementswasnotinterupter_ex} and \ref{fig:correctelementswasnotinterupted_inex} gives a good indication that sample 3 and 4 are not normally distributed. Both are negatively skewed (figure \ref{fig:skew}).   

\begin{figure}[h!]
	\centering
	\begin{subfigure}[b]{0.48\textwidth}
		\centering
		\includegraphics[width=\linewidth]{../../thesis-statisticmethods/statistic_analysis/figures/experienced_participants/normalplot/correct_elements_was_not_interupter}
		\caption{Sample 3 - Experienced}
		\label{fig:correctelementswasnotinterupter_ex}
	\end{subfigure}
	\begin{subfigure}[b]{0.48\textwidth}
		\centering
		\includegraphics[width=\linewidth]{../../thesis-statisticmethods/statistic_analysis/figures/inexperienced_participants/normalplot/correct_elements_was_not_interupted}
		\caption{Sample 4 - Inexperienced}
		\label{fig:correctelementswasnotinterupted_inex}
	\end{subfigure}
	\caption{Histograms with normal distribution fit with samples containing the number of correctly chosen elements in each task}
\end{figure}

D'Agostino and Pearson normality test confirm our visual analysis. Both samples accept the alternative hypothesis with p-values ($0.00443$, $0.00013$) lower than the significant level $0.05$. The null hypothesis is \underline{rejected} and $H_A$ accepted for sample 3 and 4. \\[0.3cm]
\vspace{0.3cm}

Sample 3 and 4 was Box-Cox power transformed because the null hypothesis was rejected. After the transformation, a new D'Agostino and Pearson normality test was done. Both samples also failed this test. Sample 3 and 4 are not normally distributed and need to be tested with non-parametric methods. 

\subsubsection[Sample 5, 6 and 7]{All participants divided by task, total time variable}\label{sec:task123_time_normaltest}

In this section sample 5, 6 and 7 (table \ref{tab:totaltime_tasks}) is normality tested. These samples will be used to test whether there is a significant difference between the three tasks when looking at the total time variable. 

A visual analysis of the three histograms in figure \ref{fig:totaltimeallnotinterupted_task1}, \ref{fig:totaltimeallnotinterupted_task2} and \ref{fig:totaltimeallnotinterupted_task3} show a positive skewness, just like the histograms in section \ref{sec:totaltime_ex_inex}. This gives an indication that the three samples are not normally distributed. 

\begin{figure}[h!]
	\centering
	\begin{subfigure}[b]{0.3\textwidth}
		\centering
		\includegraphics[width=\linewidth]{../../thesis-statisticmethods/statistic_analysis/figures/task1/totaltime_all_not_interupted}
		\caption{Sample 5 - Task 1}
		\label{fig:totaltimeallnotinterupted_task1}
	\end{subfigure}
	\begin{subfigure}[b]{0.3\textwidth}
		\centering
		\includegraphics[width=\linewidth]{../../thesis-statisticmethods/statistic_analysis/figures/task2/totaltime_all_not_interupted}
		\caption{Sample 6 - Task 2}
		\label{fig:totaltimeallnotinterupted_task2}
	\end{subfigure}
	\begin{subfigure}[b]{0.3\textwidth}
		\centering
		\includegraphics[width=\linewidth]{../../thesis-statisticmethods/statistic_analysis/figures/task3/totaltime_all_not_interupted}
		\caption{Sample 7 - Task 3}
		\label{fig:totaltimeallnotinterupted_task3}
	\end{subfigure}
	\caption{Histogram with normal distribution fit - sample with total time per task}
\end{figure}

The D'Agostino and Pearson normality test agreed with the visual analysis. Obtained p-values for all three samples ($2.39 * 10^{-24}$, $2.57 * 10^{-9}$ and $1.71 * 10^{-11}$) are smaller than the significance level 0.05, and the null hypothesis is rejected. The samples are not normally distributed with a significant level of 5\%.\\[0.2cm]

Because the null hypothesis was rejected the samples are Box-Cox power transformed. Histograms of each sample after the transformation is shown in figure \ref{fig:totaltimeallnotinteruptedboxcox_task1}, \ref{fig:totaltimeallnotinteruptedboxcox_task2} and \ref{fig:totaltimeallnotinteruptedboxcox_task3}. A visual analysis of the histograms gives a good indication that the transformed data is approximately normally distributed. The histograms have a skewness of approximately zero. 

\begin{figure}[h!]
	\centering
	\begin{subfigure}[b]{0.3\textwidth}
		\centering
		\includegraphics[width=\linewidth]{../../thesis-statisticmethods/statistic_analysis/figures/task1/totaltime_all_not_interupted_boxcox}
		\caption{Sample 5 - Task 1}
		\label{fig:totaltimeallnotinteruptedboxcox_task1}
	\end{subfigure}
	\begin{subfigure}[b]{0.3\textwidth}
		\centering
		\includegraphics[width=\linewidth]{../../thesis-statisticmethods/statistic_analysis/figures/task2/totaltime_all_not_interupted_boxcox}
		\caption{Sample 6 - task 2}
		\label{fig:totaltimeallnotinteruptedboxcox_task2}
	\end{subfigure}
	\begin{subfigure}[b]{0.3\textwidth}
		\centering
		\includegraphics[width=\linewidth]{../../thesis-statisticmethods/statistic_analysis/figures/task3/totaltime_all_not_interupted_boxcox}
		\caption{Sample 7 - task 3}
		\label{fig:totaltimeallnotinteruptedboxcox_task3}
	\end{subfigure}
	\caption{Histogram with normal distribution fit after Box-Cox transformation, sample with total time per task}
\end{figure}

The D'Agostino and Pearson normality test confirms the visual interpretation. The data is normally distributed after the Box-Cox transformation with a confidence interval of 95\%. The p-values ($0.164$, $0.982$ and $0.354$) of all three samples are higher than the significance level (0.05). \\[0.2cm]. Sample 5, 6 and 7 are normally distributed after the transformation and the assumptions of normality is met. 
 
 \subsubsection[Sample 8, 9 and 10]{All participants divided by task,  correct element variable}\label{sec:task123_correct_normaltest}
 In this section sample 8, 9 and 10 will be normality tested. These samples will be used to test whether there is a significant difference between the three tasks when looking at the number of correctly chosen elements variable. This variable tells us how many correct elements each participant chose on each of the three tasks. 

A visual analysis of the three histograms in figure \ref{fig:correctallnotinterupted_task1}, \ref{fig:correctallnotinterupted_task2} and \ref{fig:correctallnotinterupted_task3} show a negative skewness, just like the histograms in section \ref{sec:correct_ex_inex}. This give an indication that the three samples are not normally distributed. 
 
 \begin{figure}[H]
 	\centering
	 \begin{subfigure}[b]{0.3\textwidth}
	 	\centering
	 	\includegraphics[width=\linewidth]{../../thesis-statisticmethods/statistic_analysis/figures/task1/correct_all_not_interupted}
	 	\caption{Sample 8 -Task 1}
	 	\label{fig:correctallnotinterupted_task1}
	 \end{subfigure}
	\begin{subfigure}[b]{0.3\textwidth}
		\centering
		\includegraphics[width=\linewidth]{../../thesis-statisticmethods/statistic_analysis/figures/task2/correct_all_not_interupted}
		\caption{Sample 9 - Task 2}
		\label{fig:correctallnotinterupted_task2}
	\end{subfigure}
	 \begin{subfigure}[b]{0.3\textwidth}
	 	\centering
	 	\includegraphics[width=\linewidth]{../../thesis-statisticmethods/statistic_analysis/figures/task3/correct_all_not_interupted}
	 	\caption{Sample 10 - Task 3}
	 	\label{fig:correctallnotinterupted_task3}
	 \end{subfigure}
 \caption{Histogram with normal distribution fit showing samples with number of correct elements per task}
 \end{figure}
 
 D'Agostino and Pearson normality test confirms our visual analysis of the histograms in two of three samples. Sample 9 passes the normality test, even though the p-value (0.099) is close to the significance level (0.05). Sample 8 and sample 10 do not pass the normality test with a confidence interval of 95\%. Both samples obtained a p-value ($0.00022$ and $0.0047$) smaller than the significant level. The null hypothesis is rejected for sample 8 and 10, and the alternative hypothesis is accepted. The null hypothesis is accepted for sample 9. \\[0.2cm]

A Box-Cox power transformation is applied to all three samples. The transformation changes the data, and to correctly compare the samples, sample 9 also has to be transformed even though it's original data is normally distributed. The transformed samples are shown in histogram \ref{fig:correctboxcox_task1}, \ref{fig:correctboxcox_task2} and \ref{fig:correctboxcox_task3}. All three are negatively skewed, sample 9 and 10 less than sample 8. A visual conclusion is difficult in this case. \\[0.2cm]


\begin{figure}[h!]
	\centering
	\begin{subfigure}[b]{0.3\textwidth}
		\centering
		\includegraphics[width=\linewidth]{../../thesis-statisticmethods/statistic_analysis/figures/task1/correct_boxcox}
		\caption{Sample 8 - Task 1}
		\label{fig:correctboxcox_task1}
	\end{subfigure}
	\begin{subfigure}[b]{0.3\textwidth}
		\centering
		\includegraphics[width=\linewidth]{../../thesis-statisticmethods/statistic_analysis/figures/task2/correct_boxcox}
		\caption{Sample 9 - Task 2}
		\label{fig:correctboxcox_task2}
	\end{subfigure}
	\begin{subfigure}[b]{0.3\textwidth}
		\centering
		\includegraphics[width=\linewidth]{../../thesis-statisticmethods/statistic_analysis/figures/task3/correct_boxcox}
		\caption{Sample 10 - Task 3}
		\label{fig:correctboxcox_task3}
	\end{subfigure}
	\caption{Histogram with normal distribution fit showing samples with number of correct elements per task - after Box-Cox transformation}
\end{figure}

D'Agostino and Pearson normality test accepts null hypothesis on sample 9 and 10 and rejects it on sample 8. Sample 9 and 10 has p-values ($0.0752$ and$0.2104$) higher than the significance level , while sample 8 p-value ($0.2104$) is significantly lower. When using these three samples in hypothesis tests a non-parametric method should be used. This is because sample 8 is not normally distributed. 

\subsubsection[Sample 11, 12 and 13]{Experienced participants divided by task, total time variable}
In this section, sample 11, 12 and 13 are normality tested. These samples will be used to test whether there is s significant difference between the three tasks total time results when looking at only experienced participants. 

A visual interpretation of the histograms in \ref{fig:sample11_12_13_histogram} show that all three samples are positively skewed (figure \ref{fig:skew}). This is a fairly strong evidence that the samples are not normally distributed.

 \begin{figure}[H]
 	\centering
	 \begin{subfigure}[b]{0.32\textwidth}
	 	\centering
	 	\includegraphics[width=\linewidth]{../../thesis-statisticmethods/statistic_analysis/figures/task1/totaltime_experienced}
	 	\caption{Sample 11 - task 1}
	 	\label{fig:totaltimeexperienced_task1}
	 \end{subfigure}
	 \begin{subfigure}[b]{0.32\textwidth}
	 	\centering
	 	\includegraphics[width=\linewidth]{../../thesis-statisticmethods/statistic_analysis/figures/task2/totaltime_experienced}
	 	\caption{Sample 12 - task 2}
	 	\label{fig:totaltimeexperienced_task2}
	 \end{subfigure}
	 \begin{subfigure}[b]{0.32\textwidth}
	 	\centering
	 	\includegraphics[width=\linewidth]{../../thesis-statisticmethods/statistic_analysis/figures/task3/totaltime_experienced}
	 	\caption{Sample 13 - task 3}
	 	\label{fig:totaltimeexperienced}
	 \end{subfigure}
 	\caption{Histograms with normal distribution fit with samples containing total time to complete each task}
 	\label{fig:sample11_12_13_histogram}
 \end{figure}
 
D'Agostino and Pearson normality test confirms our visual interpretation of the three histograms. All three p-values ($1.229 * 10^{-14}$, $2.678 * 10^{-5}$ and $0.000884$) are lower than the significance level (0.05). Sample 11, 12 and 13 do not pass the normality test with a confidence interval of 95\%. The null hypothesis is rejected. \\[0.2cm]

A Box-Cox power transformation is applied to all three samples because the null hypothesis was rejected. Histograms with normal distribution fit containing the transformed data is shown in figure \ref{fig:sample11_12_13_boxcox_histogram}. Visually, the histograms look normally distributed with minimal skewness. \\

\begin{figure}[H]
	\centering
	\begin{subfigure}[b]{0.32\textwidth}
		\centering
		\includegraphics[width=\linewidth]{../../thesis-statisticmethods/statistic_analysis/figures/task1/totaltime_experienced_boxcox}
		\caption{Sample 11 - task 1}
		\label{fig:totaltimeexperiencedboxcox_task1}
	\end{subfigure}
	\begin{subfigure}[b]{0.32\textwidth}
		\centering
		\includegraphics[width=\linewidth]{../../thesis-statisticmethods/statistic_analysis/figures/task2/totaltime_experienced_boxcox}
		\caption{Sample 12 - task 2}
		\label{fig:totaltimeexperiencedboxcox_task2}
	\end{subfigure}
	\begin{subfigure}[b]{0.32\textwidth}
		\centering
		\includegraphics[width=\linewidth]{../../thesis-statisticmethods/statistic_analysis/figures/task3/totaltime_experienced_boxcox}
		\caption{Sample 13 - task 3}
		\label{fig:totaltimeexperiencedboxcox_task3}
	\end{subfigure}
	\caption{Histograms with normal fit containing Box-Cox transformed data}
	\label{fig:sample11_12_13_boxcox_histogram}
\end{figure}

The Box-Cox power transformed data is tested with D'Agostino and Pearson normality test. All three samples obtained p-values ($0.694$, $0.955$ and $0.887$) larger than the significance level (0.05). Within a confidence interval of 95\%, the test concludes that sample 11, 12 and 13 is normally distributed. Sample 11, 12 and 13 can be used in methods assuming normally distributed samples. 

\subsubsection[Sample 14, 15 and 16]{Experienced participants divided by task, correct elements variable}
This section will test if sample 14, 15 and 16 follows the normal distribution. These samples will be used to test whether there is a significant difference in the number of correct elements between the three tasks when looking at only experienced participants. Before testing the hypothesis the samples need to be normality tested. 

A visual interpretation of the histograms in \ref{fig:sample14,15,16_normhistogram} show that all three samples are slightly negatively skewed (figure \ref{fig:skew}). The skew is not as much as other histograms in this chapter. Sample 15 (\ref{fig:correctexperienced_task1}) and sample 16 (\ref{fig:correctexperienced_task2}) has less skew than sample 14 (\ref{fig:correctexperienced_task3}).  

\begin{figure}[H]
	\centering
	\begin{subfigure}[b]{0.32\textwidth}
		\centering
		\includegraphics[width=\linewidth]{../../thesis-statisticmethods/statistic_analysis/figures/task1/correct_experienced}
		\caption{Sample 14 - Task 1}
		\label{fig:correctexperienced_task1}
	\end{subfigure}
	\begin{subfigure}[b]{0.32\textwidth}
		\centering
		\includegraphics[width=\linewidth]{../../thesis-statisticmethods/statistic_analysis/figures/task2/correct_experienced}
		\caption{Sample 15 - Task 2}
		\label{fig:correctexperienced_task2}
	\end{subfigure}
	\begin{subfigure}[b]{0.32\textwidth}
		\centering
		\includegraphics[width=\linewidth]{../../thesis-statisticmethods/statistic_analysis/figures/task3/correct_experienced}
		\caption{Sample 16 - Task 3}
		\label{fig:correctexperienced_task3}
	\end{subfigure}
	\caption{Histogram with normal distribution fit showing samples with number of correct elements results for experienced participants}
	\label{fig:sample14,15,16_normhistogram}
\end{figure}

  D'Agostino and Pearson normality test confirms our visual interpretation of the three histograms. All three p-values ($0.0588$, $0.2067$ and $0.2975$) are higher than the significant level (0.05), notice that sample 14 has a lower p-value than the two other samples. This sample is not as significant as the two other samples. Sample 14, 15 and 16 pass the normality test with a confidence interval of 95\%. The null hypothesis is accepted. Sample 14, 15 and 16 can be used in tests that assume normally distributed samples.

\subsubsection[Sample 17, 18 and 19]{Inexperienced participants divided by task, total time variable}\label{sec:sample_17,18,19_normalitytest}
This section will test if sample 17, 18 and 19 follows the normal distribution. These samples will be used to test whether there is a significant difference in total time between the three tasks when looking at only inexperienced participants. Before testing the hypothesis the samples need to be normality tested. 

A visual analysis of the histograms \ref{fig:totaltimeinexperienced_task1}, \ref{fig:totaltimeinexperienced_task2} and \ref{fig:totaltimeinexperienced_task3} shows a positive skew. The skew is less than the histograms in figure \ref{fig:sample11_12_13_histogram}, but is probably still too large for the samples to be normally distributed.  

\begin{figure}[H]
	\centering
	\begin{subfigure}[b]{0.32\textwidth}
		\centering
		\includegraphics[width=\linewidth]{../../thesis-statisticmethods/statistic_analysis/figures/task1/totaltime_inexperienced}
		\caption{Sample 17 - Task 1}
		\label{fig:totaltimeinexperienced_task1}
	\end{subfigure}
	\begin{subfigure}[b]{0.32\textwidth}
		\centering
		\includegraphics[width=\linewidth]{../../thesis-statisticmethods/statistic_analysis/figures/task2/totaltime_inexperienced}
		\caption{Sample 18 - Task 2}
		\label{fig:totaltimeinexperienced_task2}
	\end{subfigure}
	\begin{subfigure}[b]{0.32\textwidth}
		\centering
		\includegraphics[width=\linewidth]{../../thesis-statisticmethods/statistic_analysis/figures/task3/totaltime_inexperienced}
		\caption{Sample 19 - Task 3}
		\label{fig:totaltimeinexperienced_task3}
	\end{subfigure}
	\caption{Histogram with normal distribution fit}
\end{figure}

The D'Agostino and Pearson normality test agrees with the visual analysis. The three obtained p-values ($1.586 * 10^{-11}$, $1.773 * 10^{-6}$ and $2.312 * 10 ^{-11}$) are all significantly lower than the significance level of 0.05. Sample 17, 18 and 19 is not normally distributed with a confidence interval of 95\%. The null hypothesis is rejected. \\[0.2cm]

The samples is Box-Cox power transformed. The histograms after transformation (\ref{fig:totaltimeinexperiencedboxcox_task1}, \ref{fig:totaltimeinexperiencedboxcox_task2} and \ref{fig:totaltimeinexperiencedboxcox_task3}) looks normal distributed. 

\begin{figure}[H]
	\centering
	\begin{subfigure}[b]{0.32\textwidth}
		\centering
		\includegraphics[width=\linewidth]{../../thesis-statisticmethods/statistic_analysis/figures/task1/totaltime_inexperienced_boxcox}
		\caption{Sample 17 - Task 1}
		\label{fig:totaltimeinexperiencedboxcox_task1}
	\end{subfigure}
	\begin{subfigure}[b]{0.32\textwidth}
		\centering
		\includegraphics[width=\linewidth]{../../thesis-statisticmethods/statistic_analysis/figures/task2/totaltime_inexperienced_boxcox}
		\caption{Sample 18 - Task 2}
		\label{fig:totaltimeinexperiencedboxcox_task2}
	\end{subfigure}
	\begin{subfigure}[b]{0.32\textwidth}
		\centering
		\includegraphics[width=\linewidth]{../../thesis-statisticmethods/statistic_analysis/figures/task3/totaltime_inexperienced_boxcox}
		\caption{Sample 19 - Task 3}
		\label{fig:totaltimeinexperiencedboxcox_task3}
	\end{subfigure}
	\caption{Histogram with normal distribution fit after Box-Cox transformation}
\end{figure}

A new D'Agostino and Pearson normality test on the transformed data confirms that all three samples are normally distributed with a significance level of 5\%. The obtained p-values ($0.139$, $0.909$ and $0.067$) are smaller than 0.05, the null hypothesis is accepted. Sample 17, 18 and 19 are normally distributed after Box-Cox power transformation. \\[0.2cm]

\subsubsection[Sample 20, 21 and 22]{Inexperienced participants  divided by task, correct elements variable}\label{sec:sample_20,21,22_normalitytest}
Sample 20, 21 and 22 contains the number of correct elements in each of the three tasks from only inexperienced participants. The samples will be used to test if inexperienced participants do better in one of the tasks. Before analysing, the samples needs to be normally tested. 

A visual interpretation of 

\begin{figure}[H]
	\centering
	\begin{subfigure}[b]{0.32\textwidth}
		\centering
		\includegraphics[width=\linewidth]{../../thesis-statisticmethods/statistic_analysis/figures/task1/correct_inexperienced}
		\caption{Sample 20 - Task 1}
		\label{fig:correctinexperienced_task1}
	\end{subfigure}
	\begin{subfigure}[b]{0.32\textwidth}
		\centering
		\includegraphics[width=\linewidth]{../../thesis-statisticmethods/statistic_analysis/figures/task2/correct_inexperienced}
		\caption{Sample 21 - Task 2}
		\label{fig:correctinexperienced_task2}
	\end{subfigure}
	\begin{subfigure}[b]{0.32\textwidth}
		\centering
		\includegraphics[width=\linewidth]{../../thesis-statisticmethods/statistic_analysis/figures/task3/correct_inexperienced}
		\caption{Sample 22 - Task 3}
		\label{fig:correctinexperienced_task3}
	\end{subfigure}
	\caption{Histogram with normal distribution fit}
\end{figure}

The D'Agostino and Pearson normality test rejects the null hypothesis on sample 20 and 22 and accepts the null hypothesis on sample 21. Obtained p-values ($0.007$ and $0.004$) are lower than 0.05 for sample 20 and 22 and for sample 21 ($0.523$) higher than 0.05 The test concludes that sample 21 are normally distributed and sample 20 and 22 is not normally distributed with a significant level of 5\%. \newline

A Box-Cox power transformation is applied to all three samples. The transformation changes the data, and to correctly compare the samples, sample 21 also has to be transformed even though it's original data is normally distributed. The transformed samples are shown in histogram \ref{fig:correctinexperiencedboxcox_task1}, \ref{fig:correctinexperiencedboxcox_task2} and \ref{fig:correctinexperiencedboxcox_task3}. All three are slightly negatively skewed. The resulting histograms is shown in figure \ref{fig:correctinexperiencedboxcox_task1}, \ref{fig:correctinexperiencedboxcox_task2} and \ref{fig:correctinexperiencedboxcox_task3}. All three histograms are less skewed than the original histograms.

\begin{figure}[H]
	\centering
	\begin{subfigure}[b]{0.32\textwidth}
		\centering
		\includegraphics[width=\linewidth]{../../thesis-statisticmethods/statistic_analysis/figures/task1/correct_inexperienced_boxcox}
		\caption{Sample 20 - Task 1}
		\label{fig:correctinexperiencedboxcox_task1}
	\end{subfigure}
	\begin{subfigure}[b]{0.32\textwidth}
		\centering
		\includegraphics[width=\linewidth]{../../thesis-statisticmethods/statistic_analysis/figures/task2/correct_inexperienced_boxcox}
		\caption{Sample 21 - Task 2}
		\label{fig:correctinexperiencedboxcox_task2}
	\end{subfigure}
	\begin{subfigure}[b]{0.32\textwidth}
		\centering
		\includegraphics[width=\linewidth]{../../thesis-statisticmethods/statistic_analysis/figures/task3/correct_inexperienced_boxcox}
		\caption{Sample 22 - Task 3}
		\label{fig:correctinexperiencedboxcox_task3}
	\end{subfigure}
	\caption{Histogram with normal distribution fit after Box-Cox transformation}
\end{figure}

The D'Agostino and Pearson normality method is tested on the transformed sample data. The three obtained p-values ($0.061$, $0.714$ and $0.311$) is lower than 0.05. Sample 20, 21 and 22 are normally distributed with a significant level of 5\%. The null hypothesis is accepted. 

\subsubsection{Normality test summary}\label{sec:normaltest_summary}

	\begin{longtable}{p{0.30\textwidth}|l|p{2cm}|p{0.25\textwidth}}
	\caption[Summary, normality tests]{Summary of normality tests done in section \ref{sec:normality_results}} \label{tab:normaltest_summary} \\
		  & Sample & Normally distributed  & Normally distributed after Box-Cox  \\ \hline
		\textit{Total time} & & & \\
		Experienced & 1 &No   & \textbf{Yes}   \\
		Inexperienced  & 2 & No & \textbf{Yes}     \\ \hline
		\textit{Correct elements} & & & \\
		Experienced & 3 & No  & No   \\
		Inexperienced  & 4 & No & No   \\ \hline
		\textit{Total time }& & & \\
		Task 1 & 5 &No  & \textbf{Yes}  \\
		Task 2 & 6 &No  & \textbf{Yes}   \\
		Task 3 & 7 & No & \textbf{Yes}  \\ \hline
		\textit{Correct elements} & & & \\
		Task 1 & 8 & No  & No  \\
		Task 2 & 9 &\textbf{Yes}  & \textbf{Yes}   \\
		Task 3 & 10 & No & \textbf{Yes}  \\ \hline
		\textit{Total time, experienced participants} & & & \\
		Task 1 & 11 & No  & \textbf{Yes}  \\
		Task 2 & 12 & No  & \textbf{Yes}   \\
		Task 3 & 13 & No & \textbf{Yes}  \\ \hline
		\textit{Correct elements, experienced participants} & & & \\
		Task 1 & 14 & \textbf{Yes}  & \textit{not tested} \\
		Task 2 & 15 & \textbf{Yes}  &  \textit{not tested} \\
		Task 3 & 16 & \textbf{Yes} & \textit{not tested} \\ \hline
		\textit{Total time, inexperienced participants} & & & \\
		Task 1 & 17& No  & \textbf{Yes}  \\
		Task 2 & 18 & No  & \textbf{Yes}   \\
		Task 3 & 19 & No & \textbf{Yes}  \\ \hline
		\textit{Correct elements, inexperienced participants} & & & \\
		Task 1 & 20 & No  & \textbf{Yes} \\
		Task 2 & 21 & \textbf{Yes}  & \textbf{Yes} \\
		Task 3 & 22 & No & \textbf{Yes} \\ \hline
	\end{longtable}

\vspace{0.5cm}

\subsection{Levene's test of Equality of Variance}\label{sec:levene_test_results}
As mentioned in section \ref{sec:t-test}, \ref{sec:anova} and \ref{sec:mannwhiteyu}, the two sample t-test, one-way \textit{ANOVA} and Mann-Whitey U test assumes that the samples come from populations with equal variances. This assumption will be examined with Levene's test. The hypothesis tested is: \\[0.3cm]

\centerline{$H_{0}$: Input samples are from populations with equal variances} 
\centerline{$H_{A}$: Input samples are from populations that do not have equal variances}
\vspace{0.3cm}

The null hypothesis is accepted if the obtained p-value is higher than the significance level. Table \ref{tab:levenestest_summary} contains the summary of the Levene's test performed on all sample pairs. All sample pairs accepted the null hypothesis accept sample 1 and 2, who obtained a p-value lower than the significance lecel. The test used a significance level of 5\% on all the tests. \\[0.2cm]

\begin{longtable}{p{0.5\textwidth}|l|p{0.25\textwidth}}
	\caption[Summary, Levene's tests]{Summary of Levene's tests} \label{tab:levenestest_summary} \\
	Sample & Obtained p-value & Samples are from populations with equal variances \\ \hline
	\textit{Total time, all} & & \\
	1 and 2 & 0.030 & \textbf{No}  \\ \hline
	\textit{Correct elements, all} & & \\
	3 and 4 &0.823& Yes   \\ \hline
	\textit{Total time, divided by task}& & \\
	5, 6, and 7 & 0.636 & Yes \\ \hline
	\textit{Correct elements, divided by task} & & \\
	8, 9, and 10 & 0.805  & Yes  \\ \hline
	\textit{Total time, experienced participants divided by task} & & \\
	11, 12, and 13 & 0.972 & Yes  \\ \hline
	\textit{Correct elements, experienced participants divided by task} & & \\
	14, 15, and 16 & 0.724 & Yes \\ \hline
	\textit{Total time, inexperienced participants divided by task} & & \\
	17, 18, and 19 & 0.499  & Yes  \\ \hline
	\textit{Correct elements, inexperienced participants divided by task} & & \\
	20, 21, and 22 & 0.626 & Yes \\ \hline
\end{longtable}

\vspace{0.5cm}

\subsection{Hypothesis testing}\label{sec:hypothesis_results}
In this section all the hypothesis will be tested. The tests order will be the same as the tables in section \ref{sec:gathereddata}. Which test that is used is determined by the results from the normality tests (section \ref{sec:normality_results} and Levene's test (section \ref{sec:levene_test_results}.

\subsubsection[Sample 1, 2]{Test differences in total time between experienced and inexperienced participants}\label{sec:t-test_result} 
%Two sample t-test results
%\textbf{Test differences between experienced and inexperienced participants}\\
This section will test if there are any difference in total time spent on the tasks between experienced and inexperienced participants. This test is covered by sample 1 and sample 2 from section \ref{sec:totaltime_ex_inex}. Sample 1 is experienced participants and sample 2 inexperienced participants. Both samples was normally distributed after a Box-Cox transformation (\ref{sec:sample1,2_normresult}). A two sample t-test will be used to answer this hypothesis since the normality assumption is valid. The hypothesis tested in this section is: \\[0.3cm]

\centerline{$H_{0}$: Equal task time between experienced and inexperienced participants} 
\centerline{$H_{A}$: Unequal task time between experienced and inexperienced participants}

\vspace{0.3cm}

If $\overline{x}_1$ equals the mean time for experienced-, and $\overline{x}_2$ the mean time for inexperienced participants, the hypothesis can be written as:\\[0.3cm]

\centerline{$H_{0}$: $\overline{x}_1$ = $\overline{x}_2$} 
\centerline{$H_{A}$: $\overline{x}_1$ $\neq$ $\overline{x}_2$}

\vspace{0.3cm}

Since we cannot assume equal variances in the two samples (\ref{sec:sample1,2_levene}), this test will use the Welch's t-test for unequal variances [\citep{Walpole2012}, p. 345]. Equation \ref{eq:ttest_twoway} is still valid. Results are shown under. The T-statistic calculated is smaller than the critical value. The t-test conclude that there is a significant difference between the means of the two population samples with a confidence interval of 95\%.\\[0.2cm]

 \begin{center}
	\begin{tcolorbox}[width=0.8\textwidth]
		\centering
		\textit{Two sample, two-way t-test, sample 1 and 2}\\
		Significance level: 5\%  \\[0.5cm]
		
		$T-statistic$: -60.442 \\
		Degree of freedom ($v$): 447 \\ %http://web.utk.edu/~cwiek/TwoSampleDoF
		Significance level ($\alpha$): 0.05 \\
		Critical value: 1.960\\[0.2cm]
		
		Using equation \ref{eq:ttest_twoway}, the absolute value of the $T-statistic$ is larger than the critical value ($|60.442|$ > $1.960$) and the null hypothesis is \underline{rejected} and $H_A$ accepted.\\[0.5cm]

	\end{tcolorbox} 
\end{center}

\vspace{0.5cm}

\textbf{Test if experienced or inexperienced participants finish the task fastest} 

Because there was a significant difference between time spent on the tasks, this section will also test who finished the task fastest. The second hypothesis tested in this section is:\newline

%\centerline{$H_{0}$: Inexperienced participants has a lower or equal total time compared to experienced participants.}
\centerline{$H_{0}$: Equal task time between all participants}
\centerline{$H_{A}$: Experience participants finish the task faster}

With sample 1 equals experienced participants and sample 2 equals inexperienced participants we get the hypothesis:\\[0.2cm]

\centerline{$H_{0}$: $\overline{x}_1$ = $\overline{x}_2$}
\centerline{$H_{A}$: $\overline{x}_1$ < $\overline{x}_2$}

This test gives the same T-statistics as the previous test, but the critical value is changed since this test is changed to a two sample, one-way t-test. Since we cannot assume equal variances in the two samples a Welch's test is performed. The test results are shown under. The $T-statistic$ is still smaller than the critical value (-64.654 < 1.645). Our test is to check if sample 1's mean is significantly larger than sample 2's mean, then we use the test written in equation \ref{eq:ttest_greater}. Our $T-statistic$ is not larger than the critical value, therefore we need to accept the null hypothesis. There is no evidence that experienced participants use less time on the tasks than the inexperienced. \\[0.2cm]

 \begin{center}
	\begin{tcolorbox}[width=0.8\textwidth]
		\centering
		\textit{Two sample, one-way t-test, sample 1 and 2}\\
		Significance level: 5\%  \\[0.5cm]
		
		$T-statistic$: -60.442 \\
		Degree of freedom ($v$): 447 \\ %http://web.utk.edu/~cwiek/TwoSampleDoF
		Significance level ($\alpha$): 0.05 \\
		Critical value: 1.645\\[0.2cm]
		
		T-statistic is smaller than the critical value ($-60.442$ < $1.645$) and the null hypothesis is \underline{accepted}.\\[0.5cm]
	\end{tcolorbox} 
\end{center}

\begin{figure}[H]
	\centering
	\includegraphics[width=0.7\linewidth]{../../thesis-statisticmethods/statistic_analysis/figures/boxplot/mean_std_participants_time}
	\caption{Sample 1 and 2 - mean (green dot) and standard deviation (blue line)}
	\label{fig:meanstdparticipantstime}
\end{figure}

Since we know that there is a significant difference between the two sample means, the author concludes that the inexperienced participants finished the task faster than the experienced participants. This can also be seen in plot \ref{fig:meanstdparticipantstime}. Inexperienced participants finished the tasks 16 seconds faster than experienced participants. \newline

\subsubsection[Sample 3, 4]{Test if there is a difference between experienced and inexperienced participants in total correct elements} %Mann-Whitey U Test results
%\textbf{Test differences between experienced and inexperienced participants}\\

This section will test if there is a difference between experienced- and inexperienced participants when looking at the number of correctly chosen elements. Sample 3 and 4 is the correct samples to use in this test. Both samples are not normally distributed, we need to use a non-parametric method. The Mann-Whitey U test is the preferred test to use on these samples. As mentioned in section \ref{sec:Wilcoxon}, the Mann-Whitey U test is preferred when the samples have ties (identical observations) in the data. From histogram \ref{fig:correctelementswasnotinterupter_ex} and \ref{fig:correctelementswasnotinterupted_inex} we see that the samples for both independent variables are similar. Mann-Whitey U test can therefore be used to compare the population medians. The hypothesis to be tested is:\\[0.3cm]

\centerline{$H_{0}$: $median_3$ = $median_4$}
\centerline{$H_{A}$: $median_3$ $\neq$ $median_4$}

The results of the test, the statistical value and finding the critical value in the Mann-Whitey U table is shown in the box under. Using equation \ref{eq:mannwhitey-ciritcalvalue} we conclude that there are not enough evidence to reject the null hypothesis with a confidence interval of 95\%. The $U-statistic$ is larger than the critical value. 

 \begin{center}
	\begin{tcolorbox}[width=0.8\textwidth]
		\centering
		\textit{Two sample t-test, sample 3 and 4}\\
		Significance level: 5\%  \\[0.5cm]
		
		$U-statistic$: 17012 \\
		Significance level ($\alpha$): 0.05 \\
		Sample size, n1:  229\\
		Sample size, n2: 200\\
		Critical-value: 127 \\[0.2cm] %http://web.utk.edu/~cwiek/TwoSampleDoF
		
		$U-statistic$ is larger than the critical value ($17012$ > $127$) and the null hypothesis is \underline{accepted}.\\[0.5cm]
	\end{tcolorbox} 
\end{center}

\begin{figure}[H]
	\centering
	\includegraphics[width=0.7\linewidth]{../../thesis-statisticmethods/statistic_analysis/figures/boxplot/mean_std_participants_correct}
	\caption{Sample 3 and 4 - mean (green dot) and standard deviation (blue line)}
	\label{fig:meanstdparticipantscorrect}
\end{figure}

Results from til section show that there is is not enough evidence to conclude that there is any difference between experienced- and inexperienced participants when looking at the number of correctly chosen elements per task. The author concludes that experienced- and inexperienced participants did equally well on the task. 

\subsubsection[Sample 5, 6 and 7]{Test if total time differs between the three tasks}\label{sec:anova_result} %One-way \textit{ANOVA} test results

This section will test if there is a difference in total time between the three tasks in this survey. Sample 5, 6, and 7 is the samples to use in this test. The one-way \textit{ANOVA} test will be used in this section. This test is used when there are more than two samples being compared (section \ref{sec:anova}). The assumption that sample 5, 6 and 7 come from populations with equal variances are met \ref{sec:sample5,6,7}), the samples are also normally distributed after a Box-Cox transformation (\ref{sec:task123_time_normaltest}). The hypothesis tested here is:\\

\centerline{$H_{0}$: $\overline{x}_5$ = $\overline{x}_6$ = $\overline{x}_7$}
\centerline{$H_{A}$: Total time is different between at least two of the tasks}

Using equation \ref{eq:anova_reject} and results shown in the box under, the \textit{ANOVA} test rejects the null hypothesis. The $f-value$ is lower than the critical value. With a confidence interval of 95\% the author claim that there is a difference between the mean value of the three tasks.

 \begin{center}
	\begin{tcolorbox}[width=0.8\textwidth]
		\centering
		\textit{One-way \textit{ANOVA}, sample 5, 6 and 7}\\
		Significance level: 5\%  \\[0.5cm]
		
		%$P-value$: $2.805 * 10^{-222}$ \\
		$f-value$: $2123.308$ \\
		Significance level ($\alpha$): 0.05 \\
		$v_1$ = 2, $v_2$ = 426
		Critical-value: 3.00 \\[0.2cm] %http://web.utk.edu/~cwiek/TwoSampleDoF
		
		$f-value$ is significantly lower than the critical value ($2123.308$ > $3.00$) and the null hypothesis is \underline{rejected}, $H_A$ is accepted\\[0.5cm]
	\end{tcolorbox} 
\end{center}

\begin{figure}[h!]
	\centering
	\includegraphics[width=0.7\linewidth]{../../thesis-statisticmethods/statistic_analysis/figures/boxplot/mean_std_task123png_time}
	\caption{Sample 5, 6 and 7 - mean (green dot) and standard deviation (blue line)}
	\label{fig:meanstdtask123pngtime}
\end{figure}

Since the null hypothesis was rejected \textit{Tukey's method} will be used to make comparisons between task one, two and three. This test did not find any significant difference between the three tasks. Visual evaluation of figure \ref{fig:meanstdtask123pngtime} show that task 1 was competed slightly faster than the two other tasks. 

 \begin{center}
	\begin{tcolorbox}[width=0.8\textwidth]
		\centering
		\textit{Tukey's test, sample 5, 6 and 7}\\
		Significance level: 5\%  \\[0.5cm]
		Task 1 - Sample 5, Task 2 - Sample 6 and Task 3 - Sample 7 \\[0.5cm]
		
		Task 1 and Task 2 do not differ significantly \\
		Task 1 and Task 3 do not differ significantly \\
		Task 2 and Task 3 do not differ significantly \\[0.2cm]
	\end{tcolorbox} 
\end{center}

\vspace{0.3cm}
Based on the results from this section the author concludes that there is a difference in total time between at least two of the tasks, but the difference is not enough. \textit{Tukey's test} can't find a significant difference. 

\subsubsection[Sample 8, 9, and 10]{Test if the number of correct elements differs between the three tasks}\label{sec:sample_8_9_10_kruskal)} %Kruskal-Wallis test results

This section will test if there is a difference in the number of correctly chosen elements between the three survey tests. The test will use sample 8, 9 and 10. Since sample 8 are not normally distributed (\ref{sec:task123_correct_normaltest}) a non-parametric test should be used. The Kruskal-Wallis test is the non-parametric eqvivalent to one-way \textit{ANOVA} (\ref{sec:kruskal-w-test}). It is used to test equality of medians when the samples are not normally distributed. This method will test if there is any difference between task 1, 2 and 3 when looking at the dependent variable number of correctly chosen elements. The hypothesis tested is: \\[0.2cm]

\centerline{$H_{0}$: $median_8$ = $median_9$ = $median_10$}
\centerline{$H_{A}$: Number of correctly chosen elements is different between at least two of the tasks}

Using equation \ref{eq:kruskapw-accept}, the Kruskal-Wallis test rejects the null hypothesis. The obtained $H-value$ is smaller than the critical value found. P-value is approximately zero and this gives a good indication that the result is significant. With a confidence interval of 95\% the author claim that there is a difference between the median value of the three tasks. \\[0.2cm]

 \begin{center}
	\begin{tcolorbox}[width=0.8\textwidth]
		\centering
		\textit{Kruskal-Wallis test, sample 8, 9 and 10}\\
		Significance level: 5\%  \\[0.5cm]
		
		$P-value$: $3.967* 10^{-72}$ \\
		$H-value$: $328.816$ \\
		Significance level ($\alpha$): 0.05 \\
		$v$ = 2\\ %k-1 = 3-1 = 2
		Critical-value:  5.991\\[0.2cm] %http://web.utk.edu/~cwiek/TwoSampleDoF
		
		$H-value$ is significantly lower than the critical value ($328.816$ > $5.991$) and the null hypothesis is \underline{rejected}, $H_A$ is accepted\\[0.5cm]
	\end{tcolorbox} 
\end{center}

\begin{figure}[h!]
	\centering
	\includegraphics[width=0.7\linewidth]{../../thesis-statisticmethods/statistic_analysis/figures/boxplot/mean_std_task123png_correct}
	\caption{Sample 8, 9 and 10 - mean (green dot) and standard deviation (blue line)}
	\label{fig:meanstdtask123pngcorrect}
\end{figure}

Like in one-way \textit{ANOVA}, a \textit{post hoc} test should be used to make paired comparisons to determine which groups differ. The \textit{post hoc} test used in this section is Tukey's test. Results from Tukey's test resulted in a significant difference in the number of correctly chosen elements between task 1 and task 2, and task 1 and task 3. Figure \ref{fig:meanstdtask123pngcorrect} show that task 1 has a higher mean value than the other two tasks. Task 1 also has a smaller standard deviation than the other tasks. \\[0.2cm]

 \begin{center}
	\begin{tcolorbox}[width=0.8\textwidth]
		\centering
		\textit{Tukey's test, sample 8, 9 and 10}\\
		Significance level: 5\%  \\[0.5cm]
		Task 1 - Sample 8, Task 2 - Sample 9 and Task 3 - Sample 10 \\[0.5cm]
		
		Task 1 and Task 2 differs significantly \\
		Task 1 and Task 3 differs significantly \\
		Task 2 and Task 3 do not differ significantly \\[0.2cm]
	\end{tcolorbox} 
\end{center}

\vspace{0.3cm}

The results in this section find significant evidence that the participants had a higher number of correct elements in task 1 than in the two other tasks. 

\subsubsection[Sample 11 - 16]{Test differenced in results from experienced participants between task 1, 2 and 3}\label{sec:sample_11_12_13_anova}
This section will answer two hypothesis about experienced participant's results divided in task 1, 2 and 3. Sample 11, 12 and 13 will be used to answer the first hypothesis. These three samples are normally distributed after a Box-Cox transformation (\ref{tab:normaltest_summary}) and also come from populations with equal variances (\ref{sec:sample11,12,13_levene}). Sample 14, 15 and 16 will be used on the second hypothesis. All three samples are normally distributed (\ref{tab:normaltest_summary}) and also come from populations with equal variances (\ref{sec:sample14,15,16_levene}). The one-way \textit{ANOVA} method will be used to test both hypotheses. The first hypothesis will test if time spent on each of the three tasks differ and the second hypothesis will test if the number of correct elements in each of the three tasks differs when the samples only include results from experienced participants. 

The first hypothesis is:\\
\centerline{$H_{0}$: $\overline{x}_{11}$ = $\overline{x}_{12}$ = $\overline{x}_{13}$}
\centerline{$H_{A}$: Total time is different between at least two of the tasks}

Using equation \ref{eq:anova_reject}, the one-way \textit{ANOVA} test rejects the null hypothesis. The obtained $f-value$ ($1216.919$) is higher than the critical value ($3.00$). The calculated p-value is also approximately zero and this gives a good indication that the result is significant. With a confidence interval of 95\% the author claim that there is a time difference between the three tasks. 

When the null hypothesis is rejected a \textit{post hoc} test is used to compare each task with each other. Tukey's \textit{post hos} test did not find any significant difference between the three tasks with a significant level of 5\%. Figure \ref{fig:meanstdexperiencedtask123time} show an approximately similar mean value in all three tasks. %task 2 has a lower mean and less standard deviation. 

\begin{figure}[H]
	\centering
	\includegraphics[width=0.7\linewidth]{../../thesis-statisticmethods/statistic_analysis/figures/boxplot/mean_std_experienced_task123_time}
	\caption{Sample 11, 12 and 13 - mean (green dot) and standard deviation (blue line)}
	\label{fig:meanstdexperiencedtask123time}
\end{figure}

The author concludes that there is a significant difference between at least two of the tasks when looking at experienced participants mean task time, but the difference is not enough so that \textit{Tukey's test} can find which tasks that differ. \newline

The second hypothesis is:\\
\centerline{$H_{0}$: $\overline{x}_11$ = $\overline{x}_12$ = $\overline{x}_13$}
\centerline{$H_{A}$: Correct elements in each task is different between at least two of the tasks}

Using equation \ref{eq:anova_reject}, the one-way \textit{ANOVA} test rejects the null hypothesis. The obtained $f-value$ ($8.210$) is higher than the critical value ($3.00$). The p-value is also approximately zero and this gives a good indication that the result is significant. With a confidence interval of 95\% the author claim that there is a difference between the mean value of at least two of the tasks. 

Since the null hypothesis was rejected, a \textit{post hoc} test should be used to make paired comparisons to determine which groups differ. Tukey's \textit{post-hoc} test resulted in a significant difference in the number of correctly chosen elements between task 1 and task 2, and task 1 and task 3. Figure \ref{fig:meanstdexperiencedtask123correct} show that task 1 has a higher mean value than the two other tasks. Task 1 also has a smaller standard deviation. 

\begin{figure}[H]
	\centering
	\includegraphics[width=0.7\linewidth]{../../thesis-statisticmethods/statistic_analysis/figures/boxplot/mean_std_experienced_task123_correct}
	\caption{Sample 14, 15 and 16 - mean (green dot) and standard deviation (blue line)}
	\label{fig:meanstdexperiencedtask123correct}
\end{figure}

\vspace{0.3cm}

With the results found in this section, the author concludes that there is a significant difference in the number of correctly chosen elements between the three tasks. The experienced participants got the best result on task 1. Time spent on each task also differ between at least two of the tasks, but the difference is not significant enough so that Tukey's test can determine a difference. Figure \ref{fig:meanstdexperiencedtask123time} show that task 2 has the lowest mean time value of the three tasks.  


\subsubsection[Sample 17 -  22]{Test differenced in results from inexperienced participants between task 1, 2 and 33}\label{sec:sample_17_18_19_anova}

This section will answer the same hypothesis at the previous section, only with results from inexperienced participants. Sample 17, 18 and 19 will be used on the first hypothesis. These samples are normally distributed (\ref{sec:sample_17,18,19_normalitytest}) and come from populations with equal variances (\ref{sec:sample17,18,19_levene}). Sample 20, 21 and 22 will be used on the second hypothesis. All three samples are normally distributed (\ref{sec:sample_20,21,22_normalitytest}) and also come from populations with equal variances (\ref{sec:sample20,21,22_levene}). The one-way \textit{ANOVA} will be used to test both hypotheses. The first hypothesis will test if time spent on each task differ and the second hypothesis will test if the number of correct elements in each task differs.

The first hypothesis is:\\
\centerline{$H_{0}$: $\overline{x}_{17}$ = $\overline{x}_{18}$ = $\overline{x}_{19}$}
\centerline{$H_{A}$: Total time is different between at least two of the tasks}

The one-way \textit{ANOVA} test rejects the null hypothesis and accepts the alternative hypothesis ($H_{A}$) with a significant level of 5\%. The obtained f-value from the test is higher than the critical value ($905.34$ > $3.00$). Since the alternative hypothesis was accepted, Tukey's \textit{post hoc} test is used to make compared comparisons between task 1, task 2 and task 3. The test doesn't find a significant difference when comparing each of the three tasks with a significant level of 5\%. Figure \ref{fig:meanstdinexperiencedtask123time} show that inexperienced participants spent more time on task 2 than the other tasks, but the difference is not significant according to Tukey's test.

\begin{figure}[H]
	\centering
	\includegraphics[width=0.6\linewidth]{../../thesis-statisticmethods/statistic_analysis/figures/boxplot/mean_std_inexperienced_task123_time}
	\caption{Mean (green dot) and standard deviation (blue line) for sample 17, 18 and 19}
	\label{fig:meanstdinexperiencedtask123time}
\end{figure}

The seconds hypothesis is:\\
\centerline{$H_{0}$: $\overline{x}_{20}$ = $\overline{x}_{21}$ = $\overline{x}_{22}$}
\centerline{$H_{A}$: Number of correct elements is different between at least two of the tasks}

The one-way \textit{ANOVA} test rejects the null hypothesis ($H_0$) and accepts the alternative hypothesis ($H_{A}$) with a significance level of 5\%. The obtained f-value from the test is higher than the critical value ($189.05$ > $3.00$). Since the null hypothesis was rejected Tukey's \textit{post hoc} test will be used to make comparisons between the three tasks. This test cannot find a significant difference when comparing the tasks with a significant level of 5\%. Looking at figure \ref{fig:meanstdinexperiencedtask123correct}, task 1 has a higher mean time value than the two other tasks, also task 2 has a higher mean than task 3. Even though there are differences in the number of correct elements between the tasks, it is not significant according to Tukey's test. 

\begin{figure}[H]
	\centering
	\includegraphics[width=0.6\linewidth]{../../thesis-statisticmethods/statistic_analysis/figures/boxplot/mean_std_inexperienced_task123_correct}
	\caption{Mean (green dot) and standard deviation (blue line) for sample 20, 21 and 22}
	\label{fig:meanstdinexperiencedtask123correct}
\end{figure}

With the results found in this section, the author concludes that there is a significant difference in both total time spent on each task and the number of correctly chosen elements between at least two of the tasks. The differences are not significant enough so that Tukey's test can determine which task differs. Figure \ref{fig:meanstdinexperiencedtask123time} and \ref{fig:meanstdinexperiencedtask123correct} show that task 1 has the lowest mean time value and the highest mean correct value. 

\subsubsection{Hypothesis test summary}

	\begin{longtable}{p{0.65\textwidth}|p{0.15\textwidth}|p{0.15\textwidth}}  %\multicolumn{1}{c}{}
	\caption[Summary, normality tests]{Summary of hypothesis tests done in section \ref{sec:hypothesis_results}} \label{tab:hypothesistest_summary} \\
		Hypothesis (\textcolor{cyan}{Dependent variable}, \textcolor{blue}{Independent variable}) & \textit{Participants} Sample number& Hypothesis is accepted \\[0.2cm] \hline
		\textcolor{cyan}{Total time}, \textcolor{blue}{Experienced} and \textcolor{blue}{Inexperienced} & \textit{All} &  \\
		There are a difference between experienced and inexperienced participants & 1 and 2 & \textbf{Yes} \\
		Experienced participants finish the tasks faster than inexperienced  & 1 and 2 & No   \\ 
		Inexperienced participants finish the tasks faster than experienced  & 1 and 2 & \textbf{Yes}   \\ \hline
		\textcolor{cyan}{Correct elements}, \textcolor{blue}{Experienced} and \textcolor{blue}{Inexperienced} & \textit{All} &  \\
		There are a difference between experienced and inexperienced participants & 3 and 4 & No   \\ \hline
		\textcolor{cyan}{Total time}, \textcolor{blue}{Task 1}, \textcolor{blue}{Task 2} and \textcolor{blue}{Task 3}& \textit{All} &  \\
		 Total time is different between at least two of the tasks & 5, 6 and 7 & \textbf{Yes}   \\
		 Task 1 significantly differs from Task 2 & 5, 6 and 7 & No  \\ 
		 Task 1 significantly differs from Task 3 & 5, 6 and 7 & No  \\ 
		 Task 2 significantly differs from Task 3 & 5, 6 and 7 & No  \\ \hline
		\textcolor{cyan}{Correct elements}, \textcolor{blue}{Task 1}, \textcolor{blue}{Task 2}, \textcolor{blue}{Task 3} & \textit{All} &  \\
		The number of correctly chosen elements is different between at least two of the tasks & 8, 9 and 10 & \textbf{Yes}  \\
		Task 1 significantly differs from Task 2 & 8, 9 and 10 & \textbf{Yes}  \\ 
		Task 1 significantly differs from Task 3 & 8, 9 and 10 & \textbf{Yes}  \\ 
		Task 2 significantly differs from Task 3 & 8, 9 and 10 & No  \\ \hline
		\textcolor{cyan}{Total time}, \textcolor{blue}{Task 1}, \textcolor{blue}{Task 2}, \textcolor{blue}{Task 3} & \textit{Experienced}  &  \\
		Total time is different between at least two of the tasks & 11, 12 and 13 & \textbf{Yes}  \\
		Task 1 significantly differs from Task 2 & 11, 12 and 13  & No  \\ 
		Task 1 significantly differs from Task 3 & 11, 12 and 13  & No  \\ 
		Task 2 significantly differs from Task 3 & 11, 12 and 13 & No  \\ \hline
		\textcolor{cyan}{Correct elements}, \textcolor{blue}{Task 1}, \textcolor{blue}{Task 2}, \textcolor{blue}{Task 3} & \textit{Experienced}  &  \\
		Number of correct elements differs between at least two of the tasks & 14, 15 and 16 & \textbf{Yes}  \\
		Task 1 significantly differs from Task 2 & 14, 15 and 16 & \textbf{Yes}  \\ 
		Task 1 significantly differs from Task 3 & 14, 15 and 16 & \textbf{Yes} \\ 
		Task 2 significantly differs from Task 3 & 14, 15 and 16 & No  \\ \hline
		\textcolor{cyan}{Total time}, \textcolor{blue}{Task 1}, \textcolor{blue}{Task 2}, \textcolor{blue}{Task 3} & \textit{Inexperienced}  &  \\
		Total time is different between at least two of the tasks & 17, 18 and 19 & \textbf{Yes}  \\
		Task 1 significantly differs from Task 2 & 17, 18 and 19 & No  \\ 
		Task 1 significantly differs from Task 3 & 17, 18 and 19 & No  \\ 
		Task 2 significantly differs from Task 3 & 17, 18 and 19 & No  \\ \hline
		\textcolor{cyan}{Correct elements}, \textcolor{blue}{Task 1}, \textcolor{blue}{Task 2}, \textcolor{blue}{Task 3} & \textit{Inexperienced}  &  \\
		Number of correct elements differs between at least two of the tasks & 20, 21 and 22 & \textbf{Yes}  \\
		Task 1 significantly differs from Task 2 & 20, 21 and 22 & No  \\ 
		Task 1 significantly differs from Task 3 & 20, 21 and 22 & No \\ 
		Task 2 significantly differs from Task 3 & 20, 21 and 22 & No  \\ \hline
	\end{longtable}