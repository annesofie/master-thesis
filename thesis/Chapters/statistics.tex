\chapter{Statistics}

%My hypothesis before start

\section{Statistics theory}

\subsection{Hypothesis testing}
The null- and alternative hypothesis are statements regarding a difference or an that occur in the population of the study. The alternative hypothesis (Ha) usually represents the question to be answered or the theory to be tested, while the null hypothesis ($H_{0}$) nullifies or opposes Ha \citep{Walpole2012}. The sample collected in the study is used to test which statement is most likely (technically it's testing the evidence against the null hypothesis).  When the hypothesis is identified, both null and alternative, the next step is to find evidence and develop a strategy for or against the null hypothesis \citep{LundResearchLtd2013}.

Next step is to determine the level of statistical significance, often expressed as the \textit{p-value}. A statistical test will result in the probability (\textit{the p-value}) of observing your sample results given that the null hypothesis is true. A significant level widely used in academic research is 0.05 or 0.01 \citep{Walpole2012}. 

%Compare the mean time in experienced and not experienced
%H0: No time difference, Ha: Experienced use less time

\subsection{Normal testing}\label{subsec:normaltesting}
A \textit{goodness-of-fit} test is used to determine whether a sample of \textit{n} observations can be considered as a sample from a given specified distribution \citep{Walpole2012}. The Anderson-Darling and the Kolmogorov-Smirow tests stand out as \textit{goodness-of-fit} procedures specialized for small samples \citep{Romeu2003}. The Anderson-Darling test will be used in this study to test if the observations gathered in this study is normally distributed.  %The Kolmogorov-Smirow test is a nonparametric test
The hypothesis for the Anderson-Darling test is: \newline

\centerline{$H_{0}$: The data follows the normal distribution} 
\centerline{$H_{A}$: The data do not follow the normal distribution}

The computations in the Anderson-Darling test differs based on that is known about the distribution. In this study both the expected mean and variance is unknown. In all the Anderson-Darling tests in this study a significance level of $0.05$ is used, which gives a confidence interval of $95$ \%. If the calculated \textit{p-value} is less than the significance level ($0.05$), the null hypothesis is rejected. The larger the \textit{p-value} the closer match is the data to the normal distribution. The Anderson-Darling statistics is used to calculate the \textit{p+value} for the \textit{goodness-of-fit} test. 

\subsection{Tests on a Single Mean}
When the variance is unknown we use the Student t-distribution. The random variables $X_{1}, X_{2}, X_{3}, ..., X_{n}$ represents the random sample from a normal distribution with unknown $\mu$ and $\sigma{}^{2}$. 

%Example: In this example where p = .03, we would reject the null hypothesis and accept the alternative hypothesis. We reject it because at a significance level of 0.03 (i.e., less than a 5% chance), the result we obtained could happen too frequently for us to be confident that it was the two teaching methods that had an effect on exam performance. https://statistics.laerd.com/statistical-guides/hypothesis-testing-3.php

\subsection{Two sample test}
When estimating the difference between two means a two-sample t-test is used \citep{Walpole2012}. A two sampled test assumes two independent, random samples from distributions with means $\mu_{1}$ and $\mu_{2}$ and variances $\sigma_{1}^{2}$ and $\sigma_{2}^{2}$. %*HOW TO DETERMINE IF THEY ARE INDEPENDENT? 
The one-sided hypothesis on two means can be written as:\newline

\centerline{$H_{0}$: $\mu_{1}$ - $\mu_{2}$ = 0 or $\mu_{1}$ = $\mu_{2}$} 
\centerline{$H_{A}$: $\mu_{1}$ - $\mu_{2}$ > 0 or $\mu_{1}$ > $\mu_{1}$}
 
Before doing tests on the two means Levene's Test is used to test if the samples are from populations with equal variances. If they are equal a two-sampled t-test may be used. %https://docs.scipy.org/doc/scipy-0.14.0/reference/generated/scipy.stats.levene.html


