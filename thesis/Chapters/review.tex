\chapter{Micro-tasking review}

A report by eYeja 

\section{Crowdsourcing}

%What is it
The first time the term "crowdsourcing" appeared was in 2006 (yEka). \cite{EYeka2015} state that 85 \% of the top global brands use crowdsourcing for various purposes. With establishment of micro-task crowdsourcing platforms as Amazon's Mechanical Turk (MTurk; www.mturk.com) and CrowdFlower (www.crowdflower.com) micro-tasking is much more accessible. Micro-tasking practitioners are actively turning towards paid crowdsourcing to solve data centric tasks that require human input \citep{Gadiraju2015}. 

Crowdsourcing has become a widespread approach to dealing with machine-based computations where we leverage the human intelligence \citep{Gadiraju2015}.

\cite{Gadiraju2015} categorize typically crowdsourced tasks into six top level classes. Interesting classes within geospatial data is \textit{Verification and validation}, \textit{Interpretation and analysis} and \textit{Content creation}. These are examples of all three task classes in geospatial crowdsourcing. During imports of large datasets into OpenStreetMap micro-tasking are used to validate the new data. In humanitarian OpenSteetMap they use micro-tasking to create geospatial data in areas during crisis to support the help organizations. In machine learning process teams are starting to use micro-tasks to both validate the created data and also create test datasets to the algorithms. Interpretation and analysis tasks rely on the individual to use their interpretation skills during task completion. The task can be to choose between two layers, and decide which is best. This is the task-class used in this thesis during the survey. More in section \ref{sec:survey}.

\section{Micro-tasking}

%What is it
The task refers to the activity, production or service the company or organization wants to have done. \cite{Gadiraju2015} findings when analyzing data from MTurk, indicate rapid growth in micro-task crowdsourcing. 

Micro-task crowdsourcing refers to a problem-solving model in which a problem or task is outsourced to a distributed group of people by splitting the task or problem into smaller sub-tasks or sub-problems. The sub-tasks or sub-problems are then solved by multiple workers independently, often in return for a reward \citep{Sarasua2012}. 

Problems that are suitable for solving through micro-tasking are those that are easy to distribute into a number of simple tasks, that can be completed in parallel in a relatively short period of time (from seconds to minutes), without requiring specific skills \citep{Sarasua2012}. Research has also demonstrated that micro-tasking is effective for far more complex problems when using sophisticated workflow management techniques. MIcro-tasking can then be applied to an broader range of problems like: completing surveys, translating text between two languages, matching pictures of people, summarising text \citep{Bernstein2015a}, etc.  

\subsection{Micro-tasking platforms}

\subsubsection{MTurk}

\subsubsection{Soylent}
Is a word processing interface that enables writers to call on Mechanical Turk workers to shorten, proofread, and edit parts of their document on demand. To improve the quality of the work, the Soylen team introduced the Find-Fix-Verify crowd programming pattern. This architecture splits tasks into a series of generating and review stages \citep{Bernstein2015a}. 

\subsubsection{Tasking manager}

\subsubsection{Crowdflower}

\subsubsection{CrowdMap}
Is an approach to integrate human and computational intelligence in ontology \footnote{\label{ontology} Ontology is a formal naming and definition of types, properties, and interrelationships of the entities that really or fundamentally exists for a particular domain of discourse. Ontologies are created to limit complexity and to organize information. The ontology can then be applied to problem solving. } alignment tasks via microtask crowdsourcing \citep{Sarasua2012}. Ontology is still (2012) one of those problems that we cannot automate completely and having a human in the loop might increase the quality of the results of machine-driven approaches. 

\subsection{Usage}
 A machine learning company called "developmentSEED" use a micro-tasking solution for cleaning machine learning output data. They have created a GUI web-application solution called Skynet Scrubber. In their blog, Derek Lieu writes: "Skynet gets more capable every day, but the output is still not perfect [..] We built Skynet Scrub so we could start using Skynet data sooner". 
 

 
 \subsection{Potential}

\subsection{Challenges}
When aiming towards wider adoption of crowdsourcing one have to be aware of the challenges of using it. It is important to remember that all tasks do not fit into the micro-tasking crowdworker model. Very complex tasks that can't be partitioned are not suitable for solving through micro-tasks. 
