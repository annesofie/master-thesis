\chapter{Methology and experiment}

\section{Web-application}
This thesis used an online web-based survey to conduct the experiment. An online survey avoid the cost and effort of printing, distributing, and collecting paper forms. Many people prefer to answer a brief survey displayed on a screen instead of filling in and returning a printed form \citep{Ben2009}.   

In a self selected sample, which is some the case here, there is potentially a bias in the sample \citep{Ben2009}. %s168

\section{Pilot test}
It is important to pilot test the survey prior to actual use \citep{Ben2009}. A pilot test provides an opportunity to validate the wording of the tasks, do the paricipants understand the tasks? It also helps understand the time necessary for completing the survey, which should be communicated to the participants in prior to the survey \citep{Schade2015}. The pilot-test is conducted with a small sample of users. Results from the pilot-test can be used to determine the sample size. The sample size tells us how many responses that are needed \citep{Smith2013}. The formula for determining the sample size requires the standard deviation, how much variance to expect in the response \citep{Smith2013}. This standard deviation can be calculated from the pilot-test results. 

A pilot test was conducted with a total of eight participants, five experienced and three non-experienced participants aged from 22 to 64 years.  After the pilot test the usability was measured by using the \textit{System Usability Scale}(SUS). This scale gives an subjective measure of usability, which is usually obtained through the use of questionaires and attitude scales \citep{Brooke1996}. SUS was developed by John Brooke and consists of ten statements where the participants rate their agreement in an five-point scale \citep{Ben2009}. The usability is important to measure. If the participants doesnt understand how the web-application works, they will probably not do the survey since they then have to invest time in understanding what to do. %they will either exit the survey or answer the questions in the survey wrong. 
Usability is an important factor to get enough participants to do the whole survey and not quit halfway. 

\subsection{Execution of the pilot test}
The pilot test started with a brief information about this study and the survey. They where told to talk out load during the survey, no help or guidance was given to the participants. The author observed the participants while they conducted the survey. The author took notes and watched if the participants understood the questions in the survey correctly. After the survey a \textit{System Usability Scale} questionnaire was answered by the participants. At the end the participants was asked for general feedback on the web-application. The SUS score and the feedback was then used to determine the usability and improvements to the web-application. 

\subsection{Results from the pilot test}
The two oldest participants spent almost twice as much time on the test than the younger. Maybe it where too much cognitive load on them. Learning a new application and at the same time understanding how to do the survey and answer the questions given to them. One of them where experienced and the other non-experienced, so this is a surorising result.  CHECK THE TIME ON EACH TASK FOR THE OLDER PARTICIPANTS. 

The average SUS score was 84.64 out of 100. 
The average time spent on the survey was 18 minutes. The two oldest participants used on average 33 minutes, while the rest of the participants spent on average 13 minutes. 

