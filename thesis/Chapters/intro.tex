\chapter{Introduction}

 
 To the authors best knowledge, little research has been done on micro-tasking geospatial data. \cite{Salk2016} looked at how local knowledge and professional background, impact the volunteer performance in a Land-Cover classification task. 
 
 %Motivasjonen min for å gjøre dette her 
A task can traditionally be divided by time, place, person, object, and skill [\citep{Meier2013}, p. 13]. A task can be created by identifying the time it will require, the place where it must be done, the people who need to do the task, the object on which the work is done and finally, the skill needed for the task. Today we have technology that can create and move tasks around based on the four first categories.  Technology can allocate tasks based on the deadline and time the task requires. It can also establish communication between any team of people dependent on which people the task requires. One thing that technology can't do is change the skill of individual workers, though it can only connect people with different skills to work on the same tasks [\citep{Meier2013}, p. 14]. Crowdsourcing moves beyond this and looks at the skills of individual workers, the problem that needs to be solved and combines the best skills of workers to solve the problem. The division of labor by skill has more economic impact than the other four categories [\citep{Meier2013}, p. 15]. Crowdsourcing is a way of refactoring work in a way that exploits the worker's flexibility and gets the right skills to the right part of the problem. To get the right skills to the right part of the problem it needs to be partitioned into smaller parts. Having smaller parts will make it easier to distribute the problem. The distribution can be done through micro-tasking, also called "smart crowdsourcing" by Patrick Meier \citep{Meier2013a}. 

This thesis aim is to study if micro-tasking can successfully be expanded to involving maps and geospatial data. The OpenStreetMap community has used the method some time, and the usage so far can be evaluated as successful. %*Referer til prosjektoppgave 
This thesis also aims to find out if inexperienced individuals also manage to solve tasks on maps that involves geospatial data. The study also aims to determine if the number of elements in each micro-task has an impact on how well individuals solve the micro-tasks. The quality of the work, the number of correctly solved tasks and time, is measured. The thesis uses a survey hosted through a web-application to gather participant data. The data is then used to answer this thesis hypothesizes. The next chapter will give a thorough introduction to micro-tasking and hopefully make it clearer what this thesis aim is. Chapter 3 will explain the survey and chapter 4 will contain the statistics, both hypothesis, theory, and results. 



Albert Einstein illustrates this perfectly: “Computers are incredibly fast, accurate, but stupid. Humans are incredibly slow, inaccurate, but brilliant. Together they may be powerful beyond imagination” \citep{Holzinger2013}. 