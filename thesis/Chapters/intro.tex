\chapter{Introduction}
 %Motivasjonen min for å gjøre dette her 
Having access to information is not the same as learning and understanding it. Today, the amount of information available is enormous. Computers have the ability to learn through machine-learning, but machine-learning is dependent on well-developed training data to learn. The data has to be information placed in contextual meaning. Understanding and giving contextual meaning to information is the strengths of the human brain. Humans have the ability to create new ideas and concepts from unstructured information \citep{Ross2016}. Computers and the human brain are an unbeatable combination. Humans need computers for their speed and accuracy, and computers need the human brain to make sense of new information. 

Albert Einstein illustrates this perfectly: “Computers are incredibly fast, accurate, but stupid. Humans are incredibly slow, inaccurate, but brilliant. Together they may be powerful beyond imagination” \citep{Holzinger2013}. 

This thesis argues that a useful approach for combining the strengths of computers and humans together is micro-tasking. Micro-tasks are the smallest, simplest types of tasks and should demand little time to complete. The tasks can be carried out by humans through, or in collaboration with, computer systems \citep{Yang2016}. The author believes that micro-tasking has an unexplored potential within geospatial tasks, that is, tasks that involve data with a geographic position. %The tasks can be everything from creating geospatial data to validating or analyzing it. 
Importing geospatial data into a database, i.e., OpenStreetMap, covering huge areas can be considered a large task. This thesis will have an emphasis on the data validation and conflict handling part of the import of geospatial data.

Based on careful reading of relevant literature, the author conclude that little research has been done on micro-tasks involving geospatial data. To be able to exploit the micro-tasking method together with geospatial data, it is important to study how well humans solves these kinds of tasks. Micro-tasks are often published on a micro-tasking platform, where tasks and humans are connected. It is important to know if inexperienced individuals are capable of solving the tasks. If only experienced people can solve geospatial micro-tasks, a platform who can distinguish between people with geospatial knowledge needs to be used. The author does not have any knowledge if such a platform exists. 

In Remote-Sensing, humans perform land-cover classification tasks \citep{Salk2016}. At least two papers have studied whether there are significant differences in quality of the information contributed by experts and non-experts.
\cite{Salk2016} concluded that there was little first-order relationship between professional background and the task accuracy.
When comparing specialists with non-specialists, the non-specialists performed slightly better on images near home. 
\cite{See2013} concluded that there was little difference between experts and non-experts, and also found that the non-experts improved more than experts over time. Overall the non-experts were as reliable in what they identified as the experts. These two studies show that the person's background is not necessarily important. \cite{See2013} argued that with proper targeted training material the differences between experts and non-experts could potentially decrease.   
%\cite{Salk2016} looked at how local knowledge and professional background impact the volunteer performance in a land-cover classification tasks. The study used data from an online land-cover classification game where the participants determined if a satellite image or ground photograph contained cropland. The participants could answer 'yes', 'no', or 'maybe'.  
%\cite{See2013} had a similar study. It looked at whether there are significant differences in quality in the information contributed by experts and non-experts. The participants were given pixel outlines of 1 km resolution and were asked to determine the percentage of human impact and the land cover type. 
%This thesis aim is to study how well geospatial tasks is solved through micro-tasking. The quality of each completed task when using the micro-task method is important. In this thesis, the quality is measured through the number of correctly chosen elements in each task. The resulting data will distinguish between experienced and inexperienced participants. A micro-task should be small enough so that all individuals can complete the task, independent on their background and experience. 

The goal of this thesis is to study if micro-tasking can successfully be expanded to involving maps and geospatial data. The OpenStreetMap (OSM) community has used the method for some years, and the usage so far can be evaluated as successful \citep{Erichsen2016}. This thesis will look at the OSM community's usage of micro-tasking to examine if all individuals, independent of background, manage to solve geospatial tasks. The thesis also aims to determine if the number of elements in each micro-task has an impact on time spent per task and the quality of the task results. 

There is potentially much work creating micro-tasks. The large task needs to be appropriately broken down to micro-tasks that are easy, enjoyable, and fast to solve. This breakdown requires design skills and proper tutorials and examples for new workers \citep{Schulze2012}. 
%Humans make mistakes, and these errors need to be detected. Quality mechanisms also need to be determined, so that the output data is of high quality, without errors. 
Guidelines can be used to avoid putting too much emphasis on the preparations. This thesis can give a set of guidelines to be utilized on how to break down large geospatial tasks.\\

The aim of this thesis is to answer the research questions: 
\begin{enumerate}
	%\item Inexperienced workers cannot solve micro-tasks containing geospatial data as good as experienced workers
	%\item Can inexperienced individuals solve micro-tasks containing geospatial data as competently as experienced individuals?
	%\item How many elements should a micro-task containing geospatial data include?
	%\item The fewer elements in a micro-task, the better the worker solves the task
	\item Is it possible to give micro-tasks containing geospatial data to inexperienced workers?
	\item Will the quality of the solved task increase with fewer elements present in each micro-task?
	\item What is the number of elements optimal within a micro-task to get it completed as quickly as possible?
\end{enumerate}
\vspace{0.3cm}
An online web experiment was developed to answer the research questions. The experiment contains three tasks varying the number of elements given to the participant. Each task has the same two questions representing two micro-tasks. One question involves map interaction and the other an interpretation task containing metadata. 

The next chapter will give a thorough introduction to micro-tasking and clarify the aim of this thesis. Chapter three will introduce the experiment, the web application hosting the experiment and results from the experiments pilot test. Chapter four will present the individuals participating in the experiment, provide a description of the statistical theory used in the analysis, and give a summary of the gathered data. The last section in chapter four will contain the analyses of hypotheses which in sum will answer the research questions. Chapter five contains the discussion, where the results are summed and discussed. At last, the thesis will give a conclusion and outline further work in this area.
 
%The next chapter will give a thorough introduction to micro-tasking and hopefully make it clearer what this thesis aim is. Chapter 3 will explain the survey and chapter 4 will contain the statistics, both hypothesis, theory, and results. 
 
%A task can traditionally be divided by time, place, person, object, and skill [\citep{Meier2013}, p. 13]. A task can be created by identifying the time it will require, the place where it must be done, the people who need to do the task, the object on which the work is done and finally, the skill needed for the task. Today we have technology that can create and move tasks around based on the four first categories.  Technology can allocate tasks based on the deadline and time the task requires. It can also establish communication between any team of people dependent on which people the task requires. One thing that technology can't do is change the skill of individual workers, though it can only connect people with different skills to work on the same tasks [\citep{Meier2013}, p. 14]. Crowdsourcing moves beyond this and looks at the skills of individual workers, the problem that needs to be solved and combines the best skills of workers to solve the problem. The division of labor by skill has more economic impact than the other four categories [\citep{Meier2013}, p. 15]. Crowdsourcing is a way of refactoring work in a way that exploits the worker's flexibility and gets the right skills to the right part of the problem. To get the right skills to the right part of the problem it needs to be partitioned into smaller parts. Having smaller parts will make it easier to distribute the problem. The distribution can be done through micro-tasking, also called "smart crowdsourcing" by Patrick Meier \citep{Meier2013a}. 

