\chapter*{Abstract}
\addcontentsline{toc}{chapter}{Abstract}%

This paper proposes to extend the usage of the micro-tasking method to involve geospatial tasks. There is an unexplored potential in micro-tasking geospatial tasks. The research questions tested: 1) Is it possible to give micro-tasks containing geospatial data to inexperienced workers? 2) Will the quality of the solved task increase with fewer elements present in each micro-task? 3) What are the number of elements optimal within a micro-task to get it completed as quickly as possible?

An online web experiment was developed and implemented to gather data about how individuals solve geospatial micro-tasks. Statistical analysis is conducted on the collected data to answer the research questions. The experiment registered the participant's background to see if the quality of the solved micro-tasks differs between experienced and inexperienced participants. The tasks varied the number of elements that had to be handled at the same time to complete the micro-task. This approach was used to determine if the number has an influence on the quality of the solved micro-tasks. 
%Participants completed the same two micro-tasks in three different tasks. The tasks varied the number of elements that had to be handled to complete the micro-task. This approach was used to determine if the number has an influence on the quality of the solved micro-tasks. 

Statistical analysis found significant evidence of inexperienced participants finishing the micro-tasks faster than the experienced. The quality of the completed micro-tasks did not differ between experts and non-experts. When examining the three different tasks in the experiment, the task containing the fewest elements had statistically better quality than the two other. There was also a difference in time spent completing the three tasks, but not enough to be significant. The author concludes that geospatial micro-tasks can be given to inexperienced individuals and if the quality of the task results is important, fewer elements will increase the quality. 

%Experts and non-experts did equally well on the micro-tasks, and the non-experts finished the micro-tasks faster. 
%This paper propose a method for extracting buildings in satellite photos. 
%The proposed network makes use of a digital surface model and multispectral satellite data. It 
