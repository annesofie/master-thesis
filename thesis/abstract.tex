\chapter*{Abstract}
\addcontentsline{toc}{chapter}{Abstract}%

Micro-tasking is the process of dividing a large task into smaller parts, making the task easier to distribute and lowering its complexity. This thesis proposes to extend the usage of the micro-tasking method to involve geospatial tasks. To the authors best knowledge little research has been conducted on how to best exploit micro-tasking together with geospatial data. This thesis explores how to partition a large task into smaller parts when it involves geospatial data and also analyses if individuals with no experience manage to solve these tasks.  

An online web experiment was developed and implemented to gather data about how individuals solve geospatial micro-tasks. Statistical analysis is conducted on the collected data to answer the research questions. The author concludes that geospatial micro-tasks can be solved by inexperienced individuals and if the quality of the task results is important, fewer elements will increase the quality. 
%The experiment registered the participant's background to test if the quality of the solved micro-tasks differs between experienced and inexperienced participants. The tasks varied the number of elements that had to be handled at the same time to complete the micro-task. This approach was used to determine if the number has an influence on the quality of the solved micro-tasks. 
%Participants completed the same two micro-tasks in three different tasks. The tasks varied the number of elements that had to be handled to complete the micro-task. This approach was used to determine if the number has an influence on the quality of the solved micro-tasks. 

%The analyzes found significant evidence of inexperienced participants finishing the micro-tasks faster than the experienced. The quality of the task results did not differ between experts and non-experts. When examining the three different tasks in the experiment, the task containing the fewest elements had statistically better quality than the two other. There was also a difference in time spent completing the three tasks, but not enough to be significant. The author concludes that geospatial micro-tasks can be given to inexperienced individuals and if the quality of the task results is important, fewer elements will increase the quality. 

%Experts and non-experts did equally well on the micro-tasks, and the non-experts finished the micro-tasks faster. 
%This paper propose a method for extracting buildings in satellite photos. 
%The proposed network makes use of a digital surface model and multispectral satellite data. It 
