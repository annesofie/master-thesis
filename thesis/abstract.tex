\chapter*{Abstract}
\addcontentsline{toc}{chapter}{Abstract}%

This paper proposes to extend the usage of the micro-tasking method to also geospatial tasks. We see a huge potential in exploiting geospatial tasks with the micro-tasking method. Two research hypotheses are answered: 1) Inexperienced workers cannot solve micro-tasks containing geospatial data as good as experienced workers, 2) The fewer elements in a micro-task, the better the worker solves the task.

An online web experiment was developed and implemented to gather data about how individuals solve geospatial micro-tasks. Statistical analysis is conducted on the gather data to answer the research hypotheses. The experiment registered the individual's background to see if the quality of the solved micro-tasks differs between experienced and inexperienced participants. The tasks varied the number of elements that had to be handled at the same time to complete the micro-task. This approach was used to determine if the number has an influence on the quality of the solved micro-tasks. 
%Participants completed the same two micro-tasks in three different tasks. The tasks varied the number of elements that had to be handled to complete the micro-task. This approach was used to determine if the number has an influence on the quality of the solved micro-tasks. 

Statistical analysis found significant evidence of inexperienced participants finishing the micro-tasks faster. The quality of the completed micro-tasks did not differ when comparing experts and non-experts. When comparing the three different tasks in the experiment, the task containing the fewest elements had statistically better quality than the two other. There was also a difference in time spent completing the three tasks, but not enough to be significant.

%Experts and non-experts did equally well on the micro-tasks and the non-experts finished the micro-tasks faster. 
%This paper propose a method for extracting buildings in satellite photos. 
%The proposed network makes use of a digital surface model and multispectral satellite data. It 
